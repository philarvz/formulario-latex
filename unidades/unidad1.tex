% ==================== UNIDAD I ====================
% Funciones de varias variables

\section{Unidad I: Funciones de varias variables}

\subsection{Funciones de varias variables}

\begin{TemaBox}[Funciones de varias variables: estudio teórico]
Esta sección presenta un estudio teórico fundamental sobre las funciones de varias variables, abarcando conceptos esenciales como dominio, rango, y la distinción entre funciones explícitas e implícitas. Estos conceptos son la base para el análisis multivariable y aplicaciones en ciencias e ingeniería.
\end{TemaBox}

\paragraph{Introducción.}
Las \textbf{funciones de varias variables} son aquellas que dependen de dos o más variables independientes. A diferencia de las funciones de una variable \(y = f(x)\), estas funciones se expresan como \(z = f(x,y)\) para dos variables, o más generalmente como \(w = f(x_1, x_2, \ldots, x_n)\) para \(n\) variables.

Las funciones de varias variables permiten modelar fenómenos donde múltiples factores influyen simultáneamente en un resultado. Por ejemplo:
\begin{itemize}
  \item La temperatura \(T(x,y,z)\) en un punto del espacio tridimensional.
  \item El volumen \(V(r,h) = \pi r^2 h\) de un cilindro en función de su radio y altura.
  \item La producción \(P(L,K)\) de una empresa en función del trabajo \(L\) y capital \(K\) (función de producción en economía).
\end{itemize}

\subsubsection{Funciones escalares de varias variables}

\subsubsection{Dominio}
El \textbf{dominio} de una función de varias variables \(f(x,y)\) es el conjunto de todos los pares ordenados \((x,y)\) para los cuales la función está definida. Formalmente:
\[
\text{Dom}(f) = \{(x,y) \in \mathbb{R}^2 : f(x,y) \text{ está definida}\}
\]

Para funciones de tres o más variables, el dominio se extiende naturalmente a \(\mathbb{R}^3\) o \(\mathbb{R}^n\).

\textbf{¿Cuándo se aplica?}
\begin{itemize}
  \item Al identificar restricciones físicas o matemáticas (raíces cuadradas, denominadores, logaritmos).
  \item Para determinar la región del plano o espacio donde la función tiene sentido.
\end{itemize}

\textbf{Ejemplos ilustrativos:}
\begin{enumerate}
  \item \textbf{Función polinomial:} \(f(x,y) = x^2 + y^2\)
  
  El dominio es todo \(\mathbb{R}^2\), ya que no hay restricciones:
  \[\text{Dom}(f) = \mathbb{R}^2\]
  
  \item \textbf{Función con raíz cuadrada:} \(f(x,y) = \sqrt{9 - x^2 - y^2}\)
  
  La expresión dentro de la raíz debe ser no negativa:
  \[9 - x^2 - y^2 \geq 0 \Rightarrow x^2 + y^2 \leq 9\]
  El dominio es el disco cerrado de radio 3 centrado en el origen:
  \[\text{Dom}(f) = \{(x,y) : x^2 + y^2 \leq 9\}\]
  
  \item \textbf{Función con denominador:} \(f(x,y) = \frac{1}{x - y}\)
  
  El denominador no puede ser cero:
  \[x - y \neq 0 \Rightarrow x \neq y\]
  El dominio excluye la recta \(y = x\):
  \[\text{Dom}(f) = \{(x,y) \in \mathbb{R}^2 : x \neq y\}\]
\end{enumerate}

\subsubsection{Rango}
El \textbf{rango} (o imagen) de una función \(f(x,y)\) es el conjunto de todos los valores posibles que puede tomar la función. Formalmente:
\[
\text{Ran}(f) = \{z \in \mathbb{R} : z = f(x,y) \text{ para algún } (x,y) \in \text{Dom}(f)\}
\]

\textbf{¿Cuándo se aplica?}
\begin{itemize}
  \item Al determinar los valores de salida posibles de una función.
  \item Para analizar el comportamiento global de la función.
\end{itemize}

\textbf{Ejemplos ilustrativos:}
\begin{enumerate}
  \item \textbf{Paraboloide:} \(f(x,y) = x^2 + y^2\)
  
  Dado que \(x^2 \geq 0\) y \(y^2 \geq 0\), tenemos \(f(x,y) \geq 0\), y el mínimo se alcanza en \((0,0)\):
  \[\text{Ran}(f) = [0, +\infty)\]
  
  \item \textbf{Función con raíz:} \(f(x,y) = \sqrt{9 - x^2 - y^2}\)
  
  Como \(0 \leq x^2 + y^2 \leq 9\), tenemos:
  \[0 \leq 9 - x^2 - y^2 \leq 9 \Rightarrow 0 \leq \sqrt{9 - x^2 - y^2} \leq 3\]
  \[\text{Ran}(f) = [0, 3]\]
  
  \item \textbf{Función seno:} \(f(x,y) = \sin(x) + \cos(y)\)
  
  Dado que \(-1 \leq \sin(x) \leq 1\) y \(-1 \leq \cos(y) \leq 1\):
  \[\text{Ran}(f) = [-2, 2]\]
\end{enumerate}

\subsubsection{Funciones explícitas}
Una función se dice \textbf{explícita} cuando la variable dependiente está despejada en términos de las variables independientes. La forma general es:
\[
z = f(x,y)
\]

\textbf{¿Cuándo se aplican?}
\begin{itemize}
  \item Cuando se necesita evaluar directamente la función para valores específicos.
  \item Para graficar superficies en el espacio tridimensional.
  \item En cálculo de derivadas parciales de forma directa.
\end{itemize}

\textbf{Ejemplos ilustrativos:}
\begin{enumerate}
  \item \(z = x^2 + y^2\) — Paraboloide circular
  \item \(z = \sqrt{x^2 + y^2}\) — Cono
  \item \(z = \sin(x) \cos(y)\) — Superficie ondulatoria
  \item \(w = xyz\) — Función de tres variables
\end{enumerate}

En funciones explícitas, es directo calcular derivadas parciales:
\[
\frac{\partial z}{\partial x} = 2x, \quad \frac{\partial z}{\partial y} = 2y \quad \text{para } z = x^2 + y^2
\]

\subsubsection{Funciones implícitas}
Una función se dice \textbf{implícita} cuando no está despejada, sino que se define mediante una ecuación de la forma:
\[
F(x,y,z) = 0
\]

donde \(z\) no está aislada explícitamente.

\textbf{¿Cuándo se aplican?}
\begin{itemize}
  \item Cuando es difícil o imposible despejar la variable dependiente.
  \item En ecuaciones de superficies como esferas, elipsoides, hiperboloides.
  \item Para aplicar el teorema de la función implícita en análisis avanzado.
\end{itemize}

\textbf{Ejemplos ilustrativos:}
\begin{enumerate}
  \item \textbf{Esfera:} \(x^2 + y^2 + z^2 = 9\)
  
  Esta ecuación define \(z\) implícitamente. Si se despeja:
  \[z = \pm\sqrt{9 - x^2 - y^2}\]
  Se obtienen dos funciones explícitas (hemisferio superior e inferior).
  
  \item \textbf{Cilindro:} \(x^2 + y^2 = 4\)
  
  Define una superficie cilíndrica donde \(z\) puede tomar cualquier valor.
  
  \item \textbf{Ecuación general:} \(x^2 + y^2 - z^2 = 1\)
  
  Hiperboloide de una hoja, difícil de expresar explícitamente.
\end{enumerate}

\textbf{Derivación implícita:}
Para funciones implícitas \(F(x,y,z) = 0\), podemos calcular derivadas parciales usando:
\[
\frac{\partial z}{\partial x} = -\frac{\frac{\partial F}{\partial x}}{\frac{\partial F}{\partial z}}, \quad
\frac{\partial z}{\partial y} = -\frac{\frac{\partial F}{\partial y}}{\frac{\partial F}{\partial z}}
\]

\textbf{Ejemplo aplicado:}
Para \(x^2 + y^2 + z^2 = 9\), con \(F(x,y,z) = x^2 + y^2 + z^2 - 9\):
\[
\frac{\partial F}{\partial x} = 2x, \quad \frac{\partial F}{\partial z} = 2z
\]
\[
\frac{\partial z}{\partial x} = -\frac{2x}{2z} = -\frac{x}{z}
\]

\paragraph{Comparación: Explícitas vs. Implícitas.}
\begin{center}
\begin{tabular}{|l|l|l|}
\hline
\textbf{Aspecto} & \textbf{Función Explícita} & \textbf{Función Implícita} \\
\hline
Forma & \(z = f(x,y)\) & \(F(x,y,z) = 0\) \\
\hline
Evaluación & Directa & Requiere despejar o métodos numéricos \\
\hline
Derivadas & Directas & Requiere derivación implícita \\
\hline
Ejemplos & \(z = x^2 + y^2\) & \(x^2 + y^2 + z^2 = 9\) \\
\hline
\end{tabular}
\end{center}

\paragraph{Aplicaciones prácticas.}
Las funciones de varias variables tienen aplicaciones en:
\begin{itemize}
  \item \textbf{Física:} Campos escalares (temperatura, presión, potencial eléctrico).
  \item \textbf{Economía:} Funciones de utilidad \(U(x,y)\), funciones de producción Cobb-Douglas \(P(L,K) = AL^\alpha K^\beta\).
  \item \textbf{Ingeniería:} Análisis de estructuras, distribución de esfuerzos, optimización de diseños.
  \item \textbf{Estadística:} Regresión multivariable, funciones de densidad conjunta.
\end{itemize}

% ==================== UNIDAD 2 ====================
% Planos y superficies

\subsection{Planos y superficies}

\subsubsection{Curvas de nivel: Planos, superficies cuadráticas (elipsoides, cono, paraboloides, hiperboloides) -- Graficación}

\begin{TemaBox}[Descripción]
Una \textbf{curva de nivel} de una función \(z=f(x,y)\) es el conjunto de puntos \((x,y)\) en el plano \(xy\) tales que
\(f(x,y)=k\) para una constante fija \(k\). Geométricamente, surge al intersectar la \emph{superficie} \(z=f(x,y)\) con el
\emph{plano horizontal} \(z=k\) y proyectar el contorno sobre el plano \(xy\).
\end{TemaBox}

\begin{InfoBox}
\Meta{Propósito}{Visualizar funciones de dos variables mediante sus curvas de nivel y relacionarlas con cortes horizontales de la superficie.}
\Meta{Competencias}{Modelación, representación gráfica y comunicación matemática.}
\end{InfoBox}

\paragraph{Teoría breve.}
Para \(f(x,y)=x^2+y^2\), las curvas de nivel satisfacen \(x^2+y^2=k\).  
Cada \(k>0\) produce una circunferencia de radio \(r=\sqrt{k}\) centrada en el origen.  
El patrón de circunferencias concéntricas refleja que la superficie \(z=x^2+y^2\) es un paraboloide circular.

% ---------- Figura 2D: Curvas de nivel (CORREGIDA) ----------
\begin{figure}[h!]
\centering
\begin{tikzpicture}
  \begin{axis}[
      axis equal image,
      width=12cm,height=9cm,
      axis lines=middle, xlabel={$x$}, ylabel={$y$},
      xmin=-3.2, xmax=3.2, ymin=-3.2, ymax=3.2,
      grid=both, minor tick num=1, grid style={gray!20},
      title={Curvas de nivel de \(f(x,y)=x^2+y^2\)}
  ]
    % radios: sqrt(2), sqrt(3), 2, sqrt(5), sqrt(6)
    \def\rrA{1.414}
    \def\rrB{1.732}
    \def\rrC{2.000}
    \def\rrD{2.236}
    \def\rrE{2.449}

    \addplot[brand,thick,domain=0:360,samples=300] ({\rrA*cos(x)},{\rrA*sin(x)});
    \addlegendentry{$k=2$}

    \addplot[brand!70!black,thick,domain=0:360,samples=300] ({\rrB*cos(x)},{\rrB*sin(x)});
    \addlegendentry{$k=3$}

    \addplot[Plum,thick,domain=0:360,samples=300] ({\rrC*cos(x)},{\rrC*sin(x)});
    \addlegendentry{$k=4$}

    \addplot[Cyan,thick,domain=0:360,samples=300] ({\rrD*cos(x)},{\rrD*sin(x)});
    \addlegendentry{$k=5$}

    \addplot[Sepia,thick,domain=0:360,samples=300] ({\rrE*cos(x)},{\rrE*sin(x)});
    \addlegendentry{$k=6$}
  \end{axis}
\end{tikzpicture}
\caption{Curvas de nivel concéntricas para \(k=2,3,4,5,6\).}
\label{fig:niveles2d}
\end{figure}

% ---------- Figura 3D: Superficie + cortes (SIMPLIFICADA) ----------
\begin{figure}[h!]
\centering
\begin{tikzpicture}
  \begin{axis}[
      view={35}{20},
      width=14cm, height=9cm,
      xlabel={$x$}, ylabel={$y$}, zlabel={$z$},
      domain=-3:3, y domain=-3:3,
      samples=30, samples y=30,
      colormap/viridis,
      title={Superficie \(z=x^2+y^2\) con planos \(z=\text{const}\)}
  ]
    % Superficie principal
    \addplot3[surf,opacity=0.7] {x^2 + y^2};
    
    % Circunferencias de nivel en 3D
    \addplot3[thick,brand,domain=0:360,samples=100] ({sqrt(2)*cos(x)},{sqrt(2)*sin(x)},{2});
    \addplot3[thick,brand,domain=0:360,samples=100] ({sqrt(4)*cos(x)},{sqrt(4)*sin(x)},{4});
    \addplot3[thick,brand,domain=0:360,samples=100] ({sqrt(6)*cos(x)},{sqrt(6)*sin(x)},{6});
  \end{axis}
\end{tikzpicture}
\caption{Paraboloide \(z=x^2+y^2\) y cortes horizontales \(z=\text{const}\).}
\label{fig:niveles3d}
\end{figure}

\paragraph{Ejemplos guiados.}
\begin{itemize}
  \item \(k=2 \Rightarrow r=\sqrt{2}\approx1.41\)
  \item \(k=3 \Rightarrow r=\sqrt{3}\approx1.73\)
  \item \(k=4 \Rightarrow r=2.00\)
  \item \(k=5 \Rightarrow r=\sqrt{5}\approx2.24\)
  \item \(k=6 \Rightarrow r=\sqrt{6}\approx2.45\)
\end{itemize}

\paragraph{Ejercicios.}
\begin{EjercicioBox}[1) Identificación de curvas]
Dibuja las curvas de nivel para \(k=2,3,4,5,6\) y anota sus radios.
\end{EjercicioBox}

\begin{EjercicioBox}[2) Intersección con planos]
Encuentra las curvas resultantes de los cortes con los planos \(z=k\) y describe su forma.
\end{EjercicioBox}

\begin{EjercicioBox}[3) Normalización]
Transforma \(x^2+y^2=k\) en la forma \(\frac{x^2}{a^2}+\frac{y^2}{b^2}=1\) y deduce que \(a=b=\sqrt{k}\).
\end{EjercicioBox}

% ==================== UNIDAD 3 ====================
% Límites y continuidad en funciones de 3 variables

\subsection{Límites y continuidad en funciones de 3 variables}

\begin{TemaBox}[Descripción]
Un \textbf{límite} describe el comportamiento de una función \(f(x)\) cuando la variable independiente \(x\) se aproxima a un valor específico \(a\). La notación formal es:
\[
\lim_{x\to a} f(x) = L
\]
Esto significa que los valores de \(f(x)\) se acercan arbitrariamente a \(L\) cuando \(x\) se acerca a \(a\). Los límites son fundamentales para definir la \textbf{continuidad} de una función y establecer las bases del cálculo diferencial e integral.
\end{TemaBox}

\begin{InfoBox}
\Meta{Propósito}{Comprender el comportamiento de una función cuando la variable se aproxima a un valor determinado.}
\Meta{Competencias}{Razonamiento lógico, análisis algebraico y comprensión de la continuidad.}
\end{InfoBox}

\subsubsection{Definición y conceptos}

\paragraph{Definición formal.}
Decimos que el límite de \(f(x)\) cuando \(x\) tiende a \(a\) es \(L\) si:
\[
\lim_{x\to a} f(x) = L
\]
siempre que para cada \(\varepsilon > 0\) exista un \(\delta > 0\) tal que si \(0 < |x-a| < \delta\), entonces \(|f(x)-L| < \varepsilon\).

\paragraph{Indeterminaciones comunes.}
Algunas formas indeterminadas requieren técnicas algebraicas para resolverse:
\begin{itemize}
  \item \(\frac{0}{0}\) — Factorización, racionalización o L'Hôpital
  \item \(\frac{\infty}{\infty}\) — Simplificación algebraica
  \item \(0\cdot\infty\), \(\infty - \infty\), \(1^\infty\), \(0^0\), \(\infty^0\) — Transformaciones logarítmicas o algebraicas
\end{itemize}

\paragraph{Continuidad.}
Una función \(f(x)\) es \textbf{continua} en \(x=a\) si:
\[
\lim_{x\to a} f(x) = f(a)
\]
Es decir, el límite existe, la función está definida en \(a\), y ambos valores coinciden.

\subsubsection{Racionalización de límites}

Cuando aparecen raíces cuadradas en expresiones que producen indeterminaciones \(\frac{0}{0}\), podemos usar la técnica de \textbf{multiplicar por el conjugado}.

\paragraph{Ejemplo resuelto.}
Calcular:
\[
\lim_{x\to 0} \frac{\sqrt{9+x}-3}{x}
\]

\textbf{Solución:}
\begin{align*}
\lim_{x\to 0} \frac{\sqrt{9+x}-3}{x} &= \lim_{x\to 0} \frac{\sqrt{9+x}-3}{x} \cdot \frac{\sqrt{9+x}+3}{\sqrt{9+x}+3} \\[0.5em]
&= \lim_{x\to 0} \frac{(9+x)-9}{x(\sqrt{9+x}+3)} \\[0.5em]
&= \lim_{x\to 0} \frac{x}{x(\sqrt{9+x}+3)} \\[0.5em]
&= \lim_{x\to 0} \frac{1}{\sqrt{9+x}+3} \\[0.5em]
&= \frac{1}{\sqrt{9}+3} = \frac{1}{6}
\end{align*}

Por lo tanto, \(\boxed{\lim_{x\to 0} \frac{\sqrt{9+x}-3}{x} = \frac{1}{6}}\).

\subsubsection{Ejercicios prácticos}

\begin{EjercicioBox}[Ejercicio 1]
Calcular:
\[
\lim_{x\to -3} \frac{x+3}{x^2-9}
\]

\textbf{Solución:}
\begin{align*}
\lim_{x\to -3} \frac{x+3}{x^2-9} &= \lim_{x\to -3} \frac{x+3}{(x-3)(x+3)} \\[0.5em]
&= \lim_{x\to -3} \frac{1}{x-3} \\[0.5em]
&= \frac{1}{-3-3} = \frac{1}{-6} = \boxed{-\frac{1}{6}}
\end{align*}
\end{EjercicioBox}

\begin{EjercicioBox}[Ejercicio 2]
Calcular:
\[
\lim_{x\to 1} \frac{x^2-4x+3}{x^2+3x-4}
\]

\textbf{Solución:}
\begin{align*}
\lim_{x\to 1} \frac{x^2-4x+3}{x^2+3x-4} &= \lim_{x\to 1} \frac{(x-1)(x-3)}{(x-1)(x+4)} \\[0.5em]
&= \lim_{x\to 1} \frac{x-3}{x+4} \\[0.5em]
&= \frac{1-3}{1+4} = \frac{-2}{5} = \boxed{-\frac{2}{5}}
\end{align*}
\end{EjercicioBox}

\begin{EjercicioBox}[Ejercicio 3]
Calcular:
\[
\lim_{x\to 0} \frac{x^3-2x^2}{3x^2}
\]

\textbf{Solución:}
\begin{align*}
\lim_{x\to 0} \frac{x^3-2x^2}{3x^2} &= \lim_{x\to 0} \frac{x^2(x-2)}{3x^2} \\[0.5em]
&= \lim_{x\to 0} \frac{x-2}{3} \\[0.5em]
&= \frac{0-2}{3} = \boxed{-\frac{2}{3}}
\end{align*}
\end{EjercicioBox}

\begin{EjercicioBox}[Ejercicio 4 (Racionalización)]
Calcular:
\[
\lim_{x\to 0} \frac{\sqrt{4-2x+x^2}-2}{x}
\]

\textbf{Solución:}
\begin{align*}
\lim_{x\to 0} \frac{\sqrt{4-2x+x^2}-2}{x} &= \lim_{x\to 0} \frac{\sqrt{4-2x+x^2}-2}{x} \cdot \frac{\sqrt{4-2x+x^2}+2}{\sqrt{4-2x+x^2}+2} \\[0.5em]
&= \lim_{x\to 0} \frac{(4-2x+x^2)-4}{x(\sqrt{4-2x+x^2}+2)} \\[0.5em]
&= \lim_{x\to 0} \frac{-2x+x^2}{x(\sqrt{4-2x+x^2}+2)} \\[0.5em]
&= \lim_{x\to 0} \frac{x(-2+x)}{x(\sqrt{4-2x+x^2}+2)} \\[0.5em]
&= \lim_{x\to 0} \frac{-2+x}{\sqrt{4-2x+x^2}+2} \\[0.5em]
&= \frac{-2+0}{\sqrt{4}+2} = \frac{-2}{4} = \boxed{-\frac{1}{2}}
\end{align*}
\end{EjercicioBox}

\begin{EjercicioBox}[Ejercicio 5 (Racionalización)]
Calcular:
\[
\lim_{x\to 1} \frac{\sqrt{x+3}-2}{x-1}
\]

\textbf{Solución:}
\begin{align*}
\lim_{x\to 1} \frac{\sqrt{x+3}-2}{x-1} &= \lim_{x\to 1} \frac{\sqrt{x+3}-2}{x-1} \cdot \frac{\sqrt{x+3}+2}{\sqrt{x+3}+2} \\[0.5em]
&= \lim_{x\to 1} \frac{(x+3)-4}{(x-1)(\sqrt{x+3}+2)} \\[0.5em]
&= \lim_{x\to 1} \frac{x-1}{(x-1)(\sqrt{x+3}+2)} \\[0.5em]
&= \lim_{x\to 1} \frac{1}{\sqrt{x+3}+2} \\[0.5em]
&= \frac{1}{\sqrt{4}+2} = \frac{1}{4} = \boxed{\frac{1}{4}}
\end{align*}
\end{EjercicioBox}

\paragraph{Gráfica ilustrativa.}
La siguiente figura muestra la función \(f(x)=\frac{x^2-1}{x-1}\) y su comportamiento cerca de \(x=1\), donde existe una discontinuidad removible:

\begin{figure}[h!]
\centering
\begin{tikzpicture}
  \begin{axis}[
      width=12cm, height=8cm,
      axis lines=middle,
      xlabel={$x$}, ylabel={$y$},
      xmin=-0.5, xmax=3,
      ymin=-0.5, ymax=4,
      grid=both, grid style={gray!20},
      title={Función \(f(x)=\frac{x^2-1}{x-1}\) con discontinuidad removible en \(x=1\)}
  ]
    % Función simplificada f(x) = x+1, excepto en x=1
    \addplot[brand, thick, domain=-0.5:0.95, samples=100] {x+1};
    \addplot[brand, thick, domain=1.05:3, samples=100] {x+1};
    
    % Círculo hueco en el punto discontinuo
    \addplot[only marks, mark=o, mark size=3pt, brand, fill=white, thick] coordinates {(1,2)};
    
    % Línea punteada indicando el límite
    \addplot[dashed, gray, thick] coordinates {(1,0) (1,2)};
    \addplot[dashed, gray, thick] coordinates {(0,2) (1,2)};
    
    % Etiqueta del límite
    \node[anchor=west] at (axis cs:1.1,2.3) {$\lim_{x\to 1} f(x) = 2$};
  \end{axis}
\end{tikzpicture}
\caption{Visualización del límite: aunque \(f(1)\) no está definida, \(\lim_{x\to 1} f(x) = 2\).}
\label{fig:limite}
\end{figure}

\paragraph{Conclusión.}
\begin{quote}
\textit{``El concepto de límite es fundamental en el cálculo diferencial y sirve como base para las derivadas y la continuidad.''}\\
— Stewart, J. (2016). \textit{Cálculo de varias variables} (8a ed.). Cengage Learning.
\end{quote}

