% ==================== UNIDAD I ====================
% Funciones de varias variables

\section{Unidad I: Funciones de varias variables}

\subsection{Funciones de varias variables}

\begin{TemaBox}[Funciones de varias variables: estudio teórico]
Esta sección presenta un estudio teórico fundamental sobre las funciones de varias variables, abarcando conceptos esenciales como dominio, rango, y la distinción entre funciones explícitas e implícitas. Estos conceptos son la base para el análisis multivariable y aplicaciones en ciencias e ingeniería.
\end{TemaBox}

\paragraph{Introducción.}
Las \textbf{funciones de varias variables} son aquellas que dependen de dos o más variables independientes. A diferencia de las funciones de una variable \(y = f(x)\), estas funciones se expresan como \(z = f(x,y)\) para dos variables, o más generalmente como \(w = f(x_1, x_2, \ldots, x_n)\) para \(n\) variables.

Las funciones de varias variables permiten modelar fenómenos donde múltiples factores influyen simultáneamente en un resultado. Por ejemplo:
\begin{itemize}
  \item La temperatura \(T(x,y,z)\) en un punto del espacio tridimensional.
  \item El volumen \(V(r,h) = \pi r^2 h\) de un cilindro en función de su radio y altura.
  \item La producción \(P(L,K)\) de una empresa en función del trabajo \(L\) y capital \(K\) (función de producción en economía).
\end{itemize}

\subsubsection{Funciones escalares de varias variables}

\subsubsection{Dominio}
El \textbf{dominio} de una función de varias variables \(f(x,y)\) es el conjunto de todos los pares ordenados \((x,y)\) para los cuales la función está definida. Formalmente:
\[
\text{Dom}(f) = \{(x,y) \in \mathbb{R}^2 : f(x,y) \text{ está definida}\}
\]

Para funciones de tres o más variables, el dominio se extiende naturalmente a \(\mathbb{R}^3\) o \(\mathbb{R}^n\).

\textbf{¿Cuándo se aplica?}
\begin{itemize}
  \item Al identificar restricciones físicas o matemáticas (raíces cuadradas, denominadores, logaritmos).
  \item Para determinar la región del plano o espacio donde la función tiene sentido.
\end{itemize}

\textbf{Ejemplos ilustrativos:}
\begin{enumerate}
  \item \textbf{Función polinomial:} \(f(x,y) = x^2 + y^2\)
  
  El dominio es todo \(\mathbb{R}^2\), ya que no hay restricciones:
  \[\text{Dom}(f) = \mathbb{R}^2\]
  
  \item \textbf{Función con raíz cuadrada:} \(f(x,y) = \sqrt{9 - x^2 - y^2}\)
  
  La expresión dentro de la raíz debe ser no negativa:
  \[9 - x^2 - y^2 \geq 0 \Rightarrow x^2 + y^2 \leq 9\]
  El dominio es el disco cerrado de radio 3 centrado en el origen:
  \[\text{Dom}(f) = \{(x,y) : x^2 + y^2 \leq 9\}\]
  
  \item \textbf{Función con denominador:} \(f(x,y) = \frac{1}{x - y}\)
  
  El denominador no puede ser cero:
  \[x - y \neq 0 \Rightarrow x \neq y\]
  El dominio excluye la recta \(y = x\):
  \[\text{Dom}(f) = \{(x,y) \in \mathbb{R}^2 : x \neq y\}\]
\end{enumerate}

\subsubsection{Rango}
El \textbf{rango} (o imagen) de una función \(f(x,y)\) es el conjunto de todos los valores posibles que puede tomar la función. Formalmente:
\[
\text{Ran}(f) = \{z \in \mathbb{R} : z = f(x,y) \text{ para algún } (x,y) \in \text{Dom}(f)\}
\]

\textbf{¿Cuándo se aplica?}
\begin{itemize}
  \item Al determinar los valores de salida posibles de una función.
  \item Para analizar el comportamiento global de la función.
\end{itemize}

\textbf{Ejemplos ilustrativos:}
\begin{enumerate}
  \item \textbf{Paraboloide:} \(f(x,y) = x^2 + y^2\)
  
  Dado que \(x^2 \geq 0\) y \(y^2 \geq 0\), tenemos \(f(x,y) \geq 0\), y el mínimo se alcanza en \((0,0)\):
  \[\text{Ran}(f) = [0, +\infty)\]
  
  \item \textbf{Función con raíz:} \(f(x,y) = \sqrt{9 - x^2 - y^2}\)
  
  Como \(0 \leq x^2 + y^2 \leq 9\), tenemos:
  \[0 \leq 9 - x^2 - y^2 \leq 9 \Rightarrow 0 \leq \sqrt{9 - x^2 - y^2} \leq 3\]
  \[\text{Ran}(f) = [0, 3]\]
  
  \item \textbf{Función seno:} \(f(x,y) = \sin(x) + \cos(y)\)
  
  Dado que \(-1 \leq \sin(x) \leq 1\) y \(-1 \leq \cos(y) \leq 1\):
  \[\text{Ran}(f) = [-2, 2]\]
\end{enumerate}

\subsubsection{Funciones explícitas}
Una función se dice \textbf{explícita} cuando la variable dependiente está despejada en términos de las variables independientes. La forma general es:
\[
z = f(x,y)
\]

\textbf{¿Cuándo se aplican?}
\begin{itemize}
  \item Cuando se necesita evaluar directamente la función para valores específicos.
  \item Para graficar superficies en el espacio tridimensional.
  \item En cálculo de derivadas parciales de forma directa.
\end{itemize}

\textbf{Ejemplos ilustrativos:}
\begin{enumerate}
  \item \(z = x^2 + y^2\) — Paraboloide circular
  \item \(z = \sqrt{x^2 + y^2}\) — Cono
  \item \(z = \sin(x) \cos(y)\) — Superficie ondulatoria
  \item \(w = xyz\) — Función de tres variables
\end{enumerate}

En funciones explícitas, es directo calcular derivadas parciales:
\[
\frac{\partial z}{\partial x} = 2x, \quad \frac{\partial z}{\partial y} = 2y \quad \text{para } z = x^2 + y^2
\]

\subsubsection{Ejercicios: Dominio y rango de funciones de una variable}

Encuentra el dominio y rango de las siguientes funciones:

\begin{enumerate}
  \item \(f(x) = x^2 + 3x + 2\)
  
  \textbf{Solución:}
  \begin{itemize}
    \item \textbf{Dominio:} Es una función polinomial, sin restricciones. \(\text{Dom}(f) = \mathbb{R}\) o \((-\infty, +\infty)\)
    \item \textbf{Rango:} Completando cuadrados: \(f(x) = (x + \frac{3}{2})^2 + 2 - \frac{9}{4} = (x + \frac{3}{2})^2 - \frac{1}{4}\)
    
    El mínimo ocurre en \(x = -\frac{3}{2}\), con valor \(f(-\frac{3}{2}) = -\frac{1}{4}\)
    
    \(\text{Ran}(f) = [-\frac{1}{4}, +\infty)\)
  \end{itemize}
  
  \item \(f(x) = \frac{x+2}{x+3}\)
  
  \textbf{Solución:}
  \begin{itemize}
    \item \textbf{Dominio:} El denominador no puede ser cero: \(x + 3 \neq 0 \Rightarrow x \neq -3\)
    
    \(\text{Dom}(f) = \mathbb{R} \setminus \{-3\}\) o \((-\infty, -3) \cup (-3, +\infty)\)
    
    \item \textbf{Rango:} Despejando \(x\) en términos de \(y\):
    \[y(x+3) = x+2 \Rightarrow yx + 3y = x + 2 \Rightarrow yx - x = 2 - 3y \Rightarrow x(y-1) = 2-3y\]
    \[x = \frac{2-3y}{y-1}\]
    
    Para que \(x\) exista, necesitamos \(y - 1 \neq 0 \Rightarrow y \neq 1\)
    
    \(\text{Ran}(f) = \mathbb{R} \setminus \{1\}\) o \((-\infty, 1) \cup (1, +\infty)\)
  \end{itemize}
  
  \item \(f(x) = \frac{2x-1}{x^2-x-12}\)
  
  \textbf{Solución:}
  \begin{itemize}
    \item \textbf{Dominio:} Factorizamos el denominador: \(x^2 - x - 12 = (x-4)(x+3)\)
    
    El denominador es cero cuando \(x = 4\) o \(x = -3\)
    
    \(\text{Dom}(f) = \mathbb{R} \setminus \{-3, 4\}\) o \((-\infty, -3) \cup (-3, 4) \cup (4, +\infty)\)
    
    \item \textbf{Rango:} Por el método de despejar \(y\):
    \[y(x^2-x-12) = 2x-1 \Rightarrow yx^2 - yx - 12y = 2x - 1\]
    \[yx^2 - (y+2)x + (12y-1) = 0\]
    
    Si \(y = 0\): \(-2x - 1 = 0 \Rightarrow x = -\frac{1}{2}\) (válido)
    
    Si \(y \neq 0\): Para que exista \(x\) real, el discriminante debe ser \(\geq 0\):
    \[\Delta = (y+2)^2 - 4y(12y-1) \geq 0\]
    \[\Delta = y^2 + 4y + 4 - 48y^2 + 4y \geq 0\]
    \[-47y^2 + 8y + 4 \geq 0 \Rightarrow 47y^2 - 8y - 4 \leq 0\]
    
    \(\text{Ran}(f) = [-\frac{2}{47}, \frac{4}{47}]\) (aproximadamente)
  \end{itemize}
  
  \item \(f(x) = \sqrt{x} + 3\)
  
  \textbf{Solución:}
  \begin{itemize}
    \item \textbf{Dominio:} La expresión dentro de la raíz debe ser no negativa: \(x \geq 0\)
    
    \(\text{Dom}(f) = [0, +\infty)\)
    
    \item \textbf{Rango:} El mínimo ocurre cuando \(x = 0\): \(f(0) = \sqrt{0} + 3 = 3\)
    
    A medida que \(x\) aumenta, \(\sqrt{x}\) también aumenta sin límite.
    
    \(\text{Ran}(f) = [3, +\infty)\)
  \end{itemize}
  
  \item \(f(x) = \sqrt{x+8}\)
  
  \textbf{Solución:}
  \begin{itemize}
    \item \textbf{Dominio:} La expresión dentro de la raíz debe ser no negativa: \(x + 8 \geq 0 \Rightarrow x \geq -8\)
    
    \(\text{Dom}(f) = [-8, +\infty)\)
    
    \item \textbf{Rango:} El mínimo ocurre cuando \(x = -8\): \(f(-8) = \sqrt{-8+8} = \sqrt{0} = 0\)
    
    A medida que \(x\) aumenta, \(\sqrt{x+8}\) también aumenta sin límite.
    
    \(\text{Ran}(f) = [0, +\infty)\)
  \end{itemize}
\end{enumerate}

\subsubsection{Ejercicios: Variables dependientes e independientes}

Identifica la variable dependiente e independiente en las siguientes expresiones:

\begin{enumerate}
  \item \(y = 2 + x^2\)
  
  \textbf{Respuesta:} \textbf{D:} \(y\) \quad \textbf{I:} \(x\)
  
  \item \(z = y + 9\)
  
  \textbf{Respuesta:} \textbf{D:} \(z\) \quad \textbf{I:} \(y\)
  
  \item \(y = 2x + 3x^2 + 9\)
  
  \textbf{Respuesta:} \textbf{D:} \(y\) \quad \textbf{I:} \(x\)
  
  \item \(t^2 + 9t = x\)
  
  \textbf{Respuesta:} \textbf{D:} \(x\) \quad \textbf{I:} \(t\)
  
  (Equivalentemente: \(x = t^2 + 9t\))
  
  \item \(9x + 20 = z\)
  
  \textbf{Respuesta:} \textbf{D:} \(z\) \quad \textbf{I:} \(x\)
  
  (Equivalentemente: \(z = 9x + 20\))
  
  \item \(t^5 - y^4 + 2x = z\)
  
  \textbf{Respuesta:} \textbf{D:} \(z\) \quad \textbf{I:} \(t, y, x\) (tres variables independientes)
  
  (Equivalentemente: \(z = t^5 - y^4 + 2x\))
  
  \item \(y^2 + 2x + 8y = t\)
  
  \textbf{Respuesta:} \textbf{D:} \(t\) \quad \textbf{I:} \(x, y\) (dos variables independientes)
  
  (Equivalentemente: \(t = y^2 + 2x + 8y\))
\end{enumerate}

\subsubsection{Funciones implícitas}
Una función se dice \textbf{implícita} cuando no está despejada, sino que se define mediante una ecuación de la forma:
\[
F(x,y,z) = 0
\]

donde \(z\) no está aislada explícitamente.

\textbf{¿Cuándo se aplican?}
\begin{itemize}
  \item Cuando es difícil o imposible despejar la variable dependiente.
  \item En ecuaciones de superficies como esferas, elipsoides, hiperboloides.
  \item Para aplicar el teorema de la función implícita en análisis avanzado.
\end{itemize}

\textbf{Ejemplos ilustrativos:}
\begin{enumerate}
  \item \textbf{Esfera:} \(x^2 + y^2 + z^2 = 9\)
  
  Esta ecuación define \(z\) implícitamente. Si se despeja:
  \[z = \pm\sqrt{9 - x^2 - y^2}\]
  Se obtienen dos funciones explícitas (hemisferio superior e inferior).
  
  \item \textbf{Cilindro:} \(x^2 + y^2 = 4\)
  
  Define una superficie cilíndrica donde \(z\) puede tomar cualquier valor.
  
  \item \textbf{Ecuación general:} \(x^2 + y^2 - z^2 = 1\)
  
  Hiperboloide de una hoja, difícil de expresar explícitamente.
\end{enumerate}

\textbf{Derivación implícita:}
Para funciones implícitas \(F(x,y,z) = 0\), podemos calcular derivadas parciales usando:
\[
\frac{\partial z}{\partial x} = -\frac{\frac{\partial F}{\partial x}}{\frac{\partial F}{\partial z}}, \quad
\frac{\partial z}{\partial y} = -\frac{\frac{\partial F}{\partial y}}{\frac{\partial F}{\partial z}}
\]

\textbf{Ejemplo aplicado:}
Para \(x^2 + y^2 + z^2 = 9\), con \(F(x,y,z) = x^2 + y^2 + z^2 - 9\):
\[
\frac{\partial F}{\partial x} = 2x, \quad \frac{\partial F}{\partial z} = 2z
\]
\[
\frac{\partial z}{\partial x} = -\frac{2x}{2z} = -\frac{x}{z}
\]

\paragraph{Comparación: Explícitas vs. Implícitas.}
\begin{center}
\begin{tabular}{|l|l|l|}
\hline
\textbf{Aspecto} & \textbf{Función Explícita} & \textbf{Función Implícita} \\
\hline
Forma & \(z = f(x,y)\) & \(F(x,y,z) = 0\) \\
\hline
Evaluación & Directa & Requiere despejar o métodos numéricos \\
\hline
Derivadas & Directas & Requiere derivación implícita \\
\hline
Ejemplos & \(z = x^2 + y^2\) & \(x^2 + y^2 + z^2 = 9\) \\
\hline
\end{tabular}
\end{center}

\paragraph{Aplicaciones prácticas.}
Las funciones de varias variables tienen aplicaciones en:
\begin{itemize}
  \item \textbf{Física:} Campos escalares (temperatura, presión, potencial eléctrico).
  \item \textbf{Economía:} Funciones de utilidad \(U(x,y)\), funciones de producción Cobb-Douglas \(P(L,K) = AL^\alpha K^\beta\).
  \item \textbf{Ingeniería:} Análisis de estructuras, distribución de esfuerzos, optimización de diseños.
  \item \textbf{Estadística:} Regresión multivariable, funciones de densidad conjunta.
\end{itemize}

% ==================== UNIDAD 2 ====================
% Planos y superficies

\subsection{Planos y superficies}

\subsubsection{Curvas de nivel: Planos, superficies cuadráticas (elipsoides, cono, paraboloides, hiperboloides) -- Graficación}

\begin{TemaBox}[Descripción]
Una \textbf{curva de nivel} de una función \(z=f(x,y)\) es el conjunto de puntos \((x,y)\) en el plano \(xy\) tales que
\(f(x,y)=k\) para una constante fija \(k\). Geométricamente, surge al intersectar la \emph{superficie} \(z=f(x,y)\) con el
\emph{plano horizontal} \(z=k\) y proyectar el contorno sobre el plano \(xy\).

Las curvas de nivel permiten describir el comportamiento de la función sin necesidad de visualizar la superficie completa
en tres dimensiones: cada valor de \(k\) produce una figura distinta que representa cómo “sube” o “baja” la superficie.
Dependiendo de la forma de \(f(x,y)\), estas curvas pueden ser líneas, círculos, elipses, hipérbolas u otras curvas
características. Conforme \(k\) varía, las curvas muestran la estructura global de la superficie y permiten identificar
crestas, valles, simetrías, pendientes e incluso puntos críticos. Por ello, son una herramienta fundamental en el análisis
de funciones multivariables.
\end{TemaBox}

\begin{InfoBox}
\Meta{Propósito}{Visualizar funciones de dos variables mediante sus curvas de nivel y relacionarlas con cortes horizontales de la superficie.}
\end{InfoBox}

\subsubsection*{Planos}

\begin{TemaBox}[Descripción]
Un \textbf{plano} en el espacio tridimensional se describe mediante una ecuación lineal de la forma
\[
Ax + By + Cz = D,
\]
donde \(A\), \(B\) y \(C\) determinan la dirección del vector normal al plano y \(D\) controla su posición respecto al origen.
Las curvas de nivel asociadas a un plano provienen de la intersección con planos horizontales \(z=k\), lo cual produce siempre
rectas en el plano \(xy\). Por ello, los planos generan \textbf{curvas de nivel lineales} igualmente espaciadas.
\end{TemaBox}

\paragraph{Curvas de nivel de un plano.}
Dado un plano expresado como función:
\[
z = f(x,y) = ax + by + c,
\]
la curva de nivel para \(z = k\) resulta en:
\[
ax + by + c = k \quad \Rightarrow \quad ax + by = k - c,
\]
que es la ecuación de una recta en el plano \(xy\).  
A medida que \(k\) varía, estas rectas son paralelas entre sí, lo que refleja que la superficie tiene inclinación constante.

\begin{figure}[h!]
\centering
\includegraphics[width=0.75\textwidth]{imagenes/plano.png}
\caption{Ejemplo de plano}
\label{fig:plano}
\end{figure}

\paragraph{Ejemplo completo.}
Considera el plano
\[
z = 2x - y + 1.
\]

\textbf{1) Curvas de nivel:}  
Para \(z=k\), se obtiene:
\[
2x - y + 1 = k \quad \Rightarrow \quad y = 2x + 1 - k.
\]
Las curvas de nivel para distintos valores de \(k\) son rectas paralelas con pendiente \(2\).

\textbf{2) Ejemplo numérico:}
\[
\begin{aligned}
k = 0: &\quad y = 2x + 1 \\
k = 1: &\quad y = 2x \\
k = 2: &\quad y = 2x - 1 \\
k = 3: &\quad y = 2x - 2
\end{aligned}
\]

\textbf{3) Interpretación geométrica:}  
Cada incremento en \(k\) desplaza la recta hacia abajo en el eje \(y\), manteniendo siempre la misma pendiente.  
Esto indica que el plano tiene \textbf{pendiente uniforme} y no presenta curvatura.

\textbf{Conclusión:}  
Los planos producen curvas de nivel lineales y paralelas, lo cual refleja su naturaleza como superficies con tasa de cambio constante en todas direcciones.


\subsubsection*{Elipsoides}

\begin{TemaBox}[Descripción]
Un \textbf{elipsoide} es una superficie cerrada en el espacio tridimensional que generaliza la forma de una esfera.
Su ecuación cartesiana estándar es
\[
\frac{x^2}{a^2} + \frac{y^2}{b^2} + \frac{z^2}{c^2} = 1,
\]
donde \(a\), \(b\) y \(c\) son los semiejes en las direcciones \(x\), \(y\) y \(z\), respectivamente.  
Las \textbf{curvas de nivel} de un elipsoide, obtenidas al considerar la intersección con planos horizontales \(z=k\), producen
\textbf{elipses} siempre que \(|k| < c\), y no hay intersección para \(|k| > c\).
\end{TemaBox}

\paragraph{Curvas de nivel de un elipsoide.}
Partiendo del elipsoide estándar,
\[
\frac{x^2}{a^2} + \frac{y^2}{b^2} + \frac{z^2}{c^2} = 1,
\]
y fijando \(z = k\), se obtiene:
\[
\frac{x^2}{a^2} + \frac{y^2}{b^2} = 1 - \frac{k^2}{c^2}.
\]
Para que exista la curva de nivel, se requiere:
\[
1 - \frac{k^2}{c^2} > 0 \quad \Rightarrow \quad |k| < c.
\]
Esto corresponde a una \textbf{elipse} cuyos semiejes disminuyen conforme \(k\) se aproxima a \(-c\) o \(c\).  
Cuando \(|k| = c\), la elipse colapsa en un punto; cuando \(|k| > c\), la intersección es vacía.

\begin{figure}[h!]
\centering
\includegraphics[width=0.75\textwidth]{imagenes/elipsoide.png}
\caption{Ejemplo de elipsoide}
\label{fig:elipsoide}
\end{figure}

\paragraph{Ejemplo completo.}
Considera el elipsoide
\[
\frac{x^2}{4} + \frac{y^2}{9} + \frac{z^2}{16} = 1.
\]
Aquí \(a=2\), \(b=3\), \(c=4\).

\textbf{1) Curva de nivel en \(z=k\):}
\[
\frac{x^2}{4} + \frac{y^2}{9} = 1 - \frac{k^2}{16}.
\]

\textbf{2) Casos representativos:}

\[
k = 0: 
\quad
\frac{x^2}{4} + \frac{y^2}{9} = 1
\quad (\text{elipse máxima})
\]

\[
k = 2:
\quad
\frac{x^2}{4} + \frac{y^2}{9} = 1 - \frac{4}{16} = \frac{3}{4}
\quad (\text{elipse más pequeña})
\]

\[
k = 4:
\quad
1 - \frac{16}{16} = 0 
\quad \Rightarrow \quad \text{punto}
\]

\[
k = 5:
\quad
1 - \frac{25}{16} < 0 
\quad \Rightarrow \quad \text{sin intersección}
\]

\textbf{3) Interpretación geométrica:}  
Al cortar un elipsoide con planos horizontales se obtienen elipses que cambian de tamaño progresivamente.
El elipsoide es una superficie cerrada, por lo que el “mapa” de curvas de nivel es una familia de elipses que
disminuyen hasta llegar a un punto en los extremos.

\textbf{Conclusión:}  
Las curvas de nivel de un elipsoide son elipses para \(|k| < c\),
un punto para \(|k| = c\), y no existe intersección para \(|k| > c\).


\subsubsection*{Conos}

\begin{TemaBox}[Descripción]
Un \textbf{cono} es una superficie cuadrática que se caracteriza por tener un vértice y extenderse de manera lineal en todas las direcciones. 
El cono cuádrico más común se expresa mediante la ecuación
\[
\frac{x^2}{a^2} + \frac{y^2}{b^2} = \frac{z^2}{c^2}.
\]
Es una superficie \emph{doble}: se abre tanto hacia arriba como hacia abajo, con el vértice en el origen.  
Las \textbf{curvas de nivel} obtenidas al fijar \(z=k\) son \textbf{elipses} (o círculos en el caso \(a=b\)), excepto en \(z=0\), donde la superficie colapsa en un punto: el vértice.
\end{TemaBox}

\paragraph{Curvas de nivel de un cono.}
Dado el cono estándar
\[
\frac{x^2}{a^2} + \frac{y^2}{b^2} = \frac{z^2}{c^2},
\]
al considerar el plano horizontal \(z=k\), se obtiene:
\[
\frac{x^2}{a^2} + \frac{y^2}{b^2} = \frac{k^2}{c^2}.
\]

Esto corresponde a una \textbf{elipse} para todo \(k \neq 0\).  
Cuando \(k = 0\), la elipse se reduce al vértice \((0,0,0)\).

\begin{figure}[h!]
\centering
\includegraphics[width=0.75\textwidth]{imagenes/cono.png}
\caption{Ejemplo de cono y curvas de nivel.}
\label{fig:cono_curvas_nivel}
\end{figure}

\paragraph{Ejemplo completo.}
Considera el cono
\[
\frac{x^2}{4} + \frac{y^2}{9} = \frac{z^2}{16}.
\]
Aquí \(a=2\), \(b=3\), \(c=4\).

\textbf{1) Curva de nivel en \(z=k\):}
\[
\frac{x^2}{4} + \frac{y^2}{9} = \frac{k^2}{16}.
\]

\textbf{2) Casos representativos:}

\[
k = 1:
\quad
\frac{x^2}{4} + \frac{y^2}{9} = \frac{1}{16}
\quad (\text{elipse pequeña})
\]

\[
k = 2:
\quad
\frac{x^2}{4} + \frac{y^2}{9} = \frac{4}{16} = \frac14
\quad (\text{elipse más grande})
\]

\[
k = 4:
\quad
\frac{x^2}{4} + \frac{y^2}{9} = 1
\quad (\text{sección que corta al cono a la altura del semieje mayor})
\]

\[
k = 0:
\quad
\frac{x^2}{4} + \frac{y^2}{9} = 0
\quad \Rightarrow \quad (0,0)
\quad (\text{vértice})
\]

\textbf{3) Interpretación geométrica:}  
El cono genera curvas de nivel elípticas cuyos radios aumentan linealmente con \(|k|\).  
La superficie tiene simetría respecto al origen y un único punto singular en el vértice.

\textbf{Conclusión:}  
Los conos producen curvas de nivel elípticas para \(|k| > 0\) y un punto único cuando \(k = 0\).


\subsubsection*{Paraboloide Elíptico}

\begin{TemaBox}[Descripción]
Un \textbf{paraboloide elíptico} es una superficie cuadrática abierta cuya ecuación cartesiana estándar es
\[
z = \frac{x^2}{a^2} + \frac{y^2}{b^2}.
\]
Se trata de una superficie que se abre hacia arriba (o hacia abajo si se usa un signo negativo), 
y cuyas secciones horizontales \(z=k\) son \textbf{elipses}.  
El vértice del paraboloide se encuentra en el origen, y la superficie nunca produce valores de \(z\) negativos cuando
la ecuación está en su forma estándar con coeficientes positivos.
\end{TemaBox}

\paragraph{Curvas de nivel de un paraboloide elíptico.}
A partir de la ecuación
\[
z = \frac{x^2}{a^2} + \frac{y^2}{b^2},
\]
al fijar \(z=k\), con \(k \ge 0\), se obtiene:
\[
\frac{x^2}{a^2} + \frac{y^2}{b^2} = k.
\]

Esto corresponde a una \textbf{elipse} para \(k>0\).  
Cuando \(k=0\), la curva de nivel es un solo punto: \((0,0)\).  
No existe curva de nivel para \(k < 0\), ya que la superficie no toma valores negativos en esta forma estándar.

\begin{figure}[h!]
\centering
\begin{tikzpicture}
  \begin{axis}[
      view={35}{20},
      width=14cm, height=9cm,
      xlabel={$x$}, ylabel={$y$}, zlabel={$z$},
      domain=-3:3, y domain=-3:3,
      samples=30, samples y=30,
      colormap/viridis,
      title={Superficie \(z=x^2+y^2\) con planos \(z=\text{const}\)}
  ]
    % Superficie principal
    \addplot3[surf,opacity=0.7] {x^2 + y^2};
    
    % Circunferencias de nivel en 3D
    \addplot3[thick,brand,domain=0:360,samples=100] ({sqrt(2)*cos(x)},{sqrt(2)*sin(x)},{2});
    \addplot3[thick,brand,domain=0:360,samples=100] ({sqrt(4)*cos(x)},{sqrt(4)*sin(x)},{4});
    \addplot3[thick,brand,domain=0:360,samples=100] ({sqrt(6)*cos(x)},{sqrt(6)*sin(x)},{6});
  \end{axis}
\end{tikzpicture}
\caption{Paraboloide elíptico con cortes horizontales.}
\label{fig:niveles3d}
\end{figure}

\paragraph{Ejemplo completo.}
Considera el paraboloide elíptico
\[
z = \frac{x^2}{4} + \frac{y^2}{9}.
\]
Aquí \(a=2\) y \(b=3\).

\textbf{1) Curva de nivel en \(z=k\):}
\[
\frac{x^2}{4} + \frac{y^2}{9} = k.
\]

\textbf{2) Casos representativos:}

\[
k = 0:
\quad
\frac{x^2}{4} + \frac{y^2}{9} = 0
\quad \Rightarrow \quad (0,0)
\quad (\text{vértice})
\]

\[
k = 1:
\quad
\frac{x^2}{4} + \frac{y^2}{9} = 1
\quad \Rightarrow \quad \text{elipse}
\]

\[
k = 2:
\quad
\frac{x^2}{4} + \frac{y^2}{9} = 2
\quad \Rightarrow \quad \text{elipse más grande}
\]

\[
k = 3:
\quad
\frac{x^2}{4} + \frac{y^2}{9} = 3
\quad \Rightarrow \quad \text{elipse aún mayor}
\]

\textbf{3) Interpretación geométrica:}  
El paraboloide elíptico genera una familia de elipses que crecen de manera proporcional a \(\sqrt{k}\).  
A diferencia del elipsoide (superficie cerrada), aquí las curvas de nivel crecen indefinidamente.  
La superficie tiene un único punto crítico: el vértice en \((0,0,0)\).

\textbf{Conclusión:}  
Las curvas de nivel de un paraboloide elíptico son elipses para \(k>0\), un punto para \(k=0\), 
y no existen para \(k<0\).


\subsubsection*{Paraboloide Hiperbólico}

\begin{TemaBox}[Descripción]
Un \textbf{paraboloide hiperbólico} es una superficie cuadrática cuya ecuación estándar es
\[
z = \frac{x^2}{a^2} - \frac{y^2}{b^2}.
\]
Se caracteriza por ser una superficie \emph{ensillada}: tiene una curvatura positiva en una dirección y negativa en la otra.
Esto implica que la superficie se abre hacia arriba a lo largo del eje \(x\) y hacia abajo a lo largo del eje \(y\).
El punto \((0,0,0)\) es un \textbf{punto de silla}, donde la superficie no presenta ni máximo ni mínimo.
\end{TemaBox}

\paragraph{Curvas de nivel.}
Las curvas de nivel se obtienen fijando \(z=k\). A partir de
\[
k = \frac{x^2}{a^2} - \frac{y^2}{b^2},
\]
obtenemos diferentes tipos de curvas según el valor de \(k\):

\[
\frac{x^2}{a^2} - \frac{y^2}{b^2} = k.
\]

\textbf{Caso 1: \(k>0\).}  
La ecuación representa una \textbf{hipérbola} con eje transversal en la dirección del eje \(x\).

\textbf{Caso 2: \(k<0\).}  
Sigue siendo una \textbf{hipérbola}, pero ahora con eje transversal en la dirección del eje \(y\).

\textbf{Caso 3: \(k=0\).}  
La curva de nivel es un par de rectas que pasan por el origen:
\[
\frac{x^2}{a^2} = \frac{y^2}{b^2}
\quad \Rightarrow \quad
y = \pm \frac{b}{a} x.
\]

\begin{figure}[h!]
\centering
\includegraphics[width=0.50\textwidth]{imagenes/paraboloide_hiperbolico.png}
\caption{Paraboloide hiperbólico y sus curvas de nivel para distintos valores de \(k\).}
\label{fig:paraboloide_hiperbolico}
\end{figure}

\paragraph{Ejemplo completo.}

Consideremos el paraboloide hiperbólico:
\[
z = \frac{x^2}{4} - \frac{y^2}{9}.
\]
Aquí \(a=2\) y \(b=3\).

\textbf{1) Curva de nivel en \(z=k\):}
\[
\frac{x^2}{4} - \frac{y^2}{9} = k.
\]

\textbf{2) Casos representativos:}

\[
k = 1:
\quad
\frac{x^2}{4} - \frac{y^2}{9} = 1
\quad \Rightarrow \quad \text{hipérbola (eje transversal en \(x\))}
\]

\[
k = -1:
\quad
\frac{x^2}{4} - \frac{y^2}{9} = -1
\quad \Rightarrow \quad \text{hipérbola (eje transversal en \(y\))}
\]

\[
k = 0:
\quad
\frac{x^2}{4} - \frac{y^2}{9} = 0
\Rightarrow
x^2 = \frac{4}{9} y^2
\Rightarrow
y = \pm \frac{3}{2} x
\quad (\text{dos rectas})
\]

\textbf{3) Interpretación geométrica.}  
El paraboloide hiperbólico tiene curvatura negativa: mientras que en la dirección \(x\) la superficie se abre hacia arriba,
en la dirección \(y\) se abre hacia abajo.  
Por ello se le llama \emph{superficie de silla}.  
Este tipo de superficie aparece en arquitectura (tejados y cubiertas) y en física (superficies potenciales).

\textbf{Conclusión.}  
Las curvas de nivel del paraboloide hiperbólico siempre son hipérbolas excepto cuando \(k=0\), donde aparecen dos rectas que se cruzan en el punto de silla.


\subsubsection*{Hiperboloides}

\begin{TemaBox}[Descripción]
Los \textbf{hiperboloides} son superficies cuadráticas tridimensionales que pueden presentarse en dos formas:
\textbf{hiperboloide de una hoja} y \textbf{hiperboloide de dos hojas}.  
Ambos se describen mediante ecuaciones del tipo
\[
\pm\frac{x^2}{a^2} \pm \frac{y^2}{b^2} \pm \frac{z^2}{c^2} = 1,
\]
con combinaciones de signos que determinan la forma geométrica.  
Estas superficies son fundamentales en geometría analítica y modelan estructuras en ingeniería, óptica y arquitectura.
\end{TemaBox}

% =====================================================
% HIPERBOLOIDE DE UNA HOJA
% =====================================================

\paragraph{Hiperboloide de una hoja.}

Su ecuación estándar es:
\[
\frac{x^2}{a^2} + \frac{y^2}{b^2} - \frac{z^2}{c^2} = 1.
\]

Es una superficie abierta y continua, con secciones transversales de diferentes formas:

- Para \(z = k\), se obtiene:
\[
\frac{x^2}{a^2} + \frac{y^2}{b^2} = 1 + \frac{k^2}{c^2},
\]
es decir, \textbf{elipses}.  
- Para \(x = k\) o \(y = k\): aparecen \textbf{hipérbolas}.  

Geometría conocida como “forma de torre” o “collar” (ej.: torres de enfriamiento).

\begin{figure}[h!]
\centering
\includegraphics[width=0.3\textwidth]{imagenes/hiperboloide_una_hoja.png}
\caption{Ejemplo de hiperboloide de una hoja.}
\label{fig:hiperboloide_una_hoja}
\end{figure}

\textbf{Ejemplo completo.}  
Considera el hiperboloide
\[
\frac{x^2}{4} + \frac{y^2}{9} - \frac{z^2}{16} = 1.
\]

\textbf{1) Corte horizontal \(z=k\):}
\[
\frac{x^2}{4} + \frac{y^2}{9} = 1 + \frac{k^2}{16}.
\]
Esto es una \textbf{elipse} cuyo tamaño aumenta conforme \(|k|\) crece.

\textbf{2) Casos:}

\[
z = 0:
\quad \frac{x^2}{4} + \frac{y^2}{9} = 1
\quad (\text{elipse base})
\]

\[
z = 4:
\quad \frac{x^2}{4} + \frac{y^2}{9} = 2
\quad (\text{elipse más grande})
\]

\[
z = -4:
\quad \frac{x^2}{4} + \frac{y^2}{9} = 2
\quad (\text{simétrica})
\]

\textbf{Conclusión:}  
El hiperboloide de una hoja presenta elipses horizontales y se estrecha al acercarse al centro.


% =====================================================
% HIPERBOLOIDE DE DOS HOJAS
% =====================================================

\paragraph{Hiperboloide de dos hojas.}

Su ecuación estándar es:
\[
\frac{z^2}{c^2} - \frac{x^2}{a^2} - \frac{y^2}{b^2} = 1.
\]

En este caso la variable con signo positivo determina la dirección en la que aparecen las hojas (dos componentes separadas).  
El hiperboloide no tiene puntos para valores donde el miembro izquierdo no puede ser mayor que 1.

\textbf{Cortes horizontales \(z=k\):}
\[
\frac{x^2}{a^2} + \frac{y^2}{b^2} = \frac{k^2}{c^2} - 1.
\]

- Si \(|k| < c\): no existe curva (la superficie no está definida).  
- Si \(|k| = c\): aparece un punto.  
- Si \(|k| > c\): se obtiene una \textbf{elipse}.  

\begin{figure}[h!]
\centering
\includegraphics[width=0.3\textwidth]{imagenes/hiperboloide_dos_hojas.png}
\caption{Ejemplo de hiperboloide de dos hojas.}
\label{fig:hiperboloide_dos_hojas}
\end{figure}

\textbf{Ejemplo completo.}  
Considera el hiperboloide de dos hojas
\[
\frac{z^2}{4} - \frac{x^2}{9} - \frac{y^2}{9} = 1.
\]

\textbf{1) Curvas para \(z=k\):}
\[
\frac{x^2}{9} + \frac{y^2}{9} = \frac{k^2}{4} - 1.
\]

\textbf{2) Casos:}

\[
k = 0:
\quad \frac{k^2}{4}-1 = -1 \quad (\text{no hay puntos})
\]

\[
k = 2:
\quad \frac{x^2}{9} + \frac{y^2}{9} = 0
\quad (\text{punto único en la hoja superior})
\]

\[
k = 3:
\quad \frac{x^2}{9} + \frac{y^2}{9} = \frac{9}{4}-1 = \frac{5}{4}
\quad (\text{elipse})
\]

\[
k = -3:
\quad \text{elipse simétrica en la hoja inferior}
\]

\textbf{Conclusión:}  
El hiperboloide de dos hojas se abre en dos direcciones opuestas, con elipses cada vez más amplias conforme \(|z|\) crece.


\subsubsection{Ejercicios: Curvas de nivel}

Encuentra y describe las curvas de nivel de la siguiente función:

\begin{enumerate}
  \item \(z = x^2 + y^2\) para \(z = 4, 5, 6\)
  
  \textbf{Solución:}
  
  Esta es una función paraboloide circular (paraboloide de revolución). Para encontrar las curvas de nivel, 
  igualamos \(z\) a cada valor constante y despejamos la relación entre \(x\) e \(y\).
  
  \begin{itemize}
    \item \textbf{Cuando } \(z = 4\):
    \[
    x^2 + y^2 = 4
    \]
    Esta es la ecuación de una \textbf{circunferencia} con centro en \((0,0)\) y radio \(r = 2\).
    
    \item \textbf{Cuando } \(z = 5\):
    \[
    x^2 + y^2 = 5
    \]
    Esta es la ecuación de una \textbf{circunferencia} con centro en \((0,0)\) y radio \(r = \sqrt{5} \approx 2.236\).
    
    \item \textbf{Cuando } \(z = 6\):
    \[
    x^2 + y^2 = 6
    \]
    Esta es la ecuación de una \textbf{circunferencia} con centro en \((0,0)\) y radio \(r = \sqrt{6} \approx 2.449\).
  \end{itemize}
  
  \textbf{Interpretación geométrica:}
  \begin{itemize}
    \item Las tres curvas de nivel son circunferencias concéntricas (con el mismo centro pero radios diferentes).
    \item La función \(z = x^2 + y^2\) representa un paraboloide elíptico (en este caso, de revolución).
    \item El vértice del paraboloide está en \((0, 0, 0)\), el punto de menor altura.
    \item Conforme aumenta el valor de \(z\), las curvas de nivel son circunferencias cada vez más grandes.
    \item La distancia al origen es \(\sqrt{z}\), de modo que:
    \begin{align*}
      z = 4 &\Rightarrow r = 2 \\
      z = 5 &\Rightarrow r = \sqrt{5} \\
      z = 6 &\Rightarrow r = \sqrt{6}
    \end{align*}
  \end{itemize}
  
  \textbf{Conclusión:}
  Las curvas de nivel del paraboloide \(z = x^2 + y^2\) son circunferencias concéntricas. 
  Para cualquier valor \(z = k > 0\), la curva de nivel es una circunferencia de radio \(\sqrt{k}\) centrada en el origen.
\end{enumerate}


% ==================== UNIDAD 3 ====================
% Límites y continuidad en funciones de 3 variables

% ============================================================
%     LÍMITES Y CONTINUIDAD EN FUNCIONES DE TRES VARIABLES
% ============================================================

\section{Límites y Continuidad en Funciones de Tres Variables}

\begin{TemaBox}[Definición de función de 3 variables]
Una función de tres variables es una regla que asigna a cada punto
\((x,y,z)\in D\subset \mathbb{R}^3\) un número real:
\[
f: D\subset\mathbb{R}^3 \longrightarrow \mathbb{R}, \qquad f(x,y,z)=w.
\]
El dominio \(D\) es una región del espacio tridimensional y la imagen es un conjunto de valores reales.
\end{TemaBox}

\begin{TemaBox}[Definición formal de límite en 3 variables]
Sea \(f(x,y,z)\) una función definida en un conjunto que contiene puntos arbitrariamente cercanos a \((a,b,c)\).
Decimos que
\[
\lim_{(x,y,z)\to(a,b,c)} f(x,y,z)=L
\]
siempre que para toda sucesión \(\{(x_n,y_n,z_n)\}\) con 
\((x_n,y_n,z_n)\neq(a,b,c)\) y \((x_n,y_n,z_n)\to(a,b,c)\), se cumpla que
\[
f(x_n,y_n,z_n)\to L.
\]
Equivalentemente, para todo \(\varepsilon>0\) existe un \(\delta>0\) tal que
\[
0<\sqrt{(x-a)^2+(y-b)^2+(z-c)^2}<\delta 
\Rightarrow 
|f(x,y,z)-L|<\varepsilon.
\]
\end{TemaBox}

\subsection*{Métodos para determinar la existencia del límite}

\begin{itemize}
    \item \textbf{Sustitución directa:} funciona si la función es polinómica o el denominador no se anula.
    \item \textbf{Revisión de caminos:} si dos caminos dan límites diferentes, el límite no existe.
    \item \textbf{Acotación (Squeeze):} si la función queda entre dos límites iguales.
    \item \textbf{Coordenadas esféricas:} cuando se busca el límite al origen, usando \(x^2+y^2+z^2=\rho^2\).
\end{itemize}

% ============================================================
%              EJEMPLOS RESUELTOS DE LÍMITES
% ============================================================

\section*{Ejemplos Resueltos}

\subsection*{Ejemplo 1 — Sustitución directa}

\[
\lim_{(x,y,z)\to(1,2,-1)} (3x - y + 2z).
\]
La función es un polinomio (continua), por lo que basta sustituir:
\[
3(1)-2+2(-1)=3-2-2=-1.
\]

\[
\boxed{-1}
\]

\subsection*{Ejemplo 2 — Límite existente (acotación)}

\[
\lim_{(x,y,z)\to(0,0,0)} 
\frac{x^2 + y^2}{\sqrt{x^2+y^2+z^2}}.
\]

Se cumple:
\[
0 \le 
\frac{x^2 + y^2}{\sqrt{x^2+y^2+z^2}}
\le \sqrt{x^2+y^2}.
\]

Como \(\sqrt{x^2+y^2}\to0\), por el criterio del sándwich:

\[
\boxed{0}
\]

\subsection*{Ejemplo 3 — Límite que NO existe}

\[
\lim_{(x,y,z)\to(0,0,0)} \frac{xy}{x^2+y^2+z^2}.
\]

Camino \(x=y=t, z=0\):
\[
\frac{t^2}{2t^2}=\frac12.
\]

Camino \(x=t,y=-t,z=0\):
\[
\frac{-t^2}{2t^2}=-\frac12.
\]

Como los límites difieren:
\[
\boxed{\text{El límite no existe}}
\]

\subsection*{Ejemplo 4 — Uso de coordenadas esféricas}

\[
f(x,y,z)=\frac{x^2+y^2+z^2}{\sqrt{x^2+y^2+z^2+1}}.
\]

Sustituyendo \(x^2+y^2+z^2=\rho^2\):
\[
f(\rho)=\frac{\rho^2}{\sqrt{1+\rho^2}}\to0.
\]

\[
\boxed{0}
\]

% ============================================================
%                CONTINUIDAD EN 3 VARIABLES
% ============================================================

\section{Continuidad de Funciones de Tres Variables}

\begin{TemaBox}[Definición de continuidad]
Una función \(f(x,y,z)\) es continua en el punto \((a,b,c)\) si:
\[
\lim_{(x,y,z)\to(a,b,c)} f(x,y,z)=f(a,b,c).
\]
Esto requiere:
\begin{enumerate}
    \item \(f(a,b,c)\) definido,
    \item el límite existe,
    \item ambos valores son iguales.
\end{enumerate}
\end{TemaBox}

\subsection*{Ejemplo 5 — Función continua en todo su dominio}

\[
f(x,y,z)=\sqrt{x^2+y^2+z^2+5}.
\]

El radicando es positivo, por lo que es continua en todo \(\mathbb{R}^3\).

\subsection*{Ejemplo 6 — Discontinuidad removible}

\[
f(x,y,z)=
\begin{cases}
\dfrac{x^2+y^2+z^2}{x^2+y^2+z^2+1}, & (x,y,z)\neq (0,0,0),\\[4pt]
0, & (x,y,z)=(0,0,0).
\end{cases}
\]

En esféricas:
\[
\frac{\rho^2}{\rho^2+1}\to0.
\]

Como coincide con el valor dado:
\[
\boxed{\text{Es continua en el origen.}}
\]

\subsection*{Ejemplo 7 — Discontinuidad esencial}

\[
f(x,y,z)=\frac{x+y+z}{\sqrt{x^2+y^2+z^2}}.
\]

Camino \(x=y=z=t\):
\[
\frac{3t}{\sqrt{3t^2}}=
\begin{cases}
\sqrt3, & t>0,\\
-\sqrt3, & t<0.
\end{cases}
\]

Límites distintos:

\[
\boxed{\text{No es continua en }(0,0,0).}
\]

\subsection*{Ejemplo 8 — Continuidad excepto en la esfera unitaria}

\[
f(x,y,z)=\frac{x-y+z}{x^2+y^2+z^2-1}.
\]

El denominador se anula cuando:
\[
x^2+y^2+z^2 = 1.
\]

Esto describe la \textbf{esfera unitaria}.  
Una función racional es continua donde el denominador no es cero, por lo que:

\[
\boxed{
f \text{ es continua en } \mathbb{R}^3 \setminus \{(x,y,z)\,\mid\,x^2+y^2+z^2=1\}.
}
\]

\subsection*{Ejercicios: Límites de funciones de una variable}

Calcula los siguientes límites:

\begin{enumerate}
  \item \(\displaystyle \lim_{x \to -3} \frac{x+3}{x^2 - 9}\)
  
  \textbf{Solución:}
  Factorizamos el denominador: \(x^2 - 9 = (x-3)(x+3)\)
  
  \[\lim_{x \to -3} \frac{x+3}{(x-3)(x+3)} = \lim_{x \to -3} \frac{1}{x-3} = \frac{1}{-3-3} = \frac{1}{-6} = -\frac{1}{6}\]
  
  \item \(\displaystyle \lim_{x \to 1} \frac{x^2 - 4x + 3}{x^2 + 3x - 4}\)
  
  \textbf{Solución:}
  Factorizamos: \(x^2 - 4x + 3 = (x-1)(x-3)\) y \(x^2 + 3x - 4 = (x-1)(x+4)\)
  
  \[\lim_{x \to 1} \frac{(x-1)(x-3)}{(x-1)(x+4)} = \lim_{x \to 1} \frac{x-3}{x+4} = \frac{1-3}{1+4} = \frac{-2}{5} = -\frac{2}{5}\]
  
  \item \(\displaystyle \lim_{x \to 0} \frac{x^3 - 2x^2}{3x^2}\)
  
  \textbf{Solución:}
  Factorizamos el numerador: \(x^3 - 2x^2 = x^2(x - 2)\)
  
  \[\lim_{x \to 0} \frac{x^2(x-2)}{3x^2} = \lim_{x \to 0} \frac{x-2}{3} = \frac{0-2}{3} = -\frac{2}{3}\]
  
  \item \(\displaystyle \lim_{x \to 5} \frac{x^2 - 10x + 25}{x^2 - 3x - 10}\)
  
  \textbf{Solución:}
  Factorizamos: \(x^2 - 10x + 25 = (x-5)^2\) y \(x^2 - 3x - 10 = (x-5)(x+2)\)
  
  \[\lim_{x \to 5} \frac{(x-5)^2}{(x-5)(x+2)} = \lim_{x \to 5} \frac{x-5}{x+2} = \frac{5-5}{5+2} = \frac{0}{7} = 0\]
  
  \item \(\displaystyle \lim_{x \to 4} \frac{x^2 - 4x}{x^2 - 16}\)
  
  \textbf{Solución:}
  Factorizamos: \(x^2 - 4x = x(x-4)\) y \(x^2 - 16 = (x-4)(x+4)\)
  
  \[\lim_{x \to 4} \frac{x(x-4)}{(x-4)(x+4)} = \lim_{x \to 4} \frac{x}{x+4} = \frac{4}{4+4} = \frac{4}{8} = \frac{1}{2}\]
  
  \item \(\displaystyle \lim_{x \to 0} \frac{x^3 - 5x^2 + 3x}{4x^2 - 4x}\)
  
  \textbf{Solución:}
  Factorizamos: \(x^3 - 5x^2 + 3x = x(x^2 - 5x + 3)\) y \(4x^2 - 4x = 4x(x-1)\)
  
  \[\lim_{x \to 0} \frac{x(x^2 - 5x + 3)}{4x(x-1)} = \lim_{x \to 0} \frac{x^2 - 5x + 3}{4(x-1)} = \frac{0 - 0 + 3}{4(0-1)} = \frac{3}{-4} = -\frac{3}{4}\]
  
  \item \(\displaystyle \lim_{x \to 0} \frac{\sqrt{9+x} - 3}{x}\)
  
  \textbf{Solución:}
  Multiplicamos por el conjugado:
  
  \[\lim_{x \to 0} \frac{\sqrt{9+x} - 3}{x} \cdot \frac{\sqrt{9+x} + 3}{\sqrt{9+x} + 3} = \lim_{x \to 0} \frac{(9+x) - 9}{x(\sqrt{9+x} + 3)}\]
  
  \[= \lim_{x \to 0} \frac{x}{x(\sqrt{9+x} + 3)} = \lim_{x \to 0} \frac{1}{\sqrt{9+x} + 3} = \frac{1}{\sqrt{9} + 3} = \frac{1}{3+3} = \frac{1}{6}\]
  
  \item \(\displaystyle \lim_{x \to 1} \frac{\sqrt{x+3} - 2}{x-1}\)
  
  \textbf{Solución:}
  Multiplicamos por el conjugado:
  
  \[\lim_{x \to 1} \frac{\sqrt{x+3} - 2}{x-1} \cdot \frac{\sqrt{x+3} + 2}{\sqrt{x+3} + 2} = \lim_{x \to 1} \frac{(x+3) - 4}{(x-1)(\sqrt{x+3} + 2)}\]
  
  \[= \lim_{x \to 1} \frac{x-1}{(x-1)(\sqrt{x+3} + 2)} = \lim_{x \to 1} \frac{1}{\sqrt{x+3} + 2} = \frac{1}{\sqrt{1+3} + 2} = \frac{1}{\sqrt{4} + 2} = \frac{1}{2+2} = \frac{1}{4}\]
  
  \item \(\displaystyle \lim_{x \to -2} \frac{x+2}{\sqrt{x+3} - 1}\)
  
  \textbf{Solución:}
  Multiplicamos por el conjugado:
  
  \[\lim_{x \to -2} \frac{x+2}{\sqrt{x+3} - 1} \cdot \frac{\sqrt{x+3} + 1}{\sqrt{x+3} + 1} = \lim_{x \to -2} \frac{(x+2)(\sqrt{x+3} + 1)}{(x+3) - 1}\]
  
  \[= \lim_{x \to -2} \frac{(x+2)(\sqrt{x+3} + 1)}{x+2} = \lim_{x \to -2} (\sqrt{x+3} + 1) = \sqrt{-2+3} + 1 = \sqrt{1} + 1 = 1 + 1 = 2\]
  
  \item \(\displaystyle \lim_{x \to 0} \frac{x^2}{\sqrt{x+3} - \sqrt{3}}\)
  
  \textbf{Solución:}
  Multiplicamos por el conjugado:
  
  \[\lim_{x \to 0} \frac{x^2}{\sqrt{x+3} - \sqrt{3}} \cdot \frac{\sqrt{x+3} + \sqrt{3}}{\sqrt{x+3} + \sqrt{3}} = \lim_{x \to 0} \frac{x^2(\sqrt{x+3} + \sqrt{3})}{(x+3) - 3}\]
  
  \[= \lim_{x \to 0} \frac{x^2(\sqrt{x+3} + \sqrt{3})}{x} = \lim_{x \to 0} x(\sqrt{x+3} + \sqrt{3})\]
  
  Cuando \(x \to 0\): \(0 \cdot (\sqrt{3} + \sqrt{3}) = 0 \cdot 2\sqrt{3} = 0\)
  
  \item \(\displaystyle \lim_{x \to 0} \frac{\sqrt{4 - 2x + x^2} - 2}{x}\)
  
  \textbf{Solución:}
  Observa que \(4 - 2x + x^2 = (2-x)^2\), entonces \(\sqrt{4 - 2x + x^2} = |2-x|\)
  
  Para \(x\) cercano a 0, \(2 - x > 0\), entonces \(|2-x| = 2-x\)
  
  \[\lim_{x \to 0} \frac{2-x - 2}{x} = \lim_{x \to 0} \frac{-x}{x} = \lim_{x \to 0} (-1) = -1\]
\end{enumerate}