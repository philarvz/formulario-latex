\section{Unidad IV: Funciones vectoriales e integral de línea}

En esta unidad se estudian herramientas fundamentales del cálculo vectorial que
permiten describir y analizar curvas en el plano y en el espacio, así como
medir cantidades a lo largo de dichas curvas mediante integrales de línea.
El objetivo es que el estudiante sea capaz de interpretar, manipular y aplicar
funciones vectoriales, ecuaciones paramétricas, operaciones de cálculo sobre
ellas y el concepto de integral de línea en contextos geométricos y físicos.

\subsection{Funciones Vectoriales}

\subsubsection{Definición y análisis geométrico}

Una función vectorial es una aplicación
\[
\vec r : I \subset \mathbb{R} \to \mathbb{R}^n,
\qquad
\vec r(t) = \langle x(t), y(t), z(t) \rangle,
\]
donde cada componente es una función real. La imagen de \(\vec r\) es una curva
parametrizada en el espacio, y el parámetro \(t\) determina la forma y la manera
en que se recorre dicha curva.

\begin{InfoBox}
El parámetro no necesariamente es tiempo. Un cambio de parámetro puede modificar
la rapidez del recorrido sin alterar la geometría de la curva.
\end{InfoBox}

El análisis geométrico de una función vectorial incluye:

\begin{itemize}
    \item el dominio del parámetro,
    \item el recorrido o trayectoria,
    \item la orientación de la curva,
    \item la posible presencia de simetrías,
    \item el estudio de puntos singulares.
\end{itemize}

Una curva es \textbf{regular} en un punto si su derivada no se anula allí:
\[
\vec r'(t_0) \ne \vec 0.
\]
En tal caso, existe un vector tangente bien definido.

\subsubsection{Propiedades fundamentales}

Las funciones vectoriales heredan propiedades de las funciones reales
componente a componente. Entre las más relevantes:

\begin{itemize}
    \item \textbf{Continuidad:} \(\vec r(t)\) es continua en \(t=a\) si todas las
    componentes \(x(t), y(t), z(t)\) lo son.
    \item \textbf{Derivabilidad:} La derivada existe si las funciones
    coordenadas son derivables.
    \item \textbf{Linealidad:}
    \[
    (\vec r + \vec s)' = \vec r' + \vec s', \qquad (c\vec r)' = c\,\vec r'.
    \]
    \item \textbf{Regla del producto punto:}
    \[
    \frac{d}{dt}(\vec r \cdot \vec s)
    = \vec r' \cdot \vec s + \vec r \cdot \vec s'.
    \]
    \item \textbf{Regla del producto cruz:}
    \[
    \frac{d}{dt}(\vec r \times \vec s)
    = \vec r' \times \vec s + \vec r \times \vec s'.
    \]
\end{itemize}

\begin{InfoBox}
La continuidad y derivabilidad se miden coordenada por coordenada, lo cual
simplifica muchos análisis teóricos.
\end{InfoBox}

\subsubsection{Teoremas esenciales}

\begin{TemaBox}[Teorema: Derivabilidad implica continuidad]
Si \(\vec r\) es derivable en \(t=a\), entonces es continua en dicho punto.
\end{TemaBox}

\textbf{Demostración:}
La derivabilidad de cada componente implica su continuidad. Luego,
\[
\vec r(t) = \langle x(t), y(t), z(t)\rangle
\]
es continua porque sus componentes lo son. \(\square\)

\begin{TemaBox}[Teorema: Vector tangente unitario]
Si \(\vec r'(t) \ne \vec 0\), el vector
\[
\vec T(t) = \frac{\vec r'(t)}{\|\vec r'(t)\|}
\]
define la dirección instantánea de la curva.
\end{TemaBox}

\textbf{Demostración:}  
El vector \(\vec r'(t)\) es tangente y su normalización produce un vector
unitario en la misma dirección. \(\square\)

\subsubsection{Aplicaciones matemáticas}

Las funciones vectoriales son fundamentales en:

\begin{itemize}
    \item descripción de curvas implícitas mediante parametrizaciones,
    \item cálculo de longitudes de arco,
    \item integración a lo largo de trayectorias,
    \item construcción de soluciones a ecuaciones diferenciales,
    \item representación de intersecciones entre superficies.
\end{itemize}

\begin{InfoBox}
Parametrizar curvas simplifica cálculos que serían imposibles con ecuaciones
cartesianas de la forma \(y=f(x)\).
\end{InfoBox}

\subsubsection{Aplicaciones físicas}

En física, una función vectorial puede describir:

\begin{itemize}
    \item la posición de un objeto,
    \item la trayectoria de un proyectil,
    \item órbitas planetarias,
    \item movimientos oscilatorios,
    \item trayectorias bajo campos electromagnéticos.
\end{itemize}

\begin{InfoBox}
La velocidad es \(\vec v(t)=\vec r'(t)\) y la aceleración es
\(\vec a(t)=\vec r''(t)\). Estas magnitudes determinan la dinámica del sistema.
\end{InfoBox}

\begin{EjercicioBox}[Ejemplo: trayectoria parabólica]
Sea \(\vec r(t)=\langle t, t^2, 0\rangle\).
Halle el vector tangente unitario en \(t=1\).
\end{EjercicioBox}

\subsubsection{Reparametrización por longitud de arco}

Una curva parametrizada por \(\vec r(t)\) no siempre se recorre con rapidez
constante. Para ciertos cálculos es conveniente construir una parametrización
que dependa de la longitud recorrida.

Sea la longitud de arco desde \(t_0\) hasta \(t\) dada por:
\[
s(t) = \int_{t_0}^t \|\vec r'(\tau)\|\, d\tau.
\]

Si \(\vec r'(t)\neq 0\) en el intervalo, entonces \(s(t)\) es estrictamente
creciente y se puede invertir para obtener \(t = t(s)\). La curva
reparametrizada es:

\[
\vec R(s) = \vec r(t(s)).
\]

\begin{InfoBox}
La parametrización por longitud de arco produce una rapidez unidad:
\[
\|\vec R'(s)\| = 1.
\]
\end{InfoBox}

\begin{TemaBox}[Teorema: Reparametrización regular]
Si \(\|\vec r'(t)\|\neq 0\) en \(I\), entonces la curva puede parametrizarse por
longitud de arco en todo el intervalo.
\end{TemaBox}

\textbf{Demostración:}  
La función \(s(t)\) es continua, derivable y estrictamente creciente. Por el
teorema de la función inversa, admite una inversa derivable. \(\square\)

\subsubsection{Regularidad y suavidad de curvas}

Una curva es \textbf{regular} si \(\vec r'(t)\neq 0\) para todo \(t\). En estas
condiciones, la curva no presenta puntos en los que la dirección tangente sea
indeterminada.

Una curva es de clase \(C^k\) si \(\vec r(t)\) tiene derivadas continuas hasta
orden \(k\). Esta suavidad garantiza la existencia de cantidades geométricas
como curvatura y torsión.

\begin{InfoBox}
La regularidad excluye curvas con esquinas o cuspides. Tales puntos anulan la
derivada y requieren análisis especial.
\end{InfoBox}

\subsubsection{Curvatura y vector normal}

La curvatura mide el cambio de dirección del vector tangente:

\[
\kappa(t)=
\frac{\|\vec T'(t)\|}{\|\vec r'(t)\|},
\qquad
\vec T(t)=\frac{\vec r'(t)}{\|\vec r'(t)\|}.
\]

Alternativamente,
\[
\kappa(t) =
\frac{\|\vec r'(t)\times \vec r''(t)\|}
     {\|\vec r'(t)\|^3}.
\]

El vector normal principal se define como:
\[
\vec N(t)=
\frac{\vec T'(t)}{\|\vec T'(t)\|}.
\]

\begin{TemaBox}[Teorema: Expresión alternativa de la curvatura]
Si \(\vec r'(t)\neq 0\), entonces
\[
\kappa(t) = \frac{\|\vec r'(t)\times \vec r''(t)\|}
                  {\|\vec r'(t)\|^3}.
\]
\end{TemaBox}

\textbf{Demostración:}  
Se diferencia \(\vec T(t)\) usando la regla del cociente vectorial y se agrupa
el término perpendicular a \(\vec r'(t)\). El módulo del componente normal
coincide con la expresión dada. \(\square\)

\begin{InfoBox}
La curvatura describe cómo se dobla la curva. Si \(\kappa(t)=0\), la curva es
localmente recta en ese punto.
\end{InfoBox}

\subsubsection{Interpretación geométrica avanzada}

La geometría de una curva depende de tres elementos:

\begin{itemize}
    \item la magnitud de la velocidad,
    \item la curvatura,
    \item la torsión (en el espacio tridimensional).
\end{itemize}

La descomposición de la aceleración es:
\[
\vec a(t)
= \frac{d}{dt}\left(\|\vec r'(t)\|\right)\vec T(t)
+ \|\vec r'(t)\|^2 \kappa(t)\vec N(t).
\]

El primer término corresponde al cambio de rapidez; el segundo, al cambio de
dirección.

\begin{InfoBox}
Incluso si la rapidez es constante, la aceleración puede ser grande debido a la
curvatura, como en movimientos circulares.
\end{InfoBox}

\begin{EjercicioBox}[Ejemplo: curvatura de una hélice]
Para la hélice \(\vec r(t)=\langle a\cos t, a\sin t, bt\rangle\),
calcule \(\kappa(t)\).

La derivada es \(\vec r'(t)=\langle -a\sin t, a\cos t, b\rangle\), cuya norma es
\(\sqrt{a^2+b^2}\). Usando la fórmula alternativa:
\[
\kappa(t) = \frac{a}{a^2+b^2}.
\]
La curvatura es constante.
\end{EjercicioBox}

\subsubsection{Torsión y comportamiento espacial de curvas}

La torsión describe cómo una curva tridimensional abandona el plano osculador.
Para una curva regular de clase \(C^3\), se define como:

\[
\tau(t)
= \frac{
(\vec r'(t)\times \vec r''(t))\cdot \vec r'''(t)
}{
\|\vec r'(t)\times \vec r''(t)\|^2
}.
\]

Si \(\tau(t)=0\) para todo \(t\), la curva es plana.

\begin{TemaBox}[Teorema: Condición de planaridad]
Una curva suave satisface \(\tau(t)=0\) para todo \(t\) si y solo si su imagen
está contenida en un plano.
\end{TemaBox}

\textbf{Demostración:}  
La torsión nula implica que \(\vec r'(t), \vec r''(t), \vec r'''(t)\) son linealmente
dependientes y no generan un espacio tridimensional. Luego, todos los vectores
están contenidos en un subespacio bidimensional fijo. \(\square\)

\begin{InfoBox}
La torsión es especialmente importante en trayectorias helicoidales,
movimientos en campos magnéticos y en geometría diferencial avanzada.
\end{InfoBox}

\subsubsection{Marco de Frenet y descripción completa de la curva}

Para curvas regulares y suficientemente suaves, se definen tres vectores
ortogonales:

\[
\vec T(t) \quad \text{(tangente)}, \qquad
\vec N(t) \quad \text{(normal principal)}, \qquad
\vec B(t)=\vec T(t)\times \vec N(t) \quad \text{(binormal)}.
\]

Estos vectores forman el \textbf{marco de Frenet}, que describe cómo se mueve
la curva en el espacio.

Las ecuaciones de Frenet–Serret relacionan los vectores entre sí:

\[
\vec T'(t)=\kappa(t)\vec N(t),
\]
\[
\vec N'(t)=-\kappa(t)\vec T(t)+\tau(t)\vec B(t),
\]
\[
\vec B'(t)=-\tau(t)\vec N(t).
\]

\begin{InfoBox}
El marco de Frenet permite estudiar la geometría local de la curva sin
referencias externas, usando solo su forma intrínseca.
\end{InfoBox}

\subsubsection{Longitud de arco y análisis integral}

La longitud de una curva entre \(t=a\) y \(t=b\) es:

\[
L = \int_a^b \|\vec r'(t)\|\, dt.
\]

Si la curva está parametrizada por longitud de arco, \(s\), entonces:

\[
\frac{d\vec r}{ds} = \vec T(s).
\]

Esto simplifica cálculos derivados y facilita la construcción de marcos móviles.

\begin{EjercicioBox}[Ejemplo: longitud de una curva]
Calcule la longitud de \(\vec r(t)=\langle 3t,4t,0\rangle\) para \(0\le t\le 1\).
Como \(\|\vec r'(t)\|=5\), la longitud es \(L=5\).
\end{EjercicioBox}

\subsubsection{Observaciones avanzadas sobre curvas}

Las curvas espaciales pueden mostrar comportamientos complejos:

\begin{itemize}
    \item variación rápida de curvatura,
    \item torsión elevada en intervalos cortos,
    \item puntos donde \(\vec r'(t)=0\) (cuspides),
    \item auto-intersecciones.
\end{itemize}

En aplicaciones reales, estos fenómenos aparecen en:

\begin{itemize}
    \item trayectorias de partículas en campos magnéticos,
    \item curvas generadas por robots industriales,
    \item geometrías de cables y fibras ópticas,
    \item trayectorias de aeronaves.
\end{itemize}

\begin{InfoBox}
Las curvas generadas por funciones vectoriales permiten modelar
movimientos complejos que no pueden describirse con funciones escalar–variable
simples.
\end{InfoBox}

\begin{EjercicioBox}[Ejercicios propuestos]
\begin{enumerate}
    \item Determine si la curva \(\vec r(t)=\langle t^3,t^2,0\rangle\) es regular.
    \item Calcule la curvatura de \(\vec r(t)=\langle e^t,\cos t,\sin t\rangle\).
    \item Verifique si la curva \(\vec r(t)=\langle t, t^2, t^3\rangle\) es plana.
    \item Halle el marco de Frenet para la curva \(\vec r(t)=\langle \cos t,\sin t,0\rangle\).
    \item Encuentre la longitud de arco de \(\vec r(t)=\langle 2\cos t, 2\sin t, bt\rangle\) en un intervalo dado.
\end{enumerate}
\end{EjercicioBox}

\subsection{Ecuaciones Paramétricas}

\subsubsection{Introducción general}

Una curva en el plano o en el espacio puede describirse mediante ecuaciones
paramétricas. En lugar de una ecuación explícita del tipo \(y=f(x)\), se
introduce un parámetro \(t\) y se define:

\[
x = x(t), \qquad y = y(t), \qquad z = z(t).
\]

La pareja o terna \((x(t), y(t), z(t))\) determina la posición de un punto
móvil al variar \(t\). Esta descripción permite representar curvas que no
pueden escribirse de manera cartesiana simple.

\begin{InfoBox}
Las ecuaciones paramétricas proporcionan un control total sobre la forma de la
curva y la manera en que se recorre, algo imposible con ecuaciones cartesianas
clásicas.
\end{InfoBox}

\subsubsection{Ventajas de la parametrización}

El uso de parametrizaciones ofrece beneficios esenciales:

\begin{itemize}
    \item Permiten describir curvas que no pueden representarse como el gráfico
          de una función.
    \item Evitan restricciones como la unicidad vertical u horizontal.
    \item Facilitan cálculos como velocidad, aceleración y curvatura.
    \item Simplifican integrales a lo largo de curvas.
    \item Permiten diseñar trayectorias de manera explícita.
\end{itemize}

Ejemplos de curvas difíciles en forma cartesiana pero fáciles en forma
paramétrica incluyen:

\begin{itemize}
    \item circunferencias, elipses y cicloides,
    \item trayectorias de proyectiles,
    \item curvas auto–intersectantes,
    \item hélices espaciales,
    \item curvas definidas por movimientos oscilatorios.
\end{itemize}

\subsubsection{Conversión entre formas paramétricas y cartesianas}

Si se tienen las ecuaciones paramétricas \(x=x(t)\) y \(y=y(t)\), una ecuación
cartesiana puede obtenerse eliminando el parámetro \(t\), siempre que sea
posible.

\[
t = g(x) \quad \Rightarrow \quad y = y(g(x)).
\]

La conversión puede ser sencilla (como en la circunferencia) o imposible de
realizar en forma cerrada (como en una cicloide).

\begin{InfoBox}
La conversión cartesiana no siempre es factible. En muchos casos, la forma
paramétrica es la única representación viable de la curva.
\end{InfoBox}

\subsubsection{Derivadas paramétricas}

Si \(x=x(t)\) y \(y=y(t)\) son derivables, la pendiente de la curva está dada
por:

\[
\frac{dy}{dx} =
\frac{\frac{dy}{dt}}{\frac{dx}{dt}},
\qquad \text{si } \frac{dx}{dt}\neq 0.
\]

Para derivadas de segundo orden:

\[
\frac{d^2y}{dx^2}
=
\frac{
\frac{d}{dt}\left( \frac{dy}{dt} \, / \, \frac{dx}{dt} \right)
}{
\frac{dx}{dt}
}.
\]

\begin{TemaBox}[Teorema: Derivada paramétrica]
Si \(x(t)\) y \(y(t)\) son derivables y \(x'(t)\neq 0\), entonces
\[
\frac{dy}{dx} = \frac{y'(t)}{x'(t)}.
\]
\end{TemaBox}

\textbf{Demostración:}  
Por la regla de la cadena,  
\[
\frac{dy}{dx}
=
\frac{dy/dt}{dx/dt}.
\qquad \square
\]

\begin{InfoBox}
El cálculo paramétrico evita complicaciones en curvas donde la pendiente
vertical impediría usar la forma \(y=f(x)\).
\end{InfoBox}

\subsubsection{Velocidad y aceleración en ecuaciones paramétricas}

La función vectorial asociada es:

\[
\vec r(t)=\langle x(t), y(t), z(t)\rangle.
\]

Sus derivadas son:

\[
\vec v(t)=\vec r'(t)=\langle x'(t), y'(t), z'(t)\rangle,
\qquad
\vec a(t)=\vec r''(t).
\]

La rapidez:

\[
\|\vec v(t)\| = \sqrt{x'(t)^2 + y'(t)^2 + z'(t)^2}.
\]

\begin{InfoBox}
La rapidez no depende de la forma de la curva sino de la
parametrización. Cambiar el parámetro cambia la rapidez, pero no la trayectoria.
\end{InfoBox}

\subsubsection{Trayectorias clásicas en forma paramétrica}

Muchas curvas fundamentales se expresan con facilidad mediante parametrizaciones:

\begin{itemize}
    \item \textbf{Recta:}  
    \[
    \vec r(t)=\vec r_0 + t\vec v.
    \]

    \item \textbf{Circunferencia:}  
    \[
    \vec r(t)=\langle a\cos t, a\sin t\rangle.
    \]

    \item \textbf{Elipse:}  
    \[
    \vec r(t)=\langle a\cos t, b\sin t\rangle.
    \]

    \item \textbf{Hélice:}  
    \[
    \vec r(t)=\langle a\cos t, a\sin t, bt\rangle.
    \]

    \item \textbf{Cicloide:}  
    \[
    x(t)=a(t-\sin t),\qquad y(t)=a(1-\cos t).
    \]
\end{itemize}

\begin{InfoBox}
La cicloide es una curva cuya representación cartesiana es prácticamente
inmanejable, pero su forma paramétrica es simple y elegante.
\end{InfoBox}

\begin{EjercicioBox}[Ejemplo: pendiente paramétrica]
Para la curva \(x=t^2+1\), \(y=\ln(t+2)\), calcule \(\frac{dy}{dx}\).

\[
\frac{dy}{dx}
=
\frac{y'(t)}{x'(t)}
=
\frac{1/(t+2)}{2t}.
\]
\end{EjercicioBox}

\subsubsection{Conversión entre sistemas de coordenadas}

Una ventaja importante de la parametrización es que permite describir curvas
utilizando distintos sistemas de coordenadas. Los principales sistemas usados
en cálculo vectorial son:

\begin{itemize}
    \item coordenadas cartesianas,
    \item coordenadas polares,
    \item coordenadas cilíndricas,
    \item coordenadas esféricas.
\end{itemize}

\subsubsection*{Coordenadas polares}

Una curva puede describirse mediante:
\[
r = r(\theta),
\qquad
x = r(\theta)\cos\theta,
\qquad
y = r(\theta)\sin\theta.
\]

Ejemplos:

\begin{itemize}
    \item Espiral de Arquímedes: \(r=a\theta\).
    \item Rosa polar: \(r=a\cos(k\theta)\).
\end{itemize}

\subsubsection*{Coordenadas cilíndricas}

En el espacio,
\[
x = r\cos\theta,\qquad y = r\sin\theta,\qquad z=z.
\]

Una hélice se expresa elegantemente como:
\[
r=a, \qquad \theta=t, \qquad z=bt.
\]

\subsubsection*{Coordenadas esféricas}

\[
x=\rho\sin\phi\cos\theta,\qquad
y=\rho\sin\phi\sin\theta,\qquad
z=\rho\cos\phi.
\]

\begin{InfoBox}
La elección del sistema de coordenadas adecuado simplifica de forma radical la
descripción de trayectorias y la resolución de integrales.
\end{InfoBox}

\subsubsection{Curvas especiales avanzadas}

Existen curvas que son fundamentales en matemáticas, física e ingeniería:

\subsubsection*{Cicloide}

Generada por un punto en una rueda que gira sin deslizar:
\[
x(t)=a(t-\sin t), \qquad y(t)=a(1-\cos t).
\]

Propiedades destacadas:

\begin{itemize}
    \item optimiza tiempos de descenso (problema de la braquistócrona),
    \item modela trayectorias en mecanismos mecánicos,
    \item presenta simetrías translacionales.
\end{itemize}

\subsubsection*{Curva braquistócrona}

Solución del problema clásico de Bernoulli:
encontrar la curva por la cual una partícula desciende más rápidamente bajo
gravedad. Sorprendentemente, la solución es una cicloide, no una recta ni una
parábola.

\subsubsection*{Cardioide}

Una curva polar dada por:
\[
r = a(1-\cos\theta).
\]

Su geometría aparece en:

\begin{itemize}
    \item acústica,
    \item antenas direccionales,
    \item reflexión y catóptrica,
    \item óptica geométrica.
\end{itemize}

\subsubsection{Simetrías en curvas paramétricas}

Las simetrías de una curva se analizan observando transformaciones del
parámetro que dejan invariantes las ecuaciones.

\subsubsection*{Simetría respecto al eje \(x\)}

Si
\[
y(-t) = -y(t)
\quad \text{y} \quad x(-t)=x(t),
\]
la curva es simétrica respecto al eje \(x\).

\subsubsection*{Simetría respecto al eje \(y\)}

Si
\[
x(-t) = -x(t)
\quad \text{y} \quad y(-t)=y(t),
\]
la curva es simétrica respecto al eje \(y\).

\subsubsection*{Simetría central}

Si \(\vec r(-t)=-\vec r(t)\), la curva tiene simetría respecto al origen.

\begin{InfoBox}
Analizar simetrías ayuda a simplificar integrales, predecir comportamientos y
entender la estructura global de la curva.
\end{InfoBox}

\subsubsection{Análisis estructural de curvas paramétricas}

Una curva puede presentar:

\begin{itemize}
    \item puntos de retorno,
    \item auto–intersecciones,
    \item puntos donde \(\vec r'(t)=0\),
    \item períodos de oscilación,
    \item regiones cerradas,
    \item comportamientos asintóticos.
\end{itemize}

Estos rasgos definen la topología de la curva y son esenciales para detectar
propiedades globales.

\subsubsection*{Auto–intersecciones}

Una auto–intersección ocurre cuando existen \(t_1\neq t_2\) tales que:

\[
\vec r(t_1)=\vec r(t_2).
\]

Ejemplos clásicos:

\begin{itemize}
    \item curvas de Lissajous,
    \item trayectorias oscilatorias,
    \item figuras polares con varios pétalos.
\end{itemize}

\begin{InfoBox}
La parametrización revela el orden en que se recorren las ramas de una curva,
algo oculto en la representación cartesiana.
\end{InfoBox}

\subsubsection{Interpretación geométrica avanzada}

Las propiedades de una curva dependen de:

\begin{itemize}
    \item la forma de las ecuaciones paramétricas,
    \item el intervalo del parámetro,
    \item la rapidez con la que se recorre,
    \item la presencia de oscilaciones,
    \item la existencia de puntos críticos.
\end{itemize}

La geometría local depende de:

\[
\vec r'(t),
\qquad
\vec r''(t),
\qquad
\vec r'''(t).
\]

En particular:

\begin{itemize}
    \item \(\vec r'(t)\) determina dirección y rapidez,
    \item \(\vec r''(t)\) determina cambios de dirección,
    \item \(\vec r'''(t)\) permite estudiar inflexiones paramétricas.
\end{itemize}

\begin{InfoBox}
Una curva paramétrica permite analizar fenómenos que no son visibles en formas
cartesianas, como la orientación, la velocidad instantánea y la torsión.
\end{InfoBox}

\begin{EjercicioBox}[Ejemplo: auto–intersección]
Considere \(\vec r(t)=\langle \sin(2t), \sin(3t)\rangle\).  
Demuestre que existen valores distintos de \(t\) que producen el mismo punto,
lo cual indica que la curva tiene auto–intersecciones.

Sugerencia: la naturaleza periódica de las funciones seno con frecuencias
distintas produce patrones cerrados que se repiten en distintos valores de \(t\).
\end{EjercicioBox}

\subsubsection{Aplicaciones matemáticas avanzadas}

El uso de parametrizaciones es esencial en diversas ramas de las matemáticas:

\begin{itemize}
    \item en integración sobre curvas,
    \item en optimización con restricciones,
    \item en geometría diferencial,
    \item en el estudio de intersecciones entre superficies,
    \item en métodos numéricos para aproximación de trayectorias,
    \item en construcción de curvas suaves (splines y Bézier).
\end{itemize}

\subsubsection*{Intersecciones entre superficies}

Dadas dos superficies \(S_1\) y \(S_2\), su intersección suele ser una curva.
La parametrización de dicha curva facilita tanto su análisis como la
realización de cálculos integrales a lo largo de ella.

Ejemplo general:

Si
\[
S_1: F(x,y,z)=0, \qquad S_2: G(x,y,z)=0,
\]
entonces la curva intersección puede parametrizarse como
\[
\vec r(t)=\langle x(t), y(t), z(t)\rangle
\]
satisfaciendo ambas ecuaciones.

\begin{InfoBox}
La parametrización convierte un sistema de ecuaciones implícitas en una curva
explícita y manejable.
\end{InfoBox}

\subsubsection*{Curvas de nivel parametrizadas}

En problemas multivariables, una curva puede describirse como:
\[
f(x,y)=c.
\]

Una parametrización adecuada permite calcular integrales, longitudes de arco y
derivadas direccionales a lo largo de la curva.

\begin{TemaBox}[Teorema: Parametrización de curvas de nivel]
Sea \(f:\mathbb{R}^2\to\mathbb{R}\) de clase \(C^1\).
Si \(\nabla f(P)\neq 0\), entonces existe una parametrización suave de la curva
nivel que pasa por \(P\).
\end{TemaBox}

\textbf{Demostración:}  
Por el teorema de la función implícita existe una función que describe localmente
la curva, lo que garantiza la parametrización. \(\square\)

\subsubsection{Aplicaciones físicas avanzadas}

Las ecuaciones paramétricas permiten modelar fenómenos físicos complejos:

\begin{itemize}
    \item trayectorias bajo campos electromagnéticos,
    \item órbitas en mecánica celeste,
    \item oscilaciones no lineales,
    \item movimiento de partículas relativistas,
    \item propagación de señales en medios no homogéneos.
\end{itemize}

\begin{InfoBox}
Las parametrizaciones no solo describen dónde está un objeto, sino cómo se
mueve, con qué velocidad y cómo cambia su dirección.
\end{InfoBox}

\subsubsection{Reparametrización en curvas especiales}

Algunas curvas requieren un parámetro diferente al estándar. Por ejemplo, las
curvas que presentan oscilaciones periódicas pueden reparametrizarse usando su
frecuencia natural.

Ejemplo general:

Si una curva oscila con frecuencia \(\omega\), una reparametrización común es
\[
t = \omega u,
\]
lo que normaliza la velocidad angular y simplifica los cálculos.

\begin{InfoBox}
Las reparametrizaciones ayudan a comparar curvas diferentes y a sincronizar
movimientos periódicos.
\end{InfoBox}

\subsubsection{Estudio de la rapidez y la orientación}

La rapidez depende de la derivada del parámetro:

\[
\|\vec r'(t)\| \quad \text{puede cambiar aun cuando la curva no cambie.}
\]

Si se desea mantener rapidez constante, se utiliza la parametrización por
longitud de arco:

\[
s(t)=\int_{t_0}^t \|\vec r'(\tau)\| d\tau.
\]

Luego se invierte para obtener \(t=t(s)\).

\subsubsection{Observaciones finales sobre ecuaciones paramétricas}

Las ecuaciones paramétricas:

\begin{itemize}
    \item describen curvas imposibles de representar en forma cartesiana;
    \item permiten estudiar orientación, velocidad y aceleración;
    \item facilitan el estudio de auto–intersecciones y comportamientos globales;
    \item son fundamentales en física, ingeniería y computación gráfica;
    \item sirven como puente hacia el cálculo vectorial avanzado.
\end{itemize}

\begin{InfoBox}
En geometría diferencial, la descripción paramétrica es el lenguaje natural de
las curvas y superficies.
\end{InfoBox}

\begin{EjercicioBox}[Ejercicios propuestos]
\begin{enumerate}
    \item Determine la pendiente paramétrica de la curva
    \(x=\cos t\), \(y=\sin(2t)\).

    \item Halle la rapidez de la curva \(\vec r(t)=\langle t^2, e^t, \cos t\rangle\).

    \item Identifique puntos de auto–intersección en la curva
    \(\vec r(t)=\langle \sin t, \sin(2t)\rangle\).

    \item Clasifique la simetría de la curva \(x=\sin t\), \(y=\cos(2t)\).

    \item Encuentre una reparametrización conveniente para la curva
    \(x=\cos(3t)\), \(y=\sin(3t)\), de modo que se recorra con rapidez unidad.
\end{enumerate}
\end{EjercicioBox}

\subsection{Cálculo en Funciones Vectoriales}

\subsubsection{Velocidad y aceleración}

Dada una función vectorial
\[
\vec r(t)=\langle x(t), y(t), z(t)\rangle,
\]
la velocidad está definida como la derivada:

\[
\vec v(t)=\vec r'(t)=\langle x'(t), y'(t), z'(t)\rangle,
\]

y la aceleración como:

\[
\vec a(t)=\vec r''(t).
\]

La rapidez es la norma de la velocidad:

\[
\|\vec v(t)\|=\sqrt{x'(t)^2 + y'(t)^2 + z'(t)^2}.
\]

\begin{InfoBox}
La velocidad describe la dirección del movimiento y su magnitud describe la
rapidez. La aceleración captura los cambios tanto en dirección como en rapidez.
\end{InfoBox}

\subsubsection{Descomposición de la aceleración}

La aceleración puede descomponerse en dos componentes fundamentales:

\[
\vec a(t)
=
a_T(t)\vec T(t)
+
a_N(t)\vec N(t),
\]
donde:

\[
a_T(t)=\frac{d}{dt}\|\vec v(t)\|,
\qquad
a_N(t)=\|\vec v(t)\|^2\kappa(t).
\]

\begin{InfoBox}
Incluso si la rapidez es constante, la aceleración puede ser grande debido a la
curvatura. Esto es típico en movimientos circulares.
\end{InfoBox}

\subsubsection{Derivadas superiores}

La tercera derivada \(\vec r'''(t)\) permite analizar la torsión de la curva y
describir comportamientos espaciales más complejos.

Ejemplos:

\begin{itemize}
    \item Si \(\vec r'''(t)=0\), la curva es un polinomio vectorial de grado 2.
    \item Si \(\vec r''(t)\) es constante, la trayectoria es parabólica.
    \item Si \(\vec r''(t)\) y \(\vec r'''(t)\) son perpendiculares, surgen puntos de inflexión.
\end{itemize}

\subsubsection{Curvatura en términos de derivadas}

La curvatura puede definirse directamente en términos de \(\vec r'(t)\) y
\(\vec r''(t)\):

\[
\kappa(t)=
\frac{\|\vec r'(t)\times \vec r''(t)\|}
{\|\vec r'(t)\|^3}.
\]

\begin{TemaBox}[Teorema: Fórmula de curvatura]
Sea \(\vec r(t)\) regular y suave. Entonces la curvatura está dada por:
\[
\kappa(t)=
\frac{\|\vec r'(t)\times \vec r''(t)\|}
{\|\vec r'(t)\|^3}.
\]
\end{TemaBox}

\textbf{Demostración:}  
Se deriva el vector tangente \(\vec T(t)=\vec r'(t)/\|\vec r'(t)\|\) y se
proyecta sobre la dirección normal. El módulo del componente normal produce la
expresión dada. \(\square\)

\begin{InfoBox}
La fórmula es especialmente útil para curvas espaciales donde la derivada del
vector tangente es complicada de calcular directamente.
\end{InfoBox}

\subsubsection{Torsión en términos de derivadas}

Para curvas tridimensionales, la torsión se expresa como:

\[
\tau(t)=
\frac{(\vec r'(t)\times \vec r''(t))\cdot \vec r'''(t)}
{\|\vec r'(t)\times \vec r''(t)\|^2}.
\]

La torsión mide la desviación de la curva fuera del plano osculador.

\subsubsection{Longitud de arco}

La longitud de arco entre \(t=a\) y \(t=b\) se obtiene mediante:

\[
L=\int_a^b \|\vec r'(t)\|\, dt.
\]

Una curva puede reparametrizarse mediante longitud de arco definiendo:

\[
s(t)=\int_{t_0}^t \|\vec r'(\tau)\|d\tau.
\]

\subsubsection{Ecuaciones de Frenet–Serret}

El marco de Frenet está dado por:

\[
\vec T(t), \qquad \vec N(t), \qquad \vec B(t)=\vec T(t)\times \vec N(t).
\]

Satisface:

\[
\vec T'(t)=\kappa(t)\vec N(t),
\]
\[
\vec N'(t)=-\kappa(t)\vec T(t)+\tau(t)\vec B(t),
\]
\[
\vec B'(t)=-\tau(t)\vec N(t).
\]

\begin{InfoBox}
Estas ecuaciones describen completamente la geometría local de la curva.
\end{InfoBox}

\begin{EjercicioBox}[Ejemplo: velocidad y aceleración]
Sea \(\vec r(t)=\langle e^t,\cos t,\sin t\rangle\).
Calcule \(\vec v(t)\) y \(\vec a(t)\).

\[
\vec v(t)=\langle e^t,-\sin t,\cos t\rangle,
\qquad
\vec a(t)=\langle e^t,-\cos t,-\sin t\rangle.
\]
\end{EjercicioBox}

\subsubsection{Análisis geométrico mediante derivadas}

La primera y segunda derivada permiten identificar rasgos geométricos como:

\begin{itemize}
    \item puntos de inflexión,
    \item cambios bruscos de dirección,
    \item zonas donde la rapidez disminuye o aumenta,
    \item regiones donde la curva se aplana.
\end{itemize}

Sea \(\vec r(t)\) una curva suave:

\[
\vec r'(t) \quad \text{dirección y rapidez,}
\]
\[
\vec r''(t) \quad \text{cambios en la dirección del movimiento.}
\]

Un punto de la curva es crítico si
\[
\vec r'(t_0)=\vec 0.
\]

\begin{InfoBox}
Los puntos críticos indican cambios cualitativos importantes en la trayectoria.
Pueden corresponder a cuspides, detenciones o giros abruptos.
\end{InfoBox}

\subsubsection{Relación entre velocidad, curvatura y torsión}

La geometría de una curva tridimensional está gobernada por tres funciones:

\[
\|\vec r'(t)\|, \qquad \kappa(t), \qquad \tau(t).
\]

La rapidez describe la magnitud del movimiento, la curvatura describe cómo se
dobla la curva, y la torsión describe cómo la curva sale del plano osculador.

\begin{TemaBox}[Teorema: Descomposición de la aceleración]
La aceleración puede expresarse como:
\[
\vec a(t)
=
a_T(t)\vec T(t)
+
a_N(t)\vec N(t),
\]
donde
\[
a_T(t)=\frac{d}{dt}\|\vec v(t)\|,
\qquad
a_N(t)=\|\vec v(t)\|^2\kappa(t).
\]
\end{TemaBox}

\textbf{Demostración:}  
Se descompone la aceleración en las direcciones tangente y normal y se usa la
definición de curvatura. \(\square\)

\subsubsection{Interpretación física avanzada}

Sea una partícula cuya posición está dada por \(\vec r(t)\). Entonces:

\[
\vec v(t) \text{ representa la velocidad instantánea.}
\]
\[
\vec a(t) \text{ representa la aceleración total.}
\]

La descomposición de la aceleración implica:

\begin{itemize}
    \item \(a_T\): responsable del cambio de rapidez (aceleración tangencial).
    \item \(a_N\): responsable del cambio de dirección (aceleración normal).
\end{itemize}

\begin{InfoBox}
Un automóvil que gira a velocidad constante experimenta aceleración normal, no
tangencial. Esta distinción es crucial en física e ingeniería.
\end{InfoBox}

Ejemplos de fenómenos donde estos conceptos son esenciales:

\begin{itemize}
    \item movimiento circular uniforme,
    \item órbitas planetarias,
    \item trayectorias de partículas cargadas en campos magnéticos,
    \item movimientos oscilatorios tridimensionales.
\end{itemize}

\subsubsection{Puntos críticos y regularidad}

Sean los puntos donde \(\vec r'(t)=\vec 0\). Tales puntos requieren un análisis
especial porque:

\begin{itemize}
    \item la dirección tangente es indefinida,
    \item pueden aparecer cuspides,
    \item pueden surgir puntos de retorno,
    \item la curva puede cruzarse consigo misma.
\end{itemize}

\begin{TemaBox}[Teorema: Regularidad y tangentes]
Si \(\vec r'(t_0)\neq 0\), la curva tiene tangente bien definida en \(t_0\).  
Si \(\vec r'(t_0)=0\), la curva puede perder suavidad en ese punto.
\end{TemaBox}

\textbf{Demostración:}  
La existencia del vector tangente requiere derivada no nula. Cuando se anula,
pueden formarse cuspides o cambios abruptos de dirección. \(\square\)

\subsubsection{Comportamientos extremos de la curva}

Las derivadas de orden superior permiten analizar:

\begin{itemize}
    \item zonas donde la curva se aplana (\(\kappa\approx 0\)),
    \item zonas donde la curva gira rápidamente (\(\kappa\) grande),
    \item zonas donde la curva vibra (alta variación de \(\tau\)),
    \item trayectorias casi rectilíneas,
    \item comportamientos helicoidales,
    \item zonas de oscilación periódica.
\end{itemize}

Una curva espaciales puede presentar fenómenos como:

\begin{itemize}
    \item torsión con signo variable,
    \item inflexiones espaciales,
    \item periodicidad tridimensional,
    \item regiones donde la curvatura se anula.
\end{itemize}

\begin{InfoBox}
La geometría local de una curva es completamente descrita por su rapidez,
curvatura y torsión. Esto permite reconstruirla salvo una transformación rígida.
\end{InfoBox}

\begin{EjercicioBox}[Ejemplo: descomposición de la aceleración]
Considere \(\vec r(t)=\langle \cos t,\sin t, bt\rangle\).
Determine \(a_T\) y \(a_N\).

\[
\vec v(t)=\langle -\sin t,\cos t,b\rangle,
\qquad
\|\vec v(t)\|=\sqrt{1+b^2}.
\]

Como la rapidez es constante:
\[
a_T=0.
\]

La curvatura de la hélice:
\[
\kappa=\frac{1}{1+b^2}.
\]

Entonces:
\[
a_N=\|\vec v(t)\|^2\kappa=\frac{1+b^2}{1+b^2}=1.
\]
\end{EjercicioBox}

\subsubsection{Aplicaciones matemáticas avanzadas}

El cálculo vectorial aplicado a funciones vectoriales es esencial en múltiples
ramas de las matemáticas:

\begin{itemize}
    \item geometría diferencial,
    \item análisis de curvas y superficies,
    \item óptica geométrica,
    \item teoría de campos,
    \item métodos numéricos para trayectorias,
    \item interpolación geométrica.
\end{itemize}

\subsubsection*{Curvas como soluciones de EDO}

Muchas ecuaciones diferenciales ordinarias generan curvas como soluciones. Por
ejemplo, la ecuación
\[
\vec r''(t)=k\vec r(t)
\]
produce curvas exponenciales, sinusoidales u oscilatorias según el signo de
\(k\).

\subsubsection*{Curvas de mínima energía}

En problemas variacionales, se buscan curvas que minimicen funcionales del tipo:

\[
E[\vec r]=\int_a^b \|\vec r''(t)\|^2 dt.
\]

Estas curvas aparecen en física, en elasticidad y en diseño industrial.

\begin{InfoBox}
Las curvas que minimizan energías elásticas tienden a ser suaves, regulares y
con variación controlada de curvatura.
\end{InfoBox}

\subsubsection{Aplicaciones físicas avanzadas}

Las funciones vectoriales describen movimientos en física clásica,
electromagnetismo, mecánica celeste y relatividad.

\subsubsection*{Movimiento bajo fuerzas centrales}

Una fuerza central satisface:
\[
\vec F(\vec r)=f(\|\vec r\|)\vec r.
\]

La trayectoria es una curva plana y puede analizarse mediante funciones
paramétricas. Ejemplos:

\begin{itemize}
    \item órbitas elípticas,
    \item órbitas parabólicas,
    \item órbitas hiperbólicas.
\end{itemize}

\subsubsection*{Trayectorias de partículas cargadas}

En un campo magnético uniforme, una partícula describe una hélice:

\[
\vec r(t)=\langle a\cos(\omega t), a\sin(\omega t), ct\rangle.
\]

\begin{InfoBox}
La curvatura se relaciona directamente con la intensidad del campo magnético.
\end{InfoBox}

\subsubsection{Curvas especiales estudiadas en cálculo diferencial}

Existen curvas cuyas propiedades motivan el desarrollo de toda la teoría:

\subsubsection*{Curvas planas}

Cuando \(\tau(t)=0\), la curva es plana. Ejemplos:

\begin{itemize}
    \item líneas rectas,
    \item circunferencias,
    \item parábolas,
    \item espirales planas.
\end{itemize}

\subsubsection*{Curvas espaciales generalizadas}

Cuando tanto la torsión como la curvatura son no nulas, surgen curvas:

\begin{itemize}
    \item helicoidales,
    \item toroidales,
    \item curvas oscilatorias tridimensionales.
\end{itemize}

\begin{TemaBox}[Teorema: Determinación de la curva]
Dados \(\kappa(t)\) y \(\tau(t)\), existe una curva única (salvo rigidez) cuya
curvatura y torsión son las dadas.
\end{TemaBox}

\textbf{Demostración:}  
Las ecuaciones de Frenet–Serret definen un sistema que, junto con condiciones
iniciales, determina la curva completamente. \(\square\)

\begin{EjercicioBox}[Ejemplo: curva plana con aceleración normal]
Sea \(\vec r(t)=\langle t, t^2, 0\rangle\).
Determine si la curva es plana y calcule la curvatura.

Como \(z(t)=0\), \(\tau(t)=0\) siempre; la curva es plana.  
\[
\kappa(t)
=
\frac{\|\vec r'(t)\times \vec r''(t)\|}
     {\|\vec r'(t)\|^3}.
\]

\[
\vec r'(t)=\langle 1,2t,0\rangle,
\qquad
\vec r''(t)=\langle 0,2,0\rangle.
\]

\[
\vec r'(t)\times \vec r''(t)
=
\langle 0,0,2\rangle.
\]

\[
\kappa(t)
=
\frac{2}{(1+4t^2)^{3/2}}.
\]
\end{EjercicioBox}

\begin{EjercicioBox}[Ejercicios propuestos]
\begin{enumerate}
    \item Calcule la torsión de la curva
    \(\vec r(t)=\langle t, t^2, t^3\rangle\).

    \item Determine la longitud de arco de la curva
    \(\vec r(t)=\langle \cos t,\sin t, bt\rangle\) en un intervalo dado.

    \item Analice la descomposición de la aceleración en la curva
    \(\vec r(t)=\langle e^t, \cos t, \sin t\rangle\).

    \item Verifique si \(\vec r(t)=\langle t^3,t^2,t\rangle\) tiene puntos
    críticos.

    \item Encuentre la curvatura de la curva
    \(\vec r(t)=\langle 3t^2, 2t, 1\rangle\) y clasifique su comportamiento.
\end{enumerate}
\end{EjercicioBox}

\subsection{Integral de Línea}

\subsubsection{Introducción general}

La integral de línea es una herramienta fundamental para integrar cantidades
a lo largo de una curva en el plano o en el espacio. Permite sumar:

\begin{itemize}
    \item trabajo realizado por una fuerza,
    \item masa de un alambre curvo,
    \item flujo tangencial de un campo vectorial,
    \item variaciones de campos escalares a lo largo de trayectorias.
\end{itemize}

Sea una curva parametrizada \(\vec r(t)\), \(a\le t\le b\).  
La integral de un campo vectorial \(\vec F\) sobre la curva se define como:

\[
\int_C \vec F\cdot d\vec r = \int_a^b \vec F(\vec r(t))\cdot \vec r'(t)\, dt.
\]

\begin{InfoBox}
La integral de línea toma en cuenta no solo el campo vectorial, sino también la
dirección y la rapidez con la que se recorre la curva.
\end{InfoBox}

\subsubsection{Integral de campos escalares}

Si una curva \(C\) está descrita por \(\vec r(t)\), la integral de un campo
escalar \(f\) a lo largo de la curva es:

\[
\int_C f\, ds = \int_a^b f(\vec r(t))\|\vec r'(t)\|\, dt.
\]

Interpretaciones comunes:

\begin{itemize}
    \item masa de un alambre con densidad variable \(f\),
    \item longitud ponderada,
    \item acumulación de una magnitud sobre una trayectoria.
\end{itemize}

\subsubsection{Integral de campos vectoriales}

Sea \(\vec F=\langle P,Q,R\rangle\). La integral de línea tangencial es:

\[
\int_C \vec F\cdot d\vec r
=
\int_C P\,dx + Q\,dy + R\,dz.
\]

Usando la parametrización:

\[
dx = x'(t)\,dt,\quad dy=y'(t)\,dt,\quad dz=z'(t)\,dt,
\]

queda:

\[
\int_C \vec F\cdot d\vec r
=
\int_a^b
\left[
P(\vec r(t))x'(t)
+
Q(\vec r(t))y'(t)
+
R(\vec r(t))z'(t)
\right]dt.
\]

\begin{InfoBox}
La orientación de la curva es crucial: invertir el sentido cambia el signo de
la integral.
\end{InfoBox}

\subsubsection{Dependencia del camino}

Una característica esencial de las integrales de línea es que, en general,
dependen de la trayectoria seguida. Si dos curvas \(C_1\) y \(C_2\) unen los
puntos \(A\) y \(B\), entonces normalmente:

\[
\int_{C_1} \vec F\cdot d\vec r
\ne
\int_{C_2} \vec F\cdot d\vec r.
\]

Ejemplo típico: fuerzas no conservativas, rozamiento o campos rotacionales.

\subsubsection{Campos conservativos}

Un campo es conservativo si existe una función potencial \(\varphi\) tal que:

\[
\vec F = \nabla \varphi.
\]

En este caso:

\[
\int_C \vec F\cdot d\vec r
=
\varphi(B)-\varphi(A),
\]

independientemente del camino.

\begin{TemaBox}[Teorema fundamental para campos conservativos]
Si \(\vec F = \nabla \varphi\) en un dominio simplemente conexo, entonces:
\[
\int_C \vec F\cdot d\vec r
=
\varphi(\vec r(b)) - \varphi(\vec r(a)).
\]
\end{TemaBox}

\textbf{Demostración:}  
Por la regla de la cadena,
\[
\frac{d}{dt}\varphi(\vec r(t))
=
\nabla\varphi(\vec r(t))\cdot \vec r'(t)
=
\vec F(\vec r(t))\cdot \vec r'(t).
\]
Integrando de \(a\) a \(b\) se obtiene el resultado. \(\square\)

\begin{InfoBox}
En física, un campo conservativo realiza un trabajo independiente del camino:
solo importan los puntos inicial y final.
\end{InfoBox}

\subsubsection{Criterio de conservatividad}

En dos dimensiones:

\[
\vec F = \langle P,Q\rangle
\quad\text{es conservativo si y solo si}\quad
\frac{\partial P}{\partial y}
=
\frac{\partial Q}{\partial x}.
\]

En tres dimensiones:

\[
\vec F \text{ es conservativo } \iff \nabla\times \vec F = \vec 0.
\]

Esto funciona en dominios simplemente conexos.

\begin{InfoBox}
El rotacional nulo implica ausencia de “vórtices” en el campo, lo que
permite definir un potencial escalar.
\end{InfoBox}

\begin{EjercicioBox}[Ejemplo: campo conservativo]
Sea \(\vec F=\langle 2x,2y\rangle\).  
Determine si es conservativo y calcule su integral entre
\(A=(-1,0)\) y \(B=(1,2)\).

\[
\frac{\partial P}{\partial y}=0,
\qquad
\frac{\partial Q}{\partial x}=0,
\]
entonces es conservativo.

Un potencial es:
\[
\varphi(x,y)=x^2+y^2.
\]

\[
\int_C \vec F\cdot d\vec r
=
\varphi(B)-\varphi(A)
=
(1^2+2^2)-((-1)^2+0^2)=5-1=4.
\]
\end{EjercicioBox}

\subsubsection{Trabajo realizado por una fuerza}

Una aplicación fundamental de la integral de línea es el cálculo del trabajo:

\[
W = \int_C \vec F\cdot d\vec r.
\]

Si \(\vec F\) es la fuerza que actúa sobre una partícula con trayectoria
\(\vec r(t)\), entonces el trabajo se interpreta como:

\[
W = \int_a^b \vec F(\vec r(t))\cdot \vec r'(t)\, dt.
\]

Interpretación física:

\begin{itemize}
    \item trabajo positivo: la fuerza impulsa el movimiento,
    \item trabajo negativo: la fuerza se opone al movimiento,
    \item trabajo nulo: la fuerza es perpendicular a la trayectoria.
\end{itemize}

\begin{InfoBox}
Cuando la fuerza es conservativa, el trabajo depende únicamente de los puntos
inicial y final. Si no lo es, depende del camino completo.
\end{InfoBox}

\subsubsection{Integrales en curvas cerradas}

Si \(C\) es cerrada (se regresa al punto inicial), se usa la notación:

\[
\oint_C \vec F\cdot d\vec r.
\]

Propiedades importantes:

\begin{itemize}
    \item Si \(\vec F\) es conservativo, entonces:
    \[
    \oint_C \vec F\cdot d\vec r = 0.
    \]
    \item Si \(\vec F\) NO es conservativo, la integral puede ser positiva,
          negativa o nula dependiendo del campo.
    \item La orientación de la curva afecta el signo.
\end{itemize}

\begin{InfoBox}
Las integrales cerradas permiten detectar “rotación” o circulación en un campo
vectorial, lo cual es esencial en fluídos y electromagnetismo.
\end{InfoBox}

\subsubsection{Circulación y flujo tangencial}

La integral de línea evalúa la circulación de un campo a lo largo de la curva:

\[
\text{Circulación} = \oint_C \vec F\cdot d\vec r.
\]

Interpretación física:

\begin{itemize}
    \item mide cuánto “empuja” el campo en la dirección de la curva,
    \item detecta presencia de remolinos o vorticidad,
    \item permite medir la componente tangencial del campo.
\end{itemize}

\subsubsection{El Teorema de Green (introducción)}

El Teorema de Green relaciona una integral de línea sobre una curva cerrada con
una integral doble sobre la región encerrada.

Sea \(C\) una curva simple, cerrada y orientada positivamente (contrarreloj).
Sea \(\vec F=\langle P,Q\rangle\). Entonces:

\[
\oint_C P\, dx + Q\, dy
=
\iint_D \left( \frac{\partial Q}{\partial x}
-
\frac{\partial P}{\partial y} \right) dA.
\]

\begin{TemaBox}[Teorema de Green]
La circulación de un campo vectorial alrededor de una curva cerrada es igual al
rotacional integrado sobre la región encerrada.
\end{TemaBox}

\textbf{Demostración (idea):}  
Se divide la región \(D\) en rectángulos y se aplica el teorema fundamental del
cálculo a cada uno. Las contribuciones internas se cancelan y solo permanece la
frontera. \(\square\)

\subsubsection{Interpretación geométrica de Green}

La cantidad
\[
\frac{\partial Q}{\partial x} - \frac{\partial P}{\partial y}
\]
es el rotacional en 2D.

Interpretación:

\begin{itemize}
    \item Si es positivo, el campo induce un giro antihorario.
    \item Si es negativo, induce un giro horario.
    \item Si es cero, no hay giro neto.
\end{itemize}

\begin{InfoBox}
El Teorema de Green convierte un problema unidimensional (una curva) en uno
bidimensional (una región). Esto simplifica muchos cálculos y brinda una visión
geométrica más profunda.
\end{InfoBox}

\subsubsection{Conexión con campos conservativos}

Si \(\vec F=\langle P,Q\rangle\) es conservativo, entonces:

\[
\frac{\partial Q}{\partial x}
=
\frac{\partial P}{\partial y}.
\]

Sustituyendo en el Teorema de Green:

\[
\oint_C \vec F\cdot d\vec r
=
\iint_D 0 \, dA
=
0.
\]

Esto explica por qué los campos conservativos tienen integrales cerradas nulas.

\begin{EjercicioBox}[Ejemplo: circulación en un campo rotacional]
Sea \(\vec F=\langle -y, x\rangle\).  
Calcule \(\oint_C \vec F\cdot d\vec r\) donde \(C\) es la circunferencia unidad.

Parametrización:

\[
x=\cos t,\qquad y=\sin t,\qquad 0\le t\le 2\pi.
\]

\[
\vec r'(t)=\langle -\sin t,\cos t\rangle.
\]

\[
\vec F(\vec r(t))=\langle -\sin t, \cos t\rangle.
\]

\[
\vec F\cdot \vec r'(t)
=
(-\sin t)(-\sin t)+(\cos t)(\cos t)=1.
\]

\[
\oint_C \vec F\cdot d\vec r = \int_0^{2\pi} 1\, dt = 2\pi.
\]

\end{EjercicioBox}

\subsubsection{Integrales de línea en el espacio}

Para una curva espacial \(C\subset\mathbb{R}^3\) descrita por una
parametrización
\[
\vec r(t)=\langle x(t), y(t), z(t)\rangle,
\qquad a\le t\le b,
\]
la integral de línea de un campo vectorial \(\vec F=\langle P,Q,R\rangle\) es:

\[
\int_C \vec F\cdot d\vec r
=
\int_a^b \left[
P(\vec r(t))x'(t)
+
Q(\vec r(t))y'(t)
+
R(\vec r(t))z'(t)
\right] dt.
\]

\begin{InfoBox}
El cálculo es idéntico al caso bidimensional: la única diferencia es la
presencia del tercer componente. La interpretación geométrica es la misma.
\end{InfoBox}

\subsubsection{Conexión con gradientes}

Si un campo es gradiente de un potencial:

\[
\vec F = \nabla \varphi,
\]

entonces:

\[
\int_C \vec F\cdot d\vec r = \varphi(B)-\varphi(A).
\]

Propiedades clave:

\begin{itemize}
    \item la integral no depende del camino,
    \item las curvas cerradas siempre dan como resultado cero,
    \item el trabajo es completamente recuperable,
    \item aparecen funciones potenciales en energía mecánica, eléctrica y gravitacional.
\end{itemize}

\subsubsection{Flujo tangencial y orientación}

La integral de línea mide la alineación entre el campo y la tangente.

\[
\vec F\cdot d\vec r = \|\vec F\|\|\vec r'\|\cos\theta\, dt.
\]

Interpretación:

\begin{itemize}
    \item si \(\theta=0\): el campo impulsa completamente la trayectoria,
    \item si \(\theta=\pi\): el campo se opone totalmente,
    \item si \(\theta=\pi/2\): no contribuye al trabajo.
\end{itemize}

\begin{InfoBox}
El ángulo entre el campo y la tangente determina el trabajo realizado por unidad de longitud recorrida.
\end{InfoBox}

\subsubsection{Aplicaciones físicas avanzadas}

Las integrales de línea aparecen naturalmente en:

\begin{itemize}
    \item trabajo realizado por fuerzas variables,
    \item circulación de fluidos alrededor de una curva,
    \item corriente inducida en electromagnetismo,
    \item energía gastada por un campo eléctrico,
    \item modelado de trayectorias con fricción dependiente del camino.
\end{itemize}

Ejemplos típicos:

\subsubsection*{Trabajo en campos gravitacionales}

\[
W = \int_C m\vec g\cdot d\vec r = mg(z_B - z_A).
\]

\subsubsection*{Trabajo en campos eléctricos}

\[
\vec F = q\vec E,
\qquad
W = q\int_C \vec E\cdot d\vec r.
\]

\subsubsection*{Circulación en dinámica de fluidos}

\[
\Gamma = \oint_C \vec v\cdot d\vec r.
\]

La circulación mide la tendencia del fluido a generar remolinos.

\begin{InfoBox}
La integral de línea es el puente entre cálculo vectorial y las ecuaciones de la física moderna: Maxwell, Navier–Stokes y la mecánica lagrangiana.
\end{InfoBox}

\subsubsection{Integral de línea y campos no conservativos}

Si \(\vec F\) no puede expresarse como gradiente, entonces:

\begin{itemize}
    \item la integral depende del camino,
    \item pueden existir regiones donde la integral cerrada sea positiva o negativa,
    \item aparece energía disipada (como fricción),
    \item ocurren efectos circulatorios (vorticidad).
\end{itemize}

El rotacional caracteriza la falta de conservatividad:

\[
\nabla\times \vec F \ne \vec 0.
\]

\subsubsection{Criterio espacial de conservatividad}

Un campo en \(\mathbb{R}^3\) es conservativo si y solo si:

\[
\nabla\times \vec F = \vec 0
\quad \text{en un dominio simplemente conexo}.
\]

\begin{TemaBox}[Teorema: equivalencia en 3D]
Un campo vectorial es conservativo si y solo si su rotacional es cero y el
dominio es simplemente conexo.
\end{TemaBox}

\textbf{Demostración (idea):}  
Usando la versión tridimensional del teorema de Poincaré, un rotacional nulo
garantiza la existencia de una función potencial. \(\square\)

\begin{EjercicioBox}[Ejemplo: integral en el espacio]
Calcule
\[
\int_C \vec F\cdot d\vec r,
\quad
\vec F=\langle yz, xz, xy\rangle,
\]
donde \(C\) es la curva
\[
\vec r(t)=\langle t, t^2, t^3\rangle,\quad 0\le t\le 1.
\]

\[
\vec r'(t)=\langle 1,2t,3t^2\rangle.
\]

\[
\vec F(\vec r(t))
=
\langle t^2\cdot t^3,\, t\cdot t^3,\, t\cdot t^2\rangle
=
\langle t^5, t^4, t^3\rangle.
\]

\[
\vec F\cdot \vec r'
=
t^5(1)+t^4(2t)+t^3(3t^2)
=
t^5+2t^5+3t^5
=
6t^5.
\]

\[
\int_0^1 6t^5\, dt = t^6\bigg|_0^1 = 1.
\]
\end{EjercicioBox}

\begin{EjercicioBox}[Ejercicios de repaso]
\begin{enumerate}
    \item Determine si \(\vec F=\langle yz, xz, xy\rangle\) es conservativo
          y calcule la integral entre dos puntos.

    \item Calcule la circulación de \(\vec F=\langle -y,x,0\rangle\) alrededor
          de una curva helicoidal.

    \item Evalúe \(\int_C f\, ds\) para \(f(x,y)=x^2+y^2\) sobre la curva
          \(\vec r(t)=\langle \cos t,\sin t\rangle\).

    \item Verifique el Teorema de Green para el campo \(\langle -y,x\rangle\) en una región rectangular.

    \item Clasifique un campo vectorial dado según sea conservativo, rotacional o irrotacional usando \(\nabla\times\vec F\).
\end{enumerate}
\end{EjercicioBox}