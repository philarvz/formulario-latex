% ==================== UNIDAD III ====================
% Integral Múltiple

\section{Unidad III: Integral Múltiple}

\begin{TemaBox}[Integrales Múltiples: Fundamentos]
Las integrales múltiples extienden el concepto de integración a funciones de varias variables, permitiendo calcular volúmenes, masas, centros de masa y otras cantidades físicas sobre regiones en el plano y en el espacio. Son herramientas fundamentales en física, ingeniería, economía y todas las áreas donde se requiere el cálculo de cantidades distribuidas sobre regiones multidimensionales.
\end{TemaBox}

\subsection{Integrales dobles}

\paragraph{Introducción.}
Las \textbf{integrales dobles} son la extensión natural de las integrales definidas a funciones de dos variables. Mientras que una integral definida calcula el área bajo una curva, una integral doble calcula el volumen bajo una superficie o cantidades distribuidas sobre una región plana.

\begin{InfoBox}
\Meta{Definición formal}{La integral doble de una función \(f(x,y)\) continua sobre una región \(R \subset \mathbb{R}^2\) se define como el límite de sumas de Riemann:}
\[
\iint_R f(x,y)\, dA = \lim_{m,n \to \infty} \sum_{i=1}^m \sum_{j=1}^n f(x_{ij}^\ast, y_{ij}^\ast)\Delta A,
\]
donde \(\Delta A = \Delta x \Delta y\) es el área de cada subrectángulo.

\Meta{Interpretación geométrica}{Si \(f(x,y) \geq 0\), la integral doble representa el \textbf{volumen} del sólido bajo la superficie \(z = f(x,y)\) y sobre la región \(R\) en el plano \(xy\).}
\end{InfoBox}

\subsubsection{Regiones rectangulares}

\begin{TemaBox}[Integración sobre regiones rectangulares]
Cuando la región de integración es un rectángulo \(R = [a,b] \times [c,d]\), la integral doble se puede calcular mediante dos integrales iteradas. El orden de integración puede ser \(dy\, dx\) o \(dx\, dy\), y bajo condiciones de continuidad, ambos órdenes dan el mismo resultado.
\end{TemaBox}

\paragraph{Teorema de Fubini para rectángulos.}
Si \(f(x,y)\) es continua en el rectángulo \(R = [a,b] \times [c,d]\), entonces:
\[
\iint_R f(x,y)\, dA = \int_a^b \int_c^d f(x,y)\, dy\, dx = \int_c^d \int_a^b f(x,y)\, dx\, dy.
\]

\paragraph{Ejemplos resueltos.}

\begin{EjercicioBox}[Ejemplo 1: Integral doble sobre rectángulo]
Calcular
\[
\iint_R (x + 2y)\, dA,
\]
donde \(R = [0,2] \times [1,3]\).

\textbf{Solución:}

Integramos primero con respecto a \(y\), luego con respecto a \(x\):
\[
\begin{aligned}
\iint_R (x + 2y)\, dA &= \int_0^2 \int_1^3 (x + 2y)\, dy\, dx \\
&= \int_0^2 \left[ xy + y^2 \right]_{y=1}^{y=3} dx \\
&= \int_0^2 \left[ (3x + 9) - (x + 1) \right] dx \\
&= \int_0^2 (2x + 8)\, dx \\
&= \left[ x^2 + 8x \right]_0^2 \\
&= 4 + 16 = \boxed{20}.
\end{aligned}
\]

\textbf{Verificación con orden inverso:}
\[
\begin{aligned}
\iint_R (x + 2y)\, dA &= \int_1^3 \int_0^2 (x + 2y)\, dx\, dy \\
&= \int_1^3 \left[ \frac{x^2}{2} + 2xy \right]_{x=0}^{x=2} dy \\
&= \int_1^3 (2 + 4y)\, dy \\
&= \left[ 2y + 2y^2 \right]_1^3 \\
&= (6 + 18) - (2 + 2) = 20.
\end{aligned}
\]
\end{EjercicioBox}

\begin{EjercicioBox}[Ejemplo 2: Función producto separable]
Calcular
\[
\iint_R xy^2\, dA,
\]
donde \(R = [1,3] \times [0,2]\).

\textbf{Solución:}

Como la función es producto \(xy^2 = x \cdot y^2\), podemos separar las integrales:
\[
\begin{aligned}
\iint_R xy^2\, dA &= \int_1^3 \int_0^2 xy^2\, dy\, dx \\
&= \int_1^3 x \left( \int_0^2 y^2\, dy \right) dx \\
&= \int_1^3 x \left[ \frac{y^3}{3} \right]_0^2 dx \\
&= \int_1^3 x \cdot \frac{8}{3} dx \\
&= \frac{8}{3} \left[ \frac{x^2}{2} \right]_1^3 \\
&= \frac{8}{3} \cdot \frac{9-1}{2} = \frac{8}{3} \cdot 4 = \boxed{\frac{32}{3}}.
\end{aligned}
\]
\end{EjercicioBox}

\subsubsection{Regiones no rectangulares}

\begin{TemaBox}[Integración sobre regiones generales]
En la práctica, muchas regiones no son rectangulares. Existen dos tipos principales:
\begin{enumerate}
  \item \textbf{Región tipo I:} Acotada por abajo y arriba por funciones de \(x\)
    \[
    R = \{(x,y) : a \leq x \leq b, g_1(x) \leq y \leq g_2(x)\}
    \]
  \item \textbf{Región tipo II:} Acotada por izquierda y derecha por funciones de \(y\)
    \[
    R = \{(x,y) : c \leq y \leq d, h_1(y) \leq x \leq h_2(y)\}
    \]
\end{enumerate}
\end{TemaBox}

\paragraph{Integración sobre regiones tipo I.}
Para una región tipo I:
\[
\iint_R f(x,y)\, dA = \int_a^b \int_{g_1(x)}^{g_2(x)} f(x,y)\, dy\, dx.
\]

\paragraph{Integración sobre regiones tipo II.}
Para una región tipo II:
\[
\iint_R f(x,y)\, dA = \int_c^d \int_{h_1(y)}^{h_2(y)} f(x,y)\, dx\, dy.
\]

\begin{EjercicioBox}[Ejemplo 3: Región tipo I — triangular]
Calcular
\[
\iint_R (x + y)\, dA,
\]
donde \(R\) es la región triangular con vértices \((0,0)\), \((2,0)\) y \((2,3)\).

\textbf{Solución:}

Primero identificamos los límites. La recta que pasa por \((0,0)\) y \((2,3)\) tiene ecuación:
\[
y = \frac{3}{2}x.
\]

La región puede describirse como tipo I:
\[
R = \{(x,y) : 0 \leq x \leq 2, 0 \leq y \leq \frac{3}{2}x\}.
\]

Calculamos la integral:
\[
\begin{aligned}
\iint_R (x + y)\, dA &= \int_0^2 \int_0^{\frac{3}{2}x} (x + y)\, dy\, dx \\
&= \int_0^2 \left[ xy + \frac{y^2}{2} \right]_0^{\frac{3}{2}x} dx \\
&= \int_0^2 \left( x \cdot \frac{3x}{2} + \frac{1}{2} \cdot \frac{9x^2}{4} \right) dx \\
&= \int_0^2 \left( \frac{3x^2}{2} + \frac{9x^2}{8} \right) dx \\
&= \int_0^2 \frac{21x^2}{8} dx \\
&= \frac{21}{8} \left[ \frac{x^3}{3} \right]_0^2 \\
&= \frac{21}{8} \cdot \frac{8}{3} = \boxed{7}.
\end{aligned}
\]
\end{EjercicioBox}

\begin{EjercicioBox}[Ejemplo 4: Región tipo II]
Calcular
\[
\iint_R x^2 y\, dA,
\]
donde \(R\) es la región acotada por \(y = x^2\) y \(y = 2x\).

\textbf{Solución:}

Primero encontramos los puntos de intersección:
\[
x^2 = 2x \Rightarrow x^2 - 2x = 0 \Rightarrow x(x-2) = 0 \Rightarrow x = 0, x = 2.
\]

Para \(x \in [0,2]\), tenemos \(x^2 \leq 2x\), por lo que la región es tipo I:
\[
R = \{(x,y) : 0 \leq x \leq 2, x^2 \leq y \leq 2x\}.
\]

\[
\begin{aligned}
\iint_R x^2 y\, dA &= \int_0^2 \int_{x^2}^{2x} x^2 y\, dy\, dx \\
&= \int_0^2 x^2 \left[ \frac{y^2}{2} \right]_{x^2}^{2x} dx \\
&= \int_0^2 x^2 \left( \frac{4x^2}{2} - \frac{x^4}{2} \right) dx \\
&= \int_0^2 \left( 2x^4 - \frac{x^6}{2} \right) dx \\
&= \left[ \frac{2x^5}{5} - \frac{x^7}{14} \right]_0^2 \\
&= \frac{64}{5} - \frac{128}{14} = \frac{64}{5} - \frac{64}{7} \\
&= \frac{448 - 320}{35} = \boxed{\frac{128}{35}}.
\end{aligned}
\]
\end{EjercicioBox}

\subsubsection{Integrales dobles en coordenadas polares}

\begin{TemaBox}[Coordenadas polares en integrales dobles]
Para regiones circulares o con simetría radial, el uso de coordenadas polares simplifica significativamente el cálculo. La transformación está dada por:
\[
x = r\cos\theta, \quad y = r\sin\theta, \quad dA = r\, dr\, d\theta.
\]

La integral doble en coordenadas polares es:
\[
\iint_R f(x,y)\, dA = \iint_S f(r\cos\theta, r\sin\theta) r\, dr\, d\theta,
\]
donde \(S\) es la región correspondiente en coordenadas polares.
\end{TemaBox}

\begin{EjercicioBox}[Ejemplo 5: Integral doble en coordenadas polares]
Calcular
\[
\iint_R e^{x^2+y^2}\, dA,
\]
donde \(R\) es el disco de radio 2 centrado en el origen.

\textbf{Solución:}

En coordenadas polares, \(R\) se describe como:
\[
S = \{(r,\theta) : 0 \leq r \leq 2, 0 \leq \theta \leq 2\pi\}.
\]

Notando que \(x^2 + y^2 = r^2\):
\[
\begin{aligned}
\iint_R e^{x^2+y^2}\, dA &= \int_0^{2\pi} \int_0^2 e^{r^2} r\, dr\, d\theta \\
&= \int_0^{2\pi} d\theta \int_0^2 re^{r^2} dr \\
&= 2\pi \cdot \left[ \frac{1}{2} e^{r^2} \right]_0^2 \\
&= 2\pi \cdot \frac{1}{2}(e^4 - 1) \\
&= \boxed{\pi(e^4 - 1)}.
\end{aligned}
\]
\end{EjercicioBox}

\begin{EjercicioBox}[Ejemplo 6: Área usando coordenadas polares]
Calcular el área de la región dentro del círculo \(r = 2\cos\theta\) y fuera del círculo \(r = 1\).

\textbf{Solución:}

La región está en coordenadas polares. Para encontrar los límites de \(\theta\), resolvemos:
\[
2\cos\theta = 1 \Rightarrow \cos\theta = \frac{1}{2} \Rightarrow \theta = \pm \frac{\pi}{3}.
\]

El área es:
\[
\begin{aligned}
A &= \int_{-\pi/3}^{\pi/3} \int_1^{2\cos\theta} r\, dr\, d\theta \\
&= \int_{-\pi/3}^{\pi/3} \left[ \frac{r^2}{2} \right]_1^{2\cos\theta} d\theta \\
&= \int_{-\pi/3}^{\pi/3} \left( 2\cos^2\theta - \frac{1}{2} \right) d\theta \\
&= \int_{-\pi/3}^{\pi/3} \left( 1 + \cos(2\theta) - \frac{1}{2} \right) d\theta \\
&= \int_{-\pi/3}^{\pi/3} \left( \frac{1}{2} + \cos(2\theta) \right) d\theta \\
&= \left[ \frac{\theta}{2} + \frac{\sin(2\theta)}{2} \right]_{-\pi/3}^{\pi/3} \\
&= \frac{\pi}{3} + \frac{\sqrt{3}}{2} = \boxed{\frac{\pi}{3} + \frac{\sqrt{3}}{2}}.
\end{aligned}
\]
\end{EjercicioBox}

\paragraph{Cambio de orden de integración.}
A veces es más conveniente cambiar el orden de integración. Esto requiere reescribir los límites de integración.

\begin{EjercicioBox}[Ejemplo 4b: Cambio de orden de integración]
Calcular
\[
\int_0^1 \int_{x^2}^1 x^3 y\, dy\, dx
\]
cambiando el orden de integración.

\textbf{Solución:}

Primero identificamos la región. Del orden original \(dy\, dx\):
\[
R = \{(x,y) : 0 \leq x \leq 1, x^2 \leq y \leq 1\}.
\]

Esta es una región tipo I. Para convertirla a tipo II (orden \(dx\, dy\)):
\[
R = \{(x,y) : 0 \leq y \leq 1, 0 \leq x \leq \sqrt{y}\}.
\]

\[
\begin{aligned}
\int_0^1 \int_{x^2}^1 x^3 y\, dy\, dx &= \int_0^1 \int_0^{\sqrt{y}} x^3 y\, dx\, dy \\
&= \int_0^1 y \left[ \frac{x^4}{4} \right]_0^{\sqrt{y}} dy \\
&= \int_0^1 y \cdot \frac{y^2}{4} dy \\
&= \frac{1}{4} \int_0^1 y^3 dy \\
&= \frac{1}{4} \left[ \frac{y^4}{4} \right]_0^1 = \boxed{\frac{1}{16}}.
\end{aligned}
\]
\end{EjercicioBox}

\paragraph{Propiedades de las integrales dobles.}
Las integrales dobles satisfacen propiedades similares a las integrales definidas:

\begin{itemize}
  \item \textbf{Linealidad:} \(\iint_R [f(x,y) + g(x,y)]\, dA = \iint_R f(x,y)\, dA + \iint_R g(x,y)\, dA\)
  \item \textbf{Constante:} \(\iint_R cf(x,y)\, dA = c\iint_R f(x,y)\, dA\)
  \item \textbf{Aditividad:} Si \(R = R_1 \cup R_2\) con \(R_1 \cap R_2 = \emptyset\), entonces \(\iint_R f(x,y)\, dA = \iint_{R_1} f(x,y)\, dA + \iint_{R_2} f(x,y)\, dA\)
  \item \textbf{Comparación:} Si \(f(x,y) \geq g(x,y)\) para todo \((x,y) \in R\), entonces \(\iint_R f(x,y)\, dA \geq \iint_R g(x,y)\, dA\)
\end{itemize}

\paragraph{Aplicaciones de las integrales dobles.}
Las integrales dobles tienen aplicaciones en:
\begin{itemize}
  \item \textbf{Volumen:} \(V = \iint_R f(x,y)\, dA\) cuando \(f(x,y) \geq 0\)
  \item \textbf{Área:} \(A = \iint_R 1\, dA\)
  \item \textbf{Masa de una lámina:} \(M = \iint_R \rho(x,y)\, dA\) donde \(\rho\) es la densidad
  \item \textbf{Centro de masa:} \((\bar{x}, \bar{y}) = \left( \frac{M_y}{M}, \frac{M_x}{M} \right)\) donde \(M_x = \iint_R y\rho(x,y)\, dA\) y \(M_y = \iint_R x\rho(x,y)\, dA\)
\end{itemize}

\subsubsection{Ejercicios: Integrales dobles}

Calcula las siguientes integrales:

\paragraph{Integrales indefinidas con funciones exponenciales y trigonométricas}

\begin{enumerate}
  \item \(\displaystyle \int e^{4x} \cos(3x + 2)\, dx\)
  
  \textbf{Solución:}
  Usamos integración por partes repetida. Sea \(u = e^{4x}\) y \(dv = \cos(3x+2)\, dx\).
  
  Entonces \(du = 4e^{4x}\, dx\) y \(v = \frac{\sin(3x+2)}{3}\).
  
  \[\int e^{4x}\cos(3x+2)\, dx = \frac{e^{4x}\sin(3x+2)}{3} - \frac{4}{3}\int e^{4x}\sin(3x+2)\, dx\]
  
  Aplicamos integración por partes nuevamente:
  
  \[\int e^{4x}\sin(3x+2)\, dx = -\frac{e^{4x}\cos(3x+2)}{3} + \frac{4}{3}\int e^{4x}\cos(3x+2)\, dx\]
  
  Sustituyendo y resolviendo:
  
  \[\int e^{4x}\cos(3x+2)\, dx = \frac{e^{4x}(4\sin(3x+2) + 3\cos(3x+2))}{25} + C\]
  
  \item \(\displaystyle \int e^{3x} \sin(4x + 2)\, dx\)
  
  \textbf{Solución:}
  Análogamente a la anterior, usando integración por partes repetida:
  
  \[\int e^{3x}\sin(4x+2)\, dx = \frac{e^{3x}(3\sin(4x+2) - 4\cos(4x+2))}{25} + C\]
  
  \item \(\displaystyle \int e^{-x} \cos(2x)\, dx\)
  
  \textbf{Solución:}
  Usamos integración por partes repetida.
  
  \[\int e^{-x}\cos(2x)\, dx = \frac{e^{-x}(-\cos(2x) - 2\sin(2x))}{5} + C = \frac{e^{-x}(-\cos(2x) - 2\sin(2x))}{5} + C\]
  
  \[\text{O equivalentemente: } \frac{-e^{-x}(\cos(2x) + 2\sin(2x))}{5} + C\]
\end{enumerate}

\paragraph{Integrales dobles iteradas}

\begin{enumerate}
  \setcounter{enumi}{3}
  
  \item \(\displaystyle \int_{x=0}^{4} \int_{y=1}^{3} 6x^2y\, dy\, dx\)
  
  \textbf{Solución:}
  Integramos primero con respecto a \(y\):
  
  \[\int_1^3 6x^2y\, dy = 6x^2 \left[ \frac{y^2}{2} \right]_1^3 = 6x^2 \left( \frac{9}{2} - \frac{1}{2} \right) = 6x^2 \cdot 4 = 24x^2\]
  
  Luego integramos con respecto a \(x\):
  
  \[\int_0^4 24x^2\, dx = 24 \left[ \frac{x^3}{3} \right]_0^4 = 8[4^3 - 0] = 8 \cdot 64 = 512\]
  
  \item \(\displaystyle \int_{x=0}^{4} \int_{y=2}^{3} (6x^2y - 2x)\, dy\, dx\)
  
  \textbf{Solución:}
  Integramos primero con respecto a \(y\):
  
  \[\int_2^3 (6x^2y - 2x)\, dy = \left[ 3x^2y^2 - 2xy \right]_2^3 = (3x^2 \cdot 9 - 2x \cdot 3) - (3x^2 \cdot 4 - 2x \cdot 2)\]
  \[= 27x^2 - 6x - 12x^2 + 4x = 15x^2 - 2x\]
  
  Luego integramos con respecto a \(x\):
  
  \[\int_0^4 (15x^2 - 2x)\, dx = \left[ 5x^3 - x^2 \right]_0^4 = 5 \cdot 64 - 16 = 320 - 16 = 304\]
  
  \item \(\displaystyle \int_{x=0}^{\pi/2} \int_{y=0}^{x} \sin(x - y)\, dy\, dx\)
  
  \textbf{Solución:}
  Integramos primero con respecto a \(y\):
  
  \[\int_0^x \sin(x-y)\, dy = \left[ \cos(x-y) \right]_0^x = \cos(0) - \cos(x) = 1 - \cos(x)\]
  
  Luego integramos con respecto a \(x\):
  
  \[\int_0^{\pi/2} (1 - \cos(x))\, dx = \left[ x - \sin(x) \right]_0^{\pi/2} = \frac{\pi}{2} - 1\]
  
  \item \(\displaystyle \int_{x=0}^{\pi/2} \int_{y=0}^{x} \sin(x + 2y)\, dy\, dx\)
  
  \textbf{Solución:}
  Integramos primero con respecto a \(y\):
  
  \[\int_0^x \sin(x+2y)\, dy = \left[ -\frac{\cos(x+2y)}{2} \right]_0^x = -\frac{\cos(3x)}{2} + \frac{\cos(x)}{2} = \frac{\cos(x) - \cos(3x)}{2}\]
  
  Luego integramos con respecto a \(x\):
  
  \[\int_0^{\pi/2} \frac{\cos(x) - \cos(3x)}{2}\, dx = \frac{1}{2}\left[ \sin(x) - \frac{\sin(3x)}{3} \right]_0^{\pi/2}\]
  
  \[= \frac{1}{2}\left( 1 - \frac{\sin(3\pi/2)}{3} - 0 + 0 \right) = \frac{1}{2}\left( 1 - \frac{-1}{3} \right) = \frac{1}{2} \cdot \frac{4}{3} = \frac{2}{3}\]
  
  \item \(\displaystyle \int_{x=0}^{2} \int_{y=0}^{x} e^{x^2} \sin(2y)\, dy\, dx\)
  
  \textbf{Solución:}
  Integramos primero con respecto a \(y\):
  
  \[\int_0^x e^{x^2}\sin(2y)\, dy = e^{x^2} \left[ -\frac{\cos(2y)}{2} \right]_0^x = e^{x^2}\left( -\frac{\cos(2x)}{2} + \frac{1}{2} \right) = \frac{e^{x^2}(1 - \cos(2x))}{2}\]
  
  Luego integramos con respecto a \(x\):
  
  \[\int_0^2 \frac{e^{x^2}(1 - \cos(2x))}{2}\, dx\]
  
  Esta integral no tiene una forma cerrada simple. Se puede aproximar numéricamente o expresar en términos de funciones especiales como la función error.
  
  \textbf{Valor aproximado: } \(\approx 1.855\)
\end{enumerate}


\subsection{Integrales triples}

\begin{TemaBox}[Integrales triples: Extensión al espacio]
Las integrales triples extienden el concepto de integración a funciones de tres variables sobre regiones sólidas en el espacio tridimensional. Son esenciales para calcular volúmenes, masas, centros de masa, momentos de inercia y otras cantidades físicas distribuidas en el espacio.
\end{TemaBox}

\paragraph{Definición formal.}
La integral triple de una función \(f(x,y,z)\) continua sobre una región sólida \(V \subset \mathbb{R}^3\) se define como:
\[
\iiint_V f(x,y,z)\, dV = \lim_{m,n,p \to \infty} \sum_{i=1}^m \sum_{j=1}^n \sum_{k=1}^p f(x_{ijk}^\ast, y_{ijk}^\ast, z_{ijk}^\ast) \Delta V,
\]
donde \(\Delta V = \Delta x \Delta y \Delta z\) es el volumen de cada subcaja.

\begin{InfoBox}
\Meta{Interpretación física}{Si \(f(x,y,z) = \rho(x,y,z)\) representa la densidad de masa en un punto \((x,y,z)\), entonces la integral triple da la masa total del sólido.}

\Meta{Interpretación geométrica}{Si \(f(x,y,z) = 1\), la integral triple da el volumen de la región sólida \(V\).}
\end{InfoBox}

\subsubsection{Regiones sólidas rectangulares}

\paragraph{Teorema de Fubini para sólidos rectangulares.}
Si \(f(x,y,z)\) es continua en el bloque rectangular \(V = [a,b] \times [c,d] \times [e,f]\), entonces:
\[
\iiint_V f(x,y,z)\, dV = \int_a^b \int_c^d \int_e^f f(x,y,z)\, dz\, dy\, dx.
\]

El orden de integración puede cambiarse siempre que se ajusten los límites correspondientes.

\begin{EjercicioBox}[Ejemplo 7: Integral triple sobre bloque rectangular]
Calcular
\[
\iiint_V (x + y + z)\, dV,
\]
donde \(V = [0,1] \times [0,2] \times [0,3]\).

\textbf{Solución:}

\[
\begin{aligned}
\iiint_V (x + y + z)\, dV &= \int_0^1 \int_0^2 \int_0^3 (x + y + z)\, dz\, dy\, dx \\
&= \int_0^1 \int_0^2 \left[ xz + yz + \frac{z^2}{2} \right]_0^3 dy\, dx \\
&= \int_0^1 \int_0^2 (3x + 3y + \frac{9}{2}) dy\, dx \\
&= \int_0^1 \left[ 3xy + \frac{3y^2}{2} + \frac{9y}{2} \right]_0^2 dx \\
&= \int_0^1 (6x + 6 + 9) dx \\
&= \int_0^1 (6x + 15) dx \\
&= \left[ 3x^2 + 15x \right]_0^1 \\
&= 3 + 15 = \boxed{18}.
\end{aligned}
\]
\end{EjercicioBox}

\subsubsection{Regiones sólidas generales}

\begin{TemaBox}[Tipos de regiones sólidas]
Las regiones sólidas se clasifican según cómo están acotadas:
\begin{enumerate}
  \item \textbf{Tipo I:} Acotada por arriba y abajo por superficies \(z = g_1(x,y)\) y \(z = g_2(x,y)\), y lateralmente por una región \(D\) en el plano \(xy\)
  \item \textbf{Tipo II:} Similar, pero con acotaciones en \(y\) o \(x\) según la proyección
\end{enumerate}
\end{TemaBox}

\paragraph{Región sólida tipo I.}
Para una región sólida tipo I:
\[
\iiint_V f(x,y,z)\, dV = \iint_D \int_{g_1(x,y)}^{g_2(x,y)} f(x,y,z)\, dz\, dA,
\]
donde \(D\) es la proyección de \(V\) sobre el plano \(xy\).

\begin{EjercicioBox}[Ejemplo 8: Volumen de un tetraedro]
Calcular el volumen del tetraedro con vértices \((0,0,0)\), \((1,0,0)\), \((0,2,0)\) y \((0,0,3)\).

\textbf{Solución:}

Primero encontramos la ecuación del plano que pasa por \((1,0,0)\), \((0,2,0)\) y \((0,0,3)\):
\[
\frac{x}{1} + \frac{y}{2} + \frac{z}{3} = 1 \Rightarrow z = 3\left(1 - x - \frac{y}{2}\right).
\]

La región base en el plano \(xy\) está acotada por:
\[
x \geq 0, \quad y \geq 0, \quad x + \frac{y}{2} \leq 1.
\]

El volumen es:
\[
\begin{aligned}
V &= \int_0^1 \int_0^{2(1-x)} \int_0^{3(1-x-y/2)} dz\, dy\, dx \\
&= \int_0^1 \int_0^{2(1-x)} 3\left(1 - x - \frac{y}{2}\right) dy\, dx \\
&= \int_0^1 3 \left[ (1-x)y - \frac{y^2}{4} \right]_0^{2(1-x)} dx \\
&= \int_0^1 3 \left( 2(1-x)^2 - \frac{4(1-x)^2}{4} \right) dx \\
&= \int_0^1 3(1-x)^2 dx \\
&= 3 \left[ \frac{(1-x)^3}{-3} \right]_0^1 \\
&= 3 \cdot \frac{1}{3} = \boxed{1}.
\end{aligned}
\]
\end{EjercicioBox}

\begin{EjercicioBox}[Ejemplo 9: Masa de un sólido con densidad variable]
Calcular la masa de un sólido \(V\) dado por:
\[
V = \{(x,y,z) : 0 \leq x \leq 1, 0 \leq y \leq 2, 0 \leq z \leq 3\},
\]
con densidad \(\rho(x,y,z) = xyz\).

\textbf{Solución:}

\[
\begin{aligned}
M &= \iiint_V \rho(x,y,z)\, dV = \int_0^1 \int_0^2 \int_0^3 xyz\, dz\, dy\, dx \\
&= \int_0^1 \int_0^2 xy \left[ \frac{z^2}{2} \right]_0^3 dy\, dx \\
&= \int_0^1 \int_0^2 xy \cdot \frac{9}{2} dy\, dx \\
&= \frac{9}{2} \int_0^1 x \left[ \frac{y^2}{2} \right]_0^2 dx \\
&= \frac{9}{2} \int_0^1 2x\, dx \\
&= 9 \left[ \frac{x^2}{2} \right]_0^1 = \boxed{\frac{9}{2}}.
\end{aligned}
\]
\end{EjercicioBox}

\subsubsection{Integrales triples en coordenadas cilíndricas}

\begin{TemaBox}[Coordenadas cilíndricas]
Las coordenadas cilíndricas son útiles para regiones con simetría circular alrededor de un eje. La transformación es:
\[
x = r\cos\theta, \quad y = r\sin\theta, \quad z = z, \quad dV = r\, dr\, d\theta\, dz.
\]

La integral triple en coordenadas cilíndricas es:
\[
\iiint_V f(x,y,z)\, dV = \iiint_S f(r\cos\theta, r\sin\theta, z) r\, dr\, d\theta\, dz,
\]
donde \(S\) es la región correspondiente en coordenadas cilíndricas.
\end{TemaBox}

\begin{EjercicioBox}[Ejemplo 10: Volumen usando coordenadas cilíndricas]
Calcular el volumen del sólido limitado por el paraboloide \(z = x^2 + y^2\) y el plano \(z = 4\).

\textbf{Solución:}

En coordenadas cilíndricas, el paraboloide es \(z = r^2\). La región sólida es:
\[
S = \{(r,\theta,z) : 0 \leq r \leq 2, 0 \leq \theta \leq 2\pi, r^2 \leq z \leq 4\}.
\]

\[
\begin{aligned}
V &= \int_0^{2\pi} \int_0^2 \int_{r^2}^4 r\, dz\, dr\, d\theta \\
&= \int_0^{2\pi} d\theta \int_0^2 r(4 - r^2) dr \\
&= 2\pi \int_0^2 (4r - r^3) dr \\
&= 2\pi \left[ 2r^2 - \frac{r^4}{4} \right]_0^2 \\
&= 2\pi (8 - 4) = \boxed{8\pi}.
\end{aligned}
\]
\end{EjercicioBox}

\subsubsection{Regiones sólidas tipo II y tipo III}

\paragraph{Región sólida tipo II.}
Una región sólida tipo II está acotada por planos o superficies que definen límites en \(x\):
\[
\iiint_V f(x,y,z)\, dV = \iint_D \int_{h_1(y,z)}^{h_2(y,z)} f(x,y,z)\, dx\, dA,
\]
donde \(D\) es la proyección de \(V\) sobre el plano \(yz\).

\paragraph{Región sólida tipo III.}
Una región sólida tipo III está acotada por planos o superficies que definen límites en \(y\):
\[
\iiint_V f(x,y,z)\, dV = \iint_D \int_{k_1(x,z)}^{k_2(x,z)} f(x,y,z)\, dy\, dA,
\]
donde \(D\) es la proyección de \(V\) sobre el plano \(xz\).

\begin{EjercicioBox}[Ejemplo 9b: Volumen de un sólido entre dos superficies]
Calcular el volumen del sólido acotado por \(z = 0\), \(z = \sqrt{x^2 + y^2}\) y el cilindro \(x^2 + y^2 = 4\).

\textbf{Solución:}

Usando coordenadas cilíndricas, el sólido está descrito por:
\[
S = \{(r,\theta,z) : 0 \leq r \leq 2, 0 \leq \theta \leq 2\pi, 0 \leq z \leq r\}.
\]

\[
\begin{aligned}
V &= \int_0^{2\pi} \int_0^2 \int_0^r r\, dz\, dr\, d\theta \\
&= \int_0^{2\pi} d\theta \int_0^2 r^2 dr \\
&= 2\pi \left[ \frac{r^3}{3} \right]_0^2 \\
&= 2\pi \cdot \frac{8}{3} = \boxed{\frac{16\pi}{3}}.
\end{aligned}
\]
\end{EjercicioBox}

\subsubsection{Integrales triples en coordenadas esféricas}

\begin{TemaBox}[Coordenadas esféricas]
Para regiones con simetría esférica, las coordenadas esféricas son ideales. La transformación es:
\[
x = \rho\sin\phi\cos\theta, \quad y = \rho\sin\phi\sin\theta, \quad z = \rho\cos\phi,
\]
con \(dV = \rho^2\sin\phi\, d\rho\, d\phi\, d\theta\).

\begin{itemize}
  \item \(\rho \geq 0\): distancia desde el origen
  \item \(0 \leq \phi \leq \pi\): ángulo desde el eje \(z\) positivo (colatitud)
  \item \(0 \leq \theta \leq 2\pi\): ángulo en el plano \(xy\) desde el eje \(x\) positivo (azimut)
\end{itemize}
\end{TemaBox}

\begin{EjercicioBox}[Ejemplo 11: Volumen de una esfera usando coordenadas esféricas]
Calcular el volumen de la esfera \(x^2 + y^2 + z^2 \leq R^2\).

\textbf{Solución:}

En coordenadas esféricas, la esfera es:
\[
S = \{(\rho,\phi,\theta) : 0 \leq \rho \leq R, 0 \leq \phi \leq \pi, 0 \leq \theta \leq 2\pi\}.
\]

\[
\begin{aligned}
V &= \int_0^{2\pi} \int_0^{\pi} \int_0^R \rho^2\sin\phi\, d\rho\, d\phi\, d\theta \\
&= \int_0^{2\pi} d\theta \int_0^{\pi} \sin\phi\, d\phi \int_0^R \rho^2 d\rho \\
&= 2\pi \cdot [-\cos\phi]_0^{\pi} \cdot \left[ \frac{\rho^3}{3} \right]_0^R \\
&= 2\pi \cdot 2 \cdot \frac{R^3}{3} \\
&= \boxed{\frac{4\pi R^3}{3}}.
\end{aligned}
\]
\end{EjercicioBox}

\begin{EjercicioBox}[Ejemplo 11b: Masa de una esfera con densidad radial]
Calcular la masa de una esfera de radio \(R\) con densidad \(\rho = k\rho\) (densidad proporcional a la distancia desde el centro), donde \(\rho\) es la coordenada esférica.

\textbf{Solución:}

En coordenadas esféricas, la densidad es \(\rho(\rho) = k\rho\). La masa es:
\[
\begin{aligned}
M &= \iiint_V k\rho \cdot \rho^2\sin\phi\, d\rho\, d\phi\, d\theta \\
&= k \int_0^{2\pi} \int_0^{\pi} \int_0^R \rho^3\sin\phi\, d\rho\, d\phi\, d\theta \\
&= k \int_0^{2\pi} d\theta \int_0^{\pi} \sin\phi\, d\phi \int_0^R \rho^3 d\rho \\
&= k \cdot 2\pi \cdot [-\cos\phi]_0^{\pi} \cdot \left[ \frac{\rho^4}{4} \right]_0^R \\
&= k \cdot 2\pi \cdot 2 \cdot \frac{R^4}{4} \\
&= \boxed{\pi k R^4}.
\end{aligned}
\]
\end{EjercicioBox}

\paragraph{Elección entre coordenadas cilíndricas y esféricas.}
\begin{itemize}
  \item \textbf{Coordenadas cilíndricas:} Útiles para sólidos con simetría alrededor de un eje (cilindros, paraboloides, conos)
  \item \textbf{Coordenadas esféricas:} Útiles para sólidos con simetría alrededor de un punto (esferas, hemisferios)
  \item \textbf{Coordenadas cartesianas:} Útiles cuando los límites son rectas o planos paralelos a los ejes coordenados
\end{itemize}


\subsection{Cambio de variable en integrales múltiples}

\begin{TemaBox}[Cambio de variables: Transformación de regiones]
El cambio de variables permite transformar integrales complicadas sobre regiones complejas en integrales más simples sobre regiones rectangulares o con formas conocidas. El elemento clave es el \textbf{Jacobiano}, que mide cómo se transforma el área (o volumen) bajo la transformación.
\end{TemaBox}

\paragraph{Teorema del cambio de variables.}
Sea \(T: (u,v) \mapsto (x,y)\) una transformación uno a uno de clase \(C^1\) que mapea una región \(S\) en el plano \(uv\) a una región \(R\) en el plano \(xy\). Entonces:
\[
\iint_R f(x,y)\, dA = \iint_S f(x(u,v), y(u,v)) |J(u,v)|\, dA_{uv},
\]
donde el \textbf{Jacobiano} es:
\[
J(u,v) = \frac{\partial(x,y)}{\partial(u,v)} = 
\begin{vmatrix}
\dfrac{\partial x}{\partial u} & \dfrac{\partial x}{\partial v} \\
\dfrac{\partial y}{\partial u} & \dfrac{\partial y}{\partial v}
\end{vmatrix}.
\]

\begin{InfoBox}
\Meta{Interpretación}{El valor absoluto del Jacobiano \(|J|\) representa el factor por el cual el área se amplía o contrae bajo la transformación.}

\Meta{Casos especiales}{Los siguientes casos son comunes:}
\begin{itemize}
  \item Coordenadas polares: \(x = r\cos\theta, y = r\sin\theta\) → \(|J| = r\)
  \item Transformación lineal: \(|J|\) es constante (determinante de la matriz de transformación)
\end{itemize}
\end{InfoBox}

\begin{EjercicioBox}[Ejemplo 12: Cambio de variable lineal]
Calcular
\[
\iint_R (x + y)\, dA,
\]
donde \(R\) es la región acotada por \(x - y = 0\), \(x - y = 2\), \(x + y = 0\) y \(x + y = 4\).

\textbf{Solución:}

Usamos la transformación:
\[
u = x - y, \quad v = x + y.
\]

El Jacobiano:
\[
J = \frac{\partial(x,y)}{\partial(u,v)} = 
\begin{vmatrix}
\frac{1}{2} & \frac{1}{2} \\
-\frac{1}{2} & \frac{1}{2}
\end{vmatrix}
= \frac{1}{2} \cdot \frac{1}{2} - \left(-\frac{1}{2} \cdot \frac{1}{2}\right) = \frac{1}{2}.
\]

La región transformada es:
\[
S = \{(u,v) : 0 \leq u \leq 2, 0 \leq v \leq 4\}.
\]

Además, \(x + y = v\), por lo que:
\[
\begin{aligned}
\iint_R (x + y)\, dA &= \iint_S v \cdot \frac{1}{2} \, du\, dv \\
&= \frac{1}{2} \int_0^4 \int_0^2 v\, du\, dv \\
&= \frac{1}{2} \int_0^4 2v\, dv \\
&= \int_0^4 v\, dv \\
&= \left[ \frac{v^2}{2} \right]_0^4 = \boxed{8}.
\end{aligned}
\]
\end{EjercicioBox}

\begin{EjercicioBox}[Ejemplo 13: Cambio de variable a coordenadas polares generalizadas]
Calcular
\[
\iint_R e^{(x^2+y^2)/a^2}\, dA,
\]
donde \(R\) es la región elíptica \(\frac{x^2}{a^2} + \frac{y^2}{b^2} \leq 1\).

\textbf{Solución:}

Usamos el cambio:
\[
x = ar\cos\theta, \quad y = br\sin\theta,
\]
que transforma la elipse en un círculo unitario.

El Jacobiano:
\[
J = 
\begin{vmatrix}
a\cos\theta & -ar\sin\theta \\
b\sin\theta & br\cos\theta
\end{vmatrix}
= abr(\cos^2\theta + \sin^2\theta) = abr.
\]

La región transformada:
\[
S = \{(r,\theta) : 0 \leq r \leq 1, 0 \leq \theta \leq 2\pi\}.
\]

\[
\begin{aligned}
\iint_R e^{(x^2+y^2)/a^2}\, dA &= \int_0^{2\pi} \int_0^1 e^{r^2(\cos^2\theta + (b^2/a^2)\sin^2\theta)} abr\, dr\, d\theta.
\end{aligned}
\]

Para \(a = b\), esto se simplifica a:
\[
\begin{aligned}
\iint_R e^{(x^2+y^2)/a^2}\, dA &= \int_0^{2\pi} \int_0^1 e^{r^2} a^2 r\, dr\, d\theta \\
&= 2\pi a^2 \left[ \frac{1}{2} e^{r^2} \right]_0^1 \\
&= \boxed{\pi a^2(e - 1)}.
\end{aligned}
\]
\end{EjercicioBox}

\paragraph{Interpretación geométrica del Jacobiano.}
El Jacobiano \(|J(u,v)|\) representa el factor de escala del área bajo la transformación. Si \(|J| > 1\), el área se expande; si \(|J| < 1\), el área se contrae. Esto es crucial para mantener la igualdad en el cambio de variables.

\paragraph{Propiedades del Jacobiano.}
\begin{itemize}
  \item Si \(T\) es una transformación lineal, el Jacobiano es constante (el determinante de la matriz de transformación)
  \item El Jacobiano del producto de transformaciones es el producto de los Jacobianos
  \item Para coordenadas polares, cilíndricas y esféricas, el Jacobiano ya está incluido en las fórmulas estándar
\end{itemize}

\paragraph{Cambio de variables en integrales triples.}
Para una transformación \(T: (u,v,w) \mapsto (x,y,z)\), el teorema del cambio de variables es:
\[
\iiint_V f(x,y,z)\, dV = \iiint_S f(x(u,v,w), y(u,v,w), z(u,v,w)) |J(u,v,w)|\, dV_{uvw},
\]
donde el Jacobiano es:
\[
J(u,v,w) = \frac{\partial(x,y,z)}{\partial(u,v,w)} = 
\begin{vmatrix}
\dfrac{\partial x}{\partial u} & \dfrac{\partial x}{\partial v} & \dfrac{\partial x}{\partial w} \\
\dfrac{\partial y}{\partial u} & \dfrac{\partial y}{\partial v} & \dfrac{\partial y}{\partial w} \\
\dfrac{\partial z}{\partial u} & \dfrac{\partial z}{\partial v} & \dfrac{\partial z}{\partial w}
\end{vmatrix}.
\]

\paragraph{Jacobianos de transformaciones comunes.}
\begin{itemize}
  \item \textbf{Coordenadas cilíndricas:} \(|J| = r\)
  \item \textbf{Coordenadas esféricas:} \(|J| = \rho^2\sin\phi\)
  \item \textbf{Transformación de escala:} \(x = au, y = bv, z = cw\) → \(|J| = abc\)
\end{itemize}

\begin{EjercicioBox}[Ejemplo 13b: Cambio de escala en integral triple]
Calcular el volumen del elipsoide \(\frac{x^2}{a^2} + \frac{y^2}{b^2} + \frac{z^2}{c^2} \leq 1\) usando cambio de variables.

\textbf{Solución:}

Usamos la transformación:
\[
x = au, \quad y = bv, \quad z = cw.
\]

El Jacobiano es:
\[
J = 
\begin{vmatrix}
a & 0 & 0 \\
0 & b & 0 \\
0 & 0 & c
\end{vmatrix}
= abc.
\]

El elipsoide se transforma en la esfera unitaria \(u^2 + v^2 + w^2 \leq 1\). El volumen es:
\[
\begin{aligned}
V &= \iiint_V dV = \iiint_S abc\, dV_{uvw} \\
&= abc \cdot \text{Volumen de la esfera unitaria} \\
&= abc \cdot \frac{4\pi}{3} = \boxed{\frac{4\pi abc}{3}}.
\end{aligned}
\]
\end{EjercicioBox}


\subsection{Aplicaciones de integrales múltiples}

\begin{TemaBox}[Aplicaciones prácticas de integrales múltiples]
Las integrales múltiples tienen innumerables aplicaciones en física, ingeniería, economía y otras áreas. Permiten calcular cantidades distribuidas sobre regiones planas y sólidas, resolver problemas de optimización y modelar fenómenos complejos.
\end{TemaBox}

\subsubsection{Volumen de sólidos}

\paragraph{Método general.}
El volumen de un sólido acotado superiormente por \(z = f(x,y)\) e inferiormente por \(z = g(x,y)\) sobre una región \(D\) en el plano \(xy\) es:
\[
V = \iint_D [f(x,y) - g(x,y)]\, dA.
\]

\begin{EjercicioBox}[Ejemplo 14: Volumen entre dos superficies]
Calcular el volumen del sólido acotado por los paraboloides \(z = x^2 + y^2\) y \(z = 8 - x^2 - y^2\).

\textbf{Solución:}

Los paraboloides se intersectan cuando:
\[
x^2 + y^2 = 8 - x^2 - y^2 \Rightarrow x^2 + y^2 = 4.
\]

La región base es el círculo \(x^2 + y^2 \leq 4\). El volumen es:
\[
\begin{aligned}
V &= \iint_D [(8 - x^2 - y^2) - (x^2 + y^2)]\, dA \\
&= \iint_D (8 - 2x^2 - 2y^2)\, dA.
\end{aligned}
\]

Usando coordenadas polares:
\[
\begin{aligned}
V &= \int_0^{2\pi} \int_0^2 (8 - 2r^2) r\, dr\, d\theta \\
&= 2\pi \int_0^2 (8r - 2r^3) dr \\
&= 2\pi \left[ 4r^2 - \frac{r^4}{2} \right]_0^2 \\
&= 2\pi (16 - 8) = \boxed{16\pi}.
\end{aligned}
\]
\end{EjercicioBox}

\subsubsection{Masa y centro de masa}

\paragraph{Masa de una lámina.}
Para una lámina con densidad \(\rho(x,y)\), la masa es:
\[
M = \iint_R \rho(x,y)\, dA.
\]

\paragraph{Centro de masa.}
Las coordenadas del centro de masa \((\bar{x}, \bar{y})\) son:
\[
\bar{x} = \frac{M_y}{M} = \frac{1}{M} \iint_R x\rho(x,y)\, dA, \quad
\bar{y} = \frac{M_x}{M} = \frac{1}{M} \iint_R y\rho(x,y)\, dA,
\]
donde \(M_x\) y \(M_y\) son los momentos respecto a los ejes \(x\) e \(y\).

\begin{EjercicioBox}[Ejemplo 15: Centro de masa de una lámina triangular]
Encontrar el centro de masa de una lámina triangular con vértices \((0,0)\), \((2,0)\) y \((2,3)\), con densidad constante \(\rho(x,y) = k\).

\textbf{Solución:}

Primero calculamos la masa:
\[
M = k \iint_R dA = k \cdot \text{Área} = k \cdot \frac{1}{2} \cdot 2 \cdot 3 = 3k.
\]

Los momentos:
\[
\begin{aligned}
M_y &= \int_0^2 \int_0^{\frac{3}{2}x} kx\, dy\, dx \\
&= k \int_0^2 x \cdot \frac{3x}{2} dx \\
&= \frac{3k}{2} \int_0^2 x^2 dx = \frac{3k}{2} \cdot \frac{8}{3} = 4k.
\end{aligned}
\]

\[
\begin{aligned}
M_x &= \int_0^2 \int_0^{\frac{3}{2}x} ky\, dy\, dx \\
&= k \int_0^2 \left[ \frac{y^2}{2} \right]_0^{\frac{3}{2}x} dx \\
&= k \int_0^2 \frac{9x^2}{8} dx \\
&= \frac{9k}{8} \cdot \frac{8}{3} = 3k.
\end{aligned}
\]

El centro de masa:
\[
\bar{x} = \frac{4k}{3k} = \frac{4}{3}, \quad \bar{y} = \frac{3k}{3k} = 1.
\]

\[
\boxed{\left(\frac{4}{3}, 1\right)}
\]
\end{EjercicioBox}

\subsubsection{Momentos de inercia}

\paragraph{Definición.}
Para una lámina con densidad \(\rho(x,y)\), el momento de inercia respecto al eje \(x\) es:
\[
I_x = \iint_R y^2\rho(x,y)\, dA,
\]
y respecto al eje \(y\):
\[
I_y = \iint_R x^2\rho(x,y)\, dA.
\]

El momento de inercia respecto al origen (momento polar) es:
\[
I_0 = I_x + I_y = \iint_R (x^2 + y^2)\rho(x,y)\, dA.
\]

\begin{EjercicioBox}[Ejemplo 16: Momento de inercia de un disco]
Calcular el momento de inercia respecto al origen de un disco circular de radio \(R\) con densidad constante \(\rho = k\).

\textbf{Solución:}

Usando coordenadas polares:
\[
\begin{aligned}
I_0 &= \iint_R (x^2 + y^2)k\, dA = k \int_0^{2\pi} \int_0^R r^2 \cdot r\, dr\, d\theta \\
&= k \int_0^{2\pi} d\theta \int_0^R r^3 dr \\
&= k \cdot 2\pi \cdot \left[ \frac{r^4}{4} \right]_0^R \\
&= 2\pi k \cdot \frac{R^4}{4} = \boxed{\frac{\pi k R^4}{2}}.
\end{aligned}
\]
\end{EjercicioBox}

\subsubsection{Carga eléctrica y otras aplicaciones}

\paragraph{Carga eléctrica.}
Para una distribución de carga con densidad \(\sigma(x,y)\) sobre una superficie, la carga total es:
\[
Q = \iint_R \sigma(x,y)\, dA.
\]

\paragraph{Valor promedio.}
El valor promedio de una función \(f(x,y)\) sobre una región \(R\) es:
\[
f_{\text{prom}} = \frac{1}{\text{Área}(R)} \iint_R f(x,y)\, dA.
\]

\begin{EjercicioBox}[Ejemplo 17: Temperatura promedio en una placa]
Una placa circular de radio \(a\) tiene temperatura \(T(x,y) = 100 - x^2 - y^2\) (en grados Celsius). Calcular la temperatura promedio de la placa.

\textbf{Solución:}

El área de la placa es \(A = \pi a^2\). La temperatura promedio es:
\[
\begin{aligned}
T_{\text{prom}} &= \frac{1}{\pi a^2} \iint_R (100 - x^2 - y^2)\, dA \\
&= \frac{1}{\pi a^2} \int_0^{2\pi} \int_0^a (100 - r^2) r\, dr\, d\theta \\
&= \frac{1}{\pi a^2} \int_0^{2\pi} d\theta \int_0^a (100r - r^3) dr \\
&= \frac{1}{\pi a^2} \cdot 2\pi \left[ 50r^2 - \frac{r^4}{4} \right]_0^a \\
&= \frac{2}{a^2} \left( 50a^2 - \frac{a^4}{4} \right) \\
&= 100 - \frac{a^2}{2} = \boxed{100 - \frac{a^2}{2}} \text{°C}.
\end{aligned}
\]
\end{EjercicioBox}

\subsubsection{Masa y centro de masa de sólidos}

\paragraph{Masa de un sólido.}
Para un sólido con densidad \(\rho(x,y,z)\), la masa es:
\[
M = \iiint_V \rho(x,y,z)\, dV.
\]

\paragraph{Centro de masa de un sólido.}
Las coordenadas del centro de masa \((\bar{x}, \bar{y}, \bar{z})\) son:
\[
\bar{x} = \frac{M_{yz}}{M} = \frac{1}{M} \iiint_V x\rho(x,y,z)\, dV,
\]
\[
\bar{y} = \frac{M_{xz}}{M} = \frac{1}{M} \iiint_V y\rho(x,y,z)\, dV,
\]
\[
\bar{z} = \frac{M_{xy}}{M} = \frac{1}{M} \iiint_V z\rho(x,y,z)\, dV,
\]
donde \(M_{yz}\), \(M_{xz}\) y \(M_{xy}\) son los momentos respecto a los planos correspondientes.

\begin{EjercicioBox}[Ejemplo 18: Centro de masa de un sólido homogéneo]
Calcular el centro de masa del sólido limitado por el paraboloide \(z = 4 - x^2 - y^2\) y el plano \(z = 0\), suponiendo densidad constante \(\rho = k\).

\textbf{Solución:}

La proyección en el plano \(xy\) es el círculo \(x^2 + y^2 = 4\), es decir, \(0 \leq r \leq 2\).

Primero calculamos la masa:
\[
\begin{aligned}
M &= k \iiint_V dV = k \int_0^{2\pi} \int_0^2 \int_0^{4-r^2} r\, dz\, dr\, d\theta \\
&= k \int_0^{2\pi} d\theta \int_0^2 r(4 - r^2) dr \\
&= 2\pi k \int_0^2 (4r - r^3) dr \\
&= 2\pi k \left[ 2r^2 - \frac{r^4}{4} \right]_0^2 \\
&= 2\pi k (8 - 4) = 8\pi k.
\end{aligned}
\]

Por simetría, \(\bar{x} = \bar{y} = 0\). Calculamos \(\bar{z}\):
\[
\begin{aligned}
M_{xy} &= k \iiint_V z\, dV = k \int_0^{2\pi} \int_0^2 \int_0^{4-r^2} zr\, dz\, dr\, d\theta \\
&= k \int_0^{2\pi} d\theta \int_0^2 r \left[ \frac{z^2}{2} \right]_0^{4-r^2} dr \\
&= 2\pi k \int_0^2 \frac{r(4-r^2)^2}{2} dr \\
&= \pi k \int_0^2 r(16 - 8r^2 + r^4) dr \\
&= \pi k \int_0^2 (16r - 8r^3 + r^5) dr \\
&= \pi k \left[ 8r^2 - 2r^4 + \frac{r^6}{6} \right]_0^2 \\
&= \pi k \left( 32 - 32 + \frac{64}{6} \right) = \frac{32\pi k}{3}.
\end{aligned}
\]

Por lo tanto:
\[
\bar{z} = \frac{M_{xy}}{M} = \frac{32\pi k/3}{8\pi k} = \frac{4}{3}.
\]

\[
\boxed{\left(0, 0, \frac{4}{3}\right)}
\]
\end{EjercicioBox}

\subsubsection{Aplicaciones a probabilidad}

\paragraph{Funciones de densidad de probabilidad conjunta.}
Si \(X\) e \(Y\) son variables aleatorias continuas con función de densidad conjunta \(f(x,y)\), entonces:

\begin{itemize}
  \item \textbf{Probabilidad sobre una región:} \(P((X,Y) \in R) = \iint_R f(x,y)\, dA\)
  \item \textbf{Normalización:} \(\iint_{\mathbb{R}^2} f(x,y)\, dA = 1\)
  \item \textbf{Valor esperado:} \(E[g(X,Y)] = \iint_{\mathbb{R}^2} g(x,y) f(x,y)\, dA\)
\end{itemize}

\begin{EjercicioBox}[Ejemplo 19: Probabilidad con distribución uniforme]
Sea \(f(x,y) = k\) una función de densidad constante sobre la región triangular con vértices \((0,0)\), \((2,0)\) y \((0,2)\). Encontrar \(k\) y calcular \(P(X + Y \leq 1)\).

\textbf{Solución:}

Primero encontramos \(k\) usando la condición de normalización:
\[
1 = \iint_R k\, dA = k \cdot \text{Área} = k \cdot \frac{1}{2} \cdot 2 \cdot 2 = 2k \Rightarrow k = \frac{1}{2}.
\]

Para calcular \(P(X + Y \leq 1)\), la región es:
\[
R' = \{(x,y) : 0 \leq x \leq 1, 0 \leq y \leq 1 - x\}.
\]

\[
P(X + Y \leq 1) = \int_0^1 \int_0^{1-x} \frac{1}{2} dy\, dx = \frac{1}{2} \int_0^1 (1-x) dx = \frac{1}{2} \left[ x - \frac{x^2}{2} \right]_0^1 = \boxed{\frac{1}{4}}.
\]
\end{EjercicioBox}

\subsubsection{Momentos de inercia para sólidos}

\paragraph{Definición.}
Para un sólido con densidad \(\rho(x,y,z)\), los momentos de inercia son:
\[
I_x = \iiint_V (y^2 + z^2)\rho(x,y,z)\, dV,
\]
\[
I_y = \iiint_V (x^2 + z^2)\rho(x,y,z)\, dV,
\]
\[
I_z = \iiint_V (x^2 + y^2)\rho(x,y,z)\, dV.
\]

\begin{EjercicioBox}[Ejemplo 20: Momento de inercia de un sólido esférico]
Calcular el momento de inercia respecto al eje \(z\) de una esfera sólida de radio \(R\) con densidad constante \(\rho = k\).

\textbf{Solución:}

En coordenadas esféricas, \(x^2 + y^2 = \rho^2\sin^2\phi\). El momento de inercia es:
\[
\begin{aligned}
I_z &= k \iiint_V (x^2 + y^2)\, dV \\
&= k \int_0^{2\pi} \int_0^{\pi} \int_0^R \rho^2\sin^2\phi \cdot \rho^2\sin\phi\, d\rho\, d\phi\, d\theta \\
&= k \int_0^{2\pi} d\theta \int_0^{\pi} \sin^3\phi\, d\phi \int_0^R \rho^4 d\rho.
\end{aligned}
\]

Calculando por partes:
\[
\int_0^{\pi} \sin^3\phi\, d\phi = \int_0^{\pi} \sin\phi(1 - \cos^2\phi)\, d\phi = \left[-\cos\phi + \frac{\cos^3\phi}{3}\right]_0^{\pi} = \frac{4}{3}.
\]

\[
I_z = k \cdot 2\pi \cdot \frac{4}{3} \cdot \frac{R^5}{5} = \boxed{\frac{8\pi k R^5}{15}}.
\]
\end{EjercicioBox}

\paragraph{Aplicaciones en física e ingeniería.}
Las integrales múltiples tienen aplicaciones extensas en:
\begin{itemize}
  \item \textbf{Física:} Cálculo de campos gravitacionales, distribución de carga, momento de inercia de cuerpos rígidos, potencial eléctrico
  \item \textbf{Ingeniería:} Análisis estructural, transferencia de calor, dinámica de fluidos, análisis de tensiones
  \item \textbf{Economía:} Modelado de distribución de recursos, análisis de probabilidad conjunta, optimización de portafolios
  \item \textbf{Probabilidad:} Funciones de densidad de probabilidad conjunta, valores esperados, covarianza
  \item \textbf{Geometría:} Cálculo de áreas y volúmenes de regiones complejas, superficies de revolución
  \item \textbf{Estadística:} Análisis multivariable, regresión múltiple, funciones de verosimilitud
\end{itemize}

\subsubsection{Aplicaciones adicionales}

\paragraph{Área de superficies.}
El área de una superficie \(z = f(x,y)\) sobre una región \(R\) es:
\[
A = \iint_R \sqrt{1 + \left(\frac{\partial f}{\partial x}\right)^2 + \left(\frac{\partial f}{\partial y}\right)^2}\, dA.
\]

\begin{EjercicioBox}[Ejemplo 22: Área de una superficie]
Calcular el área de la parte del paraboloide \(z = x^2 + y^2\) que está dentro del cilindro \(x^2 + y^2 = 4\).

\textbf{Solución:}

Primero calculamos las derivadas parciales:
\[
\frac{\partial z}{\partial x} = 2x, \quad \frac{\partial z}{\partial y} = 2y.
\]

El elemento de área de superficie es:
\[
\sqrt{1 + (2x)^2 + (2y)^2} = \sqrt{1 + 4(x^2 + y^2)} = \sqrt{1 + 4r^2}.
\]

Usando coordenadas polares:
\[
\begin{aligned}
A &= \int_0^{2\pi} \int_0^2 \sqrt{1 + 4r^2} \cdot r\, dr\, d\theta \\
&= 2\pi \int_0^2 r\sqrt{1 + 4r^2} dr.
\end{aligned}
\]

Haciendo \(u = 1 + 4r^2\), \(du = 8r\, dr\):
\[
A = 2\pi \int_1^{17} \frac{1}{8}\sqrt{u} du = \frac{\pi}{4} \left[\frac{2u^{3/2}}{3}\right]_1^{17} = \boxed{\frac{\pi}{6}(17\sqrt{17} - 1)}.
\]
\end{EjercicioBox}

\paragraph{Valor promedio en el espacio.}
El valor promedio de una función \(f(x,y,z)\) sobre una región sólida \(V\) es:
\[
f_{\text{prom}} = \frac{1}{\text{Volumen}(V)} \iiint_V f(x,y,z)\, dV.
\]

\begin{EjercicioBox}[Ejemplo 23: Densidad promedio de un sólido]
Un sólido tiene la forma de un cubo \([0,2] \times [0,2] \times [0,2]\) con densidad \(\rho(x,y,z) = xyz\). Calcular la densidad promedio.

\textbf{Solución:}

El volumen es \(V = 8\). La masa total es:
\[
M = \int_0^2 \int_0^2 \int_0^2 xyz\, dz\, dy\, dx = \left(\int_0^2 x\, dx\right)\left(\int_0^2 y\, dy\right)\left(\int_0^2 z\, dz\right) = 2 \cdot 2 \cdot 2 = 8.
\]

La densidad promedio es:
\[
\rho_{\text{prom}} = \frac{M}{V} = \frac{8}{8} = \boxed{1}.
\]
\end{EjercicioBox}

\paragraph{Resumen de fórmulas importantes.}

\begin{center}
\begin{tabular}{|l|l|}
\hline
\textbf{Concepto} & \textbf{Fórmula} \\
\hline
Volumen bajo superficie & \(V = \iint_R f(x,y)\, dA\) \\
\hline
Volumen entre superficies & \(V = \iint_D [f(x,y) - g(x,y)]\, dA\) \\
\hline
Área de una región & \(A = \iint_R 1\, dA\) \\
\hline
Área de superficie & \(A = \iint_R \sqrt{1 + f_x^2 + f_y^2}\, dA\) \\
\hline
Masa de lámina & \(M = \iint_R \rho(x,y)\, dA\) \\
\hline
Centro de masa (lámina) & \((\bar{x}, \bar{y}) = \left(\frac{M_y}{M}, \frac{M_x}{M}\right)\) \\
\hline
Momentos (lámina) & \(M_x = \iint_R y\rho\, dA\), \(M_y = \iint_R x\rho\, dA\) \\
\hline
Momento de inercia (lámina) & \(I_0 = \iint_R (x^2 + y^2)\rho(x,y)\, dA\) \\
\hline
Masa de sólido & \(M = \iiint_V \rho(x,y,z)\, dV\) \\
\hline
Centro de masa (sólido) & \((\bar{x}, \bar{y}, \bar{z}) = \left(\frac{M_{yz}}{M}, \frac{M_{xz}}{M}, \frac{M_{xy}}{M}\right)\) \\
\hline
Valor promedio (2D) & \(f_{\text{prom}} = \frac{1}{A(R)}\iint_R f(x,y)\, dA\) \\
\hline
Valor promedio (3D) & \(f_{\text{prom}} = \frac{1}{V}\iiint_V f(x,y,z)\, dV\) \\
\hline
Coordenadas polares & \(dA = r\, dr\, d\theta\) \\
\hline
Coordenadas cilíndricas & \(dV = r\, dr\, d\theta\, dz\), \(|J| = r\) \\
\hline
Coordenadas esféricas & \(dV = \rho^2\sin\phi\, d\rho\, d\phi\, d\theta\), \(|J| = \rho^2\sin\phi\) \\
\hline
Cambio de variables (2D) & \(dA = |J(u,v)|\, du\, dv\) \\
\hline
Cambio de variables (3D) & \(dV = |J(u,v,w)|\, du\, dv\, dw\) \\
\hline
Probabilidad & \(P((X,Y) \in R) = \iint_R f(x,y)\, dA\) \\
\hline
Valor esperado & \(E[g(X,Y)] = \iint_{\mathbb{R}^2} g(x,y) f(x,y)\, dA\) \\
\hline
\end{tabular}
\end{center}

\subsubsection{Integrales dobles impropias}

\paragraph{Definición.}
Las integrales dobles pueden ser impropias cuando:
\begin{enumerate}
  \item La región de integración es no acotada
  \item La función tiene discontinuidades en la región
\end{enumerate}

Para regiones no acotadas, definimos:
\[
\iint_R f(x,y)\, dA = \lim_{n \to \infty} \iint_{R_n} f(x,y)\, dA,
\]
donde \(\{R_n\}\) es una sucesión creciente de regiones acotadas que cubren \(R\).

\begin{EjercicioBox}[Ejemplo 21: Integral doble impropia]
Calcular
\[
\iint_{\mathbb{R}^2} e^{-(x^2+y^2)}\, dA.
\]

\textbf{Solución:}

Usando coordenadas polares:
\[
\begin{aligned}
\iint_{\mathbb{R}^2} e^{-(x^2+y^2)}\, dA &= \lim_{R \to \infty} \int_0^{2\pi} \int_0^R e^{-r^2} r\, dr\, d\theta \\
&= 2\pi \lim_{R \to \infty} \int_0^R re^{-r^2} dr \\
&= 2\pi \lim_{R \to \infty} \left[-\frac{1}{2}e^{-r^2}\right]_0^R \\
&= 2\pi \cdot \frac{1}{2} = \boxed{\pi}.
\end{aligned}
\]

Este resultado es importante en probabilidad y estadística (distribución normal).
\end{EjercicioBox}

\subsubsection{Técnicas de integración numérica}

\paragraph{Integración aproximada.}
Cuando las integrales no pueden resolverse analíticamente, se usan métodos numéricos:
\begin{itemize}
  \item \textbf{Regla del punto medio:} Aproxima la función por constantes en subrectángulos
  \item \textbf{Regla de Simpson:} Extensión bidimensional de la regla de Simpson
  \item \textbf{Cuadratura de Gauss:} Usa puntos y pesos óptimos para mayor precisión
  \item \textbf{Monte Carlo:} Método probabilístico útil para dimensiones altas
\end{itemize}

\paragraph{Consideraciones prácticas.}
\begin{itemize}
  \item \textbf{Elección del orden de integración:} A veces es más fácil integrar primero respecto a una variable. Considera los límites de integración para elegir el orden óptimo. Si un límite depende de otra variable, integra primero respecto a la variable que tiene límites constantes.
  \item \textbf{Simetría:} Aprovecha la simetría de la región y la función para simplificar cálculos (por ejemplo, si \(f(-x,y) = f(x,y)\), puedes calcular sobre la mitad y duplicar). Si la región es simétrica y la función tiene cierta paridad, la integral puede ser cero o simplificarse.
  \item \textbf{Sistema de coordenadas:} Usa coordenadas polares para círculos, cilíndricas para cilindros y esféricas para esferas. El cambio de coordenadas puede convertir una integral difícil en una simple.
  \item \textbf{Cambio de variables:} Cuando la región es complicada, busca una transformación que la simplifique a una región rectangular o conocida. Identifica simetrías en la región que sugieran transformaciones útiles.
  \item \textbf{Factorización:} Si la función es producto de funciones de una sola variable, puedes separar las integrales: \(\iint_R f(x)g(y)\, dA = \left(\int_a^b f(x)\, dx\right)\left(\int_c^d g(y)\, dy\right)\).
  \item \textbf{Integrales iteradas:} Resuelve paso a paso, evaluando primero la integral interior antes de integrar respecto a la variable exterior.
\end{itemize}

\paragraph{Errores comunes a evitar.}
\begin{enumerate}
  \item Olvidar el factor \(r\) en coordenadas polares: \(dA = r\, dr\, d\theta\), no \(dr\, d\theta\).
  \item Confundir el orden de los límites de integración: el límite interior debe estar en términos de la variable exterior.
  \item No verificar que el Jacobiano en cambio de variables sea positivo o tomar su valor absoluto.
  \item No ajustar los límites de integración al cambiar el orden de integración.
  \item Olvidar incluir la densidad en cálculos de masa cuando no es constante.
\end{enumerate}