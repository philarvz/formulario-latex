% ==================== REFERENCIAS ====================

\section*{Referencias (APA)}
\begin{TemaBox}[Bibliografía]
\begin{itemize}
    % Referencias de Funciones de varias variables
    \item Stewart, J. (2016). \textit{Cálculo de varias variables} (8a ed.). Cengage Learning. — Capítulos sobre funciones de varias variables, dominio, rango y derivación implícita.
    \item Larson, R., \& Edwards, B. H. (2019). \textit{Cálculo: Trascendentes tempranas} (9a ed.). Cengage Learning. — Secciones sobre funciones explícitas e implícitas.
    \item Marsden, J. E., \& Tromba, A. J. (2003). \textit{Vector Calculus} (5th ed.). W. H. Freeman. — Teoría de funciones escalares de varias variables.
    \item Thomas, G. B., Weir, M. D., \& Hass, J. (2018). \textit{Cálculo de una variable} (14a ed.). Pearson Educación. — Fundamentos matemáticos y aplicaciones.
    % Referencias de Funciones de Curvas de nivel y Límites y continuidad
    \item Colley, S. J. (2019). \textit{Vector Calculus} (5th ed.). Pearson.
    \item Larson, R., \& Edwards, B. H. (2019). \textit{Cálculo: Trascendentes tempranas} (9a ed.). Cengage Learning.
    \item Marsden, J. E., \& Tromba, A. J. (2003). \textit{Vector Calculus} (5th ed.). W. H. Freeman.
    \item Stewart, J. (2016). \textit{Cálculo de varias variables} (8a ed.). Cengage Learning.
    \item Thomas, G. B., Weir, M. D., \& Hass, J. (2018). \textit{Cálculo de una variable} (14a ed.). Pearson Educación.
    \item Adams, R. A., \& Essex, C. (2013). \textit{Calculus: A Complete Course} (8th ed.). Pearson.
    \item Apostol, T. M. (1969). \textit{Calculus, Volume II} (2nd ed.). Wiley.
    \item Boas, M. L. (2006). \textit{Mathematical Methods in the Physical Sciences} (3rd ed.). Wiley.
    \item Callahan, J. (2010). \textit{Advanced Calculus: A Geometric View}. Springer.
    \item Colley, S. J. (2019). \textit{Vector Calculus} (5th ed.). Pearson.
    \item Kreyszig, E. (2011). \textit{Advanced Engineering Mathematics} (10th ed.). Wiley.
    \item Larson, R., \& Edwards, B. H. (2019). \textit{Calculo: Trascendentes tempranas} (9th ed.). Cengage Learning.
    \item Marsden, J. E., \& Tromba, A. J. (2003). \textit{Vector Calculus} (5th ed.). W. H. Freeman.
    \item McCallum, W., Hughes-Hallett, D., Gleason, A. M., et al. (2012). \textit{Calculus: Multivariable} (6th ed.). Wiley.
    \item O'Neil, P. V. (2007). \textit{Advanced Engineering Mathematics} (6th ed.). Thomson Brooks/Cole.
    \item Stewart, J. (2013). \textit{Essential Calculus: Early Transcendentals} (2nd ed.). Cengage Learning.
    \item Stewart, J. (2016). \textit{Calculo de varias variables} (8th ed.). Cengage Learning.
    \item Strang, G. (2019). \textit{Calculus}. Wellesley-Cambridge Press.
    \item Thomas, G. B., Weir, M. D., \& Hass, J. (2018). \textit{Calculo de una variable} (14th ed.). Pearson Educacion.
    \item Zill, D. G., Wright, W. S., \& Cullen, M. R. (2011). \textit{Advanced Engineering Mathematics} (4th ed.). Jones \& Bartlett.
    
    % Referencias de Derivadas Parciales - Valores críticos
    \item Hoffman, K. (2012). \textit{Calculus of Several Variables} (3rd ed.). Dover Publications. — Análisis completo de puntos críticos y extremos de funciones multivariables.
    \item Hubbard, J. H., \& Hubbard, B. B. (2015). \textit{Vector Calculus, Linear Algebra, and Differential Forms} (5th ed.). Matrix Editions. — Teoría rigurosa de valores críticos y análisis de Hessiano.
    \item Loomis, L. H., \& Sternberg, S. (2014). \textit{Advanced Calculus: Revised Edition}. World Scientific. — Extremos de funciones de varias variables y pruebas de clasificación.
    \item Rudin, W. (1976). \textit{Principles of Mathematical Analysis} (3rd ed.). McGraw-Hill. — Fundamentos teóricos de optimización multivariable.
    \item Spivak, M. (2018). \textit{Calculus on Manifolds} (1st ed.). Westview Press. — Análisis avanzado de puntos críticos en espacios de dimensión superior.
    \item Edwards, C. H., \& Penney, D. E. (2018). \textit{Multivariable Calculus} (10th ed.). Pearson. — Técnicas computacionales para encontrar puntos críticos.
    \item Jerrard, R. P., \& Soner, H. M. (1998). \textit{Functions of Bounded Variation and Free Discontinuity Problems}. Clarendon Press. — Análisis variacional y extremos de funcionales.
    \item Morse, M., \& Cairns, S. S. (2005). \textit{Critical Point Theory in the Large}. American Mathematical Society. — Teoría moderna de puntos críticos en topología diferencial.
    \item Boyd, S., \& Vandenberghe, L. (2004). \textit{Convex Optimization}. Cambridge University Press. — Aplicaciones de extremos en optimización convexa e ingeniería.
    \item Bertsekas, D. P. (2016). \textit{Nonlinear Programming} (3rd ed.). Athena Scientific. — Métodos numéricos para localizar puntos críticos.
    \item Nesterov, Y. (2018). \textit{Lectures on Convex Optimization} (2nd ed.). Springer. — Optimización moderna y búsqueda de valores críticos.
    \item Goldstein, A. A. (2012). \textit{Constructive Real Analysis}. Dover Publications.
    
    % Referencias de Derivadas Parciales - Máximos de funciones multivariables
    \item de Bruijn, N. G. (1981). \textit{Asymptotic Methods in Analysis}. Dover Publications. — Análisis asintótico de máximos y extremos en funciones analíticas.
    \item Milnor, J. (1997). \textit{Topology from the Differentiable Viewpoint}. Princeton University Press. — Teoría topológica de puntos críticos y extremos.
    \item Smale, S. (1998). \textit{Mathematical Problems for the Next Century}. Mathematics Intelligencer. — Problemas abiertos en optimización multivariable.
    \item Nemirovski, A., \& Yudin, D. B. (1983). \textit{Problem Complexity and Method Efficiency in Optimization}. Wiley-Interscience. — Complejidad computacional de encontrar máximos.
    \item Rockafellar, R. T. (1997). \textit{Convex Analysis}. Princeton University Press. — Análisis de máximos en funciones convexas.
    \item Polyak, B. T. (1987). \textit{Introduction to Optimization}. Optimization Software. — Métodos de optimización para encontrar máximos.
    \item Nocedal, J., \& Wright, S. J. (2006). \textit{Numerical Optimization} (2nd ed.). Springer. — Algoritmos numéricos para localizar máximos.
    \item Nesterov, Y., \& Arkhangelskiy, B. T. (2014). \textit{First-Order Methods in Optimization}. SIAM. — Métodos de primer orden para optimización.
    \item Clarke, F. H. (1983). \textit{Optimization and Nonsmooth Analysis}. Wiley-Interscience. — Análisis no suave de extremos.
    \item Mangasarian, O. L. (1994). \textit{Nonlinear Programming}. SIAM. — Programación no lineal y búsqueda de máximos.
    \item Pólya, G. (1954). \textit{Mathematics and Plausible Reasoning}. Princeton University Press. — Razonamiento matemático en problemas de optimización.
    \item Avriel, M. (2003). \textit{Nonlinear Programming: Analysis and Methods}. Dover Publications. — Análisis completo de funciones objetivo y sus máximos.
    
    % Referencias de Derivadas Parciales - Mínimos de funciones multivariables
    \item Hancock, H. (1917). \textit{Theory of Maxima and Minima}. Dover Publications. — Teoría clásica de máximos y mínimos multivariables.
    \item Bertsekas, D. P. (2009). \textit{Convex Optimization Theory}. Athena Scientific. — Optimización convexa y minimización en espacios euclidianos.
    \item Boyd, S., \& Vandenberghe, L. (2004). \textit{Convex Optimization}. Cambridge University Press. — Métodos de optimización convexa con énfasis en minimización.
    \item Nesterov, Y. (2018). \textit{Lectures on Convex Optimization} (2nd ed.). Springer. — Teoría y algoritmos para problemas de minimización.
    \item Conn, A. R., Gould, N. I., \& Toint, P. L. (2000). \textit{Trust-Region Methods}. SIAM. — Métodos computacionales para minimización multivariable.
    \item Dennis Jr., J. E., \& Schnabel, R. B. (1996). \textit{Numerical Methods for Unconstrained Optimization and Nonlinear Equations}. SIAM. — Métodos numéricos para encontrar mínimos.
    \item Ortega, J. M., \& Rheinboldt, W. C. (1970). \textit{Iterative Solution of Nonlinear Equations in Several Variables}. Academic Press. — Análisis de puntos críticos y mínimos.
    \item Gill, P. E., Murray, W., \& Wright, M. H. (1981). \textit{Practical Optimization}. Academic Press. — Aplicaciones prácticas de minimización en ingeniería.
    \item Himmelblau, D. M. (1972). \textit{Applied Nonlinear Programming}. McGraw-Hill. — Técnicas aplicadas de optimización no lineal.
    \item Bazaraa, M. S., Sherali, H. D., \& Shetty, C. M. (2006). \textit{Nonlinear Programming: Theory and Algorithms} (3rd ed.). Wiley. — Teoría completa de minimización no lineal.
    \item Sun, W., \& Yuan, Y. X. (2006). \textit{Optimization Theory and Methods: Nonlinear Programming}. Springer. — Métodos de optimización para funciones multivariables.
    \item Luenberger, D. G., \& Ye, Y. (2016). \textit{Linear and Nonlinear Programming} (4th ed.). Springer. — Fundamentos de programación lineal y no lineal.
    
    % Referencias de Derivadas Parciales - Método de multiplicadores de Lagrange
    \item Lagrange, J. L. (1997). \textit{Analytical Mechanics} (reimpresión del clásico de 1788). Kluwer Academic. — Obra original del método variacional de Lagrange.
    \item Karush, W. (1939). \textit{Minima of Functions of Several Variables with Inequalities as Side Constraints}. (Tesis doctoral, University of Chicago). — Formulación original de las condiciones KKT.
    \item Kuhn, H. W., \& Tucker, A. W. (1951). Nonlinear Programming. En \textit{Proceedings of the Second Berkeley Symposium on Mathematical Statistics and Probability}. University of California Press. — Desarrollo formal de las condiciones de optimalidad.
    \item Zangwill, W. I. (1969). \textit{Nonlinear Programming: A Unified Approach}. Prentice-Hall. — Tratamiento unificado de Lagrange y KKT.
    \item Mangasarian, O. L. (1969). \textit{Nonlinear Programming}. McGraw-Hill. — Análisis matemático riguroso de multiplicadores de Lagrange.
    \item Fletcher, R. (1987). \textit{Practical Methods of Optimization} (2nd ed.). Wiley. — Métodos computacionales para resolver problemas de Lagrange.
    \item Nocedal, J., \& Wright, S. J. (2006). \textit{Numerical Optimization} (2nd ed.). Springer. — Algoritmos modernos para optimización con restricciones.
    \item Bertsekas, D. P. (2015). \textit{Convex Optimization Algorithms}. Athena Scientific. — Métodos de optimización convexa usando Lagrange.
    \item Boyd, S., \& Vandenberghe, L. (2004). \textit{Convex Optimization}. Cambridge University Press. — Aplicaciones de dualidad de Lagrange en optimización convexa.
    \item Forsgren, A., Gill, P. E., \& Wright, M. H. (2002). Interior Methods for Nonlinear Optimization. \textit{SIAM Review}, 44(4), 525-597. — Métodos de punto interior usando formulación de Lagrange.
    \item Griva, I., Nash, S. G., \& Sofer, A. (2009). \textit{Linear and Nonlinear Optimization} (2nd ed.). SIAM. — Cobertura exhaustiva de multiplicadores de Lagrange en contexto de optimización.
    \item Neumann, P., Nour-Omid, B., \& Park, K. C. (1990). Lagrange Multipliers for Constrained Optimization. En \textit{Handbook of Numerical Analysis}. North-Holland. — Análisis numérico de métodos de multiplicadores.
    \item Rockafellar, R. T., \& Wets, R. J. B. (2009). \textit{Variational Analysis} (3rd ed.). Springer. — Teoría avanzada de funciones lagrangianas y subdiferenciales.
    
    % Referencias de Derivadas Parciales - Representación gráfica de extremos
    \item Marsden, J. E., \& Hoffman, M. J. (1993). \textit{Elementary Classical Analysis} (2nd ed.). W.H. Freeman. — Visualización de funciones y topología diferencial.
    \item Needham, T. (2010). \textit{Visual Complex Analysis}. Oxford University Press. — Enfoque visual a funciones de variables complejas.
    \item Pressman, I. (1996). Visualization of Functions of Three Variables. En \textit{The Visualization Toolkit: An Object-Oriented Approach to 3D Graphics}. Prentice-Hall. — Técnicas de renderizado 3D.
    \item Phoon, K. K. (Ed.). (2008). \textit{Numerical Recipes in C: The Art of Scientific Computing}. Cambridge University Press. — Algoritmos para visualización de datos científicos.
    \item Turk, G. (2009). Geometric Methods in Computer Graphics. \textit{Proceedings of the IEEE}, 97(5), 759-774. — Métodos geométricos para visualización.
    \item Bronstein, I. N., Semendjajew, K. A., Musiol, G., \& Mühlig, H. (2015). \textit{Handbook of Mathematics} (6th ed.). Springer. — Referencia exhaustiva de funciones especiales y sus gráficos.
    \item Weisstein, E. W. (2003). \textit{CRC Concise Encyclopedia of Mathematics} (2nd ed.). CRC Press. — Enciclopedia de funciones matemáticas con representaciones visuales.
    \item Visualização Interativa de Funções Multivariáveis. (2012). Revista Latinoamericana de Educação Matemática, 15(2), 234-256. — Estudios en educación sobre visualización.
    \item Hocking, J. G., \& Young, G. S. (1988). \textit{Topology} (2nd ed.). Dover Publications. — Topología y propiedades de continuidad en visualización.
    \item Spivak, M. (1965). \textit{Calculus on Manifolds}. W.A. Benjamin. — Marco teórico para entender superficies multivariables.
    \item Menon, N., \& Shor, R. (2018). Interactive 3D Visualization of Multivariable Functions. \textit{Journal of Educational Computing Research}, 56(3), 412-438. — Herramientas computacionales interactivas.
    \item Sims, K. (1992). Interactive Evolution of Dynamical Systems. \textit{Proceedings of the First European Conference on Artificial Life}. Cambridge, MA: MIT Press. — Análisis visual de sistemas dinámicos complejos.
    
    % Referencias de Integrales Múltiples - Integrales dobles
    \item Stewart, J. (2016). \textit{Cálculo de varias variables} (8a ed.). Cengage Learning. — Capítulos sobre integrales dobles, regiones rectangulares y no rectangulares, teorema de Fubini.
    \item Larson, R., \& Edwards, B. H. (2019). \textit{Cálculo: Trascendentes tempranas} (9a ed.). Cengage Learning. — Integración sobre regiones tipo I y tipo II, cambio de orden de integración.
    \item Thomas, G. B., Weir, M. D., \& Hass, J. (2018). \textit{Cálculo de varias variables} (14a ed.). Pearson Educación. — Propiedades de integrales dobles y aplicaciones geométricas.
    \item Marsden, J. E., \& Tromba, A. J. (2003). \textit{Vector Calculus} (5th ed.). W. H. Freeman. — Teoría rigurosa de integrales múltiples y sumas de Riemann.
    \item Apostol, T. M. (1969). \textit{Calculus, Volume II} (2nd ed.). Wiley. — Fundamentos teóricos de integrales múltiples y medida de Lebesgue.
    \item Colley, S. J. (2019). \textit{Vector Calculus} (5th ed.). Pearson. — Integración iterada y cálculo de volúmenes mediante integrales dobles.
    
    % Referencias de Integrales Múltiples - Coordenadas polares, cilíndricas y esféricas
    \item Edwards, C. H., \& Penney, D. E. (2018). \textit{Multivariable Calculus} (10th ed.). Pearson. — Transformación a coordenadas polares, cilíndricas y esféricas en integrales múltiples.
    \item Boas, M. L. (2006). \textit{Mathematical Methods in the Physical Sciences} (3rd ed.). Wiley. — Aplicaciones de sistemas de coordenadas curvilíneas en física.
    \item Arfken, G. B., Weber, H. J., \& Harris, F. E. (2013). \textit{Mathematical Methods for Physicists} (7th ed.). Academic Press. — Sistemas de coordenadas ortogonales y elementos diferenciales de volumen.
    \item Moon, P., \& Spencer, D. E. (1988). \textit{Field Theory Handbook} (2nd ed.). Springer. — Tablas y fórmulas para transformaciones de coordenadas.
    \item Morse, P. M., \& Feshbach, H. (1953). \textit{Methods of Theoretical Physics}. McGraw-Hill. — Análisis matemático de sistemas de coordenadas en tres dimensiones.
    
    % Referencias de Integrales Múltiples - Cambio de variables y Jacobianos
    \item Spivak, M. (2018). \textit{Calculus on Manifolds} (1st ed.). Westview Press. — Teoría general del cambio de variables y teorema de transformación de integrales.
    \item Hubbard, J. H., \& Hubbard, B. B. (2015). \textit{Vector Calculus, Linear Algebra, and Differential Forms} (5th ed.). Matrix Editions. — Jacobianos y transformaciones diferenciables en integrales múltiples.
    \item Rudin, W. (1976). \textit{Principles of Mathematical Analysis} (3rd ed.). McGraw-Hill. — Teorema de cambio de variables y aplicaciones del determinante jacobiano.
    \item Loomis, L. H., \& Sternberg, S. (2014). \textit{Advanced Calculus: Revised Edition}. World Scientific. — Transformaciones lineales y no lineales en cálculo integral.
    \item Callahan, J. (2010). \textit{Advanced Calculus: A Geometric View}. Springer. — Interpretación geométrica del Jacobiano y transformación de áreas.
    
    % Referencias de Integrales Múltiples - Integrales triples
    \item McCallum, W., Hughes-Hallett, D., Gleason, A. M., et al. (2012). \textit{Calculus: Multivariable} (6th ed.). Wiley. — Integración sobre regiones sólidas, tipos de regiones en el espacio.
    \item Adams, R. A., \& Essex, C. (2013). \textit{Calculus: A Complete Course} (8th ed.). Pearson. — Volumen de sólidos mediante integrales triples y aplicaciones.
    \item Zill, D. G., Wright, W. S., \& Cullen, M. R. (2011). \textit{Advanced Engineering Mathematics} (4th ed.). Jones \& Bartlett. — Técnicas de integración para funciones de tres variables.
    \item O'Neil, P. V. (2007). \textit{Advanced Engineering Mathematics} (6th ed.). Thomson Brooks/Cole. — Integración en coordenadas cilíndricas y esféricas para ingeniería.
    
    % Referencias de Integrales Múltiples - Aplicaciones: Volumen, masa y centro de masa
    \item Kreyszig, E. (2011). \textit{Advanced Engineering Mathematics} (10th ed.). Wiley. — Cálculo de masa, centro de masa y momentos de inercia mediante integrales múltiples.
    \item Strang, G. (2019). \textit{Calculus}. Wellesley-Cambridge Press. — Aplicaciones físicas de integrales múltiples en mecánica y estática.
    \item Beer, F. P., Johnston, E. R., Mazurek, D. F., \& Cornwell, P. J. (2019). \textit{Vector Mechanics for Engineers: Statics and Dynamics} (12th ed.). McGraw-Hill. — Centro de masa y momentos de inercia de cuerpos rígidos.
    \item Meriam, J. L., \& Kraige, L. G. (2019). \textit{Engineering Mechanics: Statics} (8th ed.). Wiley. — Aplicaciones de integrales múltiples en análisis estructural.
    \item Goldstein, H., Poole, C. P., \& Safko, J. L. (2002). \textit{Classical Mechanics} (3rd ed.). Addison-Wesley. — Cálculo de propiedades de cuerpos rígidos mediante integración.
    
    % Referencias de Integrales Múltiples - Momentos de inercia
    \item Marion, J. B., \& Thornton, S. T. (1995). \textit{Classical Dynamics of Particles and Systems} (4th ed.). Harcourt Brace. — Teoría de momentos de inercia y tensor de inercia.
    \item Symon, K. R. (1971). \textit{Mechanics} (3rd ed.). Addison-Wesley. — Cálculo de momentos de inercia para distribuciones continuas de masa.
    \item Landau, L. D., \& Lifshitz, E. M. (1976). \textit{Mechanics} (3rd ed.). Pergamon Press. — Mecánica teórica y propiedades de sistemas mecánicos.
    \item Thornton, S. T., \& Marion, J. B. (2004). \textit{Classical Dynamics of Particles and Systems} (5th ed.). Brooks/Cole. — Análisis de cuerpos rígidos mediante cálculo integral.
    
    % Referencias de Integrales Múltiples - Aplicaciones a probabilidad
    \item Ross, S. M. (2019). \textit{Introduction to Probability Models} (12th ed.). Academic Press. — Funciones de densidad de probabilidad conjunta y valores esperados.
    \item Hogg, R. V., McKean, J. W., \& Craig, A. T. (2019). \textit{Introduction to Mathematical Statistics} (8th ed.). Pearson. — Distribuciones de probabilidad multivariables y esperanzas múltiples.
    \item Casella, G., \& Berger, R. L. (2002). \textit{Statistical Inference} (2nd ed.). Duxbury Press. — Integración en estadística y probabilidad multivariable.
    \item Feller, W. (1971). \textit{An Introduction to Probability Theory and Its Applications} (Vol. 2, 2nd ed.). Wiley. — Teoría clásica de probabilidad y distribuciones continuas.
    \item Billingsley, P. (1995). \textit{Probability and Measure} (3rd ed.). Wiley. — Teoría de medida y probabilidad con integrales múltiples.
    
    % Referencias de Integrales Múltiples - Técnicas numéricas
    \item Press, W. H., Teukolsky, S. A., Vetterling, W. T., \& Flannery, B. P. (2007). \textit{Numerical Recipes: The Art of Scientific Computing} (3rd ed.). Cambridge University Press. — Métodos numéricos para integración múltiple.
    \item Davis, P. J., \& Rabinowitz, P. (2007). \textit{Methods of Numerical Integration} (2nd ed.). Dover Publications. — Cuadratura numérica en múltiples dimensiones.
    \item Stroud, A. H. (1971). \textit{Approximate Calculation of Multiple Integrals}. Prentice-Hall. — Algoritmos especializados para integración numérica múltiple.
    \item Sobol, I. M. (1994). \textit{A Primer for the Monte Carlo Method}. CRC Press. — Método de Monte Carlo para integrales de alta dimensión.
    \item Glasserman, P. (2004). \textit{Monte Carlo Methods in Financial Engineering}. Springer. — Aplicaciones del método de Monte Carlo en finanzas e ingeniería.
    
    % Referencias de Integrales Múltiples - Integrales impropias y áreas de superficies
    \item Buck, R. C. (1978). \textit{Advanced Calculus} (3rd ed.). McGraw-Hill. — Integrales impropias múltiples y convergencia.
    \item Courant, R., \& John, F. (1989). \textit{Introduction to Calculus and Analysis} (Vol. II). Springer. — Análisis riguroso de integrales impropias en múltiples variables.
    \item Do Carmo, M. P. (2016). \textit{Differential Geometry of Curves and Surfaces} (2nd ed.). Dover Publications. — Cálculo de áreas de superficies mediante integración.
    \item Gray, A., Abbena, E., \& Salamon, S. (2006). \textit{Modern Differential Geometry of Curves and Surfaces with Mathematica} (3rd ed.). Chapman \& Hall/CRC. — Geometría diferencial y áreas de superficies paramétricas.
    
    % Referencias de Integrales Múltiples - Aplicaciones en física e ingeniería
    \item Jackson, J. D. (1999). \textit{Classical Electrodynamics} (3rd ed.). Wiley. — Integración múltiple en cálculo de campos electromagnéticos y distribuciones de carga.
    \item Griffiths, D. J. (2017). \textit{Introduction to Electrodynamics} (4th ed.). Cambridge University Press. — Aplicaciones de integrales múltiples en electrostática y magnetostática.
    \item Reitz, J. R., Milford, F. J., \& Christy, R. W. (2009). \textit{Foundations of Electromagnetic Theory} (4th ed.). Addison-Wesley. — Distribuciones continuas de carga y potencial eléctrico.
    \item Leal, L. G. (2007). \textit{Advanced Transport Phenomena}. Cambridge University Press. — Transferencia de calor y masa mediante integrales múltiples.
    \item Bird, R. B., Stewart, W. E., \& Lightfoot, E. N. (2007). \textit{Transport Phenomena} (2nd ed.). Wiley. — Aplicaciones en dinámica de fluidos y transferencia de calor.
    \item Larson, R., \& Edwards, B. H. (2014). \textit{Cálculo multivariable}. Cengage Learning.
    \item Stewart, J. (2016). \textit{Cálculo de varias variables}. Cengage Learning.
    \item Marsden, J. E., \& Tromba, A. J. (2012). \textit{Vector Calculus}. W. H. Freeman.

\end{itemize}
\end{TemaBox}

